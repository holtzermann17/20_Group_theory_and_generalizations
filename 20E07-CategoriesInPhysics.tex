\documentclass[12pt]{article}
\usepackage{pmmeta}
\pmcanonicalname{CategoriesInPhysics}
\pmcreated{2013-03-22 19:19:35}
\pmmodified{2013-03-22 19:19:35}
\pmowner{bci1}{20947}
\pmmodifier{bci1}{20947}
\pmtitle{categories in physics}
\pmrecord{10}{42266}
\pmprivacy{1}
\pmauthor{bci1}{20947}
\pmtype{Topic}
\pmcomment{trigger rebuild}
\pmclassification{msc}{20E07}
\pmclassification{msc}{18-01}
\pmrelated{CategoricalDynamics}

\endmetadata

% this is the default PlanetMath preamble. as your knowledge
% of TeX increases, you will probably want to edit this, but
\usepackage{amsmath, amssymb, amsfonts, amsthm, amscd, latexsym}
%%\usepackage{xypic}
\usepackage[mathscr]{eucal}
% define commands here
\theoremstyle{plain}
\newtheorem{lemma}{Lemma}[section]
\newtheorem{proposition}{Proposition}[section]
\newtheorem{theorem}{Theorem}[section]
\newtheorem{corollary}{Corollary}[section]
\theoremstyle{definition}
\newtheorem{definition}{Definition}[section]
\newtheorem{example}{Example}[section]
%\theoremstyle{remark}
\newtheorem{remark}{Remark}[section]
\newtheorem*{notation}{Notation}
\newtheorem*{claim}{Claim}
\renewcommand{\thefootnote}{\ensuremath{\fnsymbol{footnote%%@
}}}
\numberwithin{equation}{section}
\newcommand{\Ad}{{\rm Ad}}
\newcommand{\Aut}{{\rm Aut}}
\newcommand{\Cl}{{\rm Cl}}
\newcommand{\Co}{{\rm Co}}
\newcommand{\DES}{{\rm DES}}
\newcommand{\Diff}{{\rm Diff}}
\newcommand{\Dom}{{\rm Dom}}
\newcommand{\Hol}{{\rm Hol}}
\newcommand{\Mon}{{\rm Mon}}
\newcommand{\Hom}{{\rm Hom}}
\newcommand{\Ker}{{\rm Ker}}
\newcommand{\Ind}{{\rm Ind}}
\newcommand{\IM}{{\rm Im}}
\newcommand{\Is}{{\rm Is}}
\newcommand{\ID}{{\rm id}}
\newcommand{\GL}{{\rm GL}}
\newcommand{\Iso}{{\rm Iso}}
\newcommand{\Sem}{{\rm Sem}}
\newcommand{\St}{{\rm St}}
\newcommand{\Sym}{{\rm Sym}}
\newcommand{\SU}{{\rm SU}}
\newcommand{\Tor}{{\rm Tor}}
\newcommand{\U}{{\rm U}}
\newcommand{\A}{\mathcal A}
\newcommand{\Ce}{\mathcal C}
\newcommand{\D}{\mathcal D}
\newcommand{\E}{\mathcal E}
\newcommand{\F}{\mathcal F}
\newcommand{\G}{\mathcal G}
\newcommand{\Q}{\mathcal Q}
\newcommand{\R}{\mathcal R}
\newcommand{\cS}{\mathcal S}
\newcommand{\cU}{\mathcal U}
\newcommand{\W}{\mathcal W}
\newcommand{\bA}{\mathbb{A}}
\newcommand{\bB}{\mathbb{B}}
\newcommand{\bC}{\mathbb{C}}
\newcommand{\bD}{\mathbb{D}}
\newcommand{\bE}{\mathbb{E}}
\newcommand{\bF}{\mathbb{F}}
\newcommand{\bG}{\mathbb{G}}
\newcommand{\bK}{\mathbb{K}}
\newcommand{\bM}{\mathbb{M}}
\newcommand{\bN}{\mathbb{N}}
\newcommand{\bO}{\mathbb{O}}
\newcommand{\bP}{\mathbb{P}}
\newcommand{\bR}{\mathbb{R}}
\newcommand{\bV}{\mathbb{V}}
\newcommand{\bZ}{\mathbb{Z}}
\newcommand{\bfE}{\mathbf{E}}
\newcommand{\bfX}{\mathbf{X}}
\newcommand{\bfY}{\mathbf{Y}}
\newcommand{\bfZ}{\mathbf{Z}}
\renewcommand{\O}{\Omega}
\renewcommand{\o}{\omega}
\newcommand{\vp}{\varphi}
\newcommand{\vep}{\varepsilon}
\newcommand{\diag}{{\rm diag}}
\newcommand{\grp}{{\mathbb G}}
\newcommand{\dgrp}{{\mathbb D}}
\newcommand{\desp}{{\mathbb D^{\rm{es}}}}
\newcommand{\Geod}{{\rm Geod}}
\newcommand{\geod}{{\rm geod}}
\newcommand{\hgr}{{\mathbb H}}
\newcommand{\mgr}{{\mathbb M}}
\newcommand{\ob}{{\rm Ob}}
\newcommand{\obg}{{\rm Ob(\mathbb G)}}
\newcommand{\obgp}{{\rm Ob(\mathbb G')}}
\newcommand{\obh}{{\rm Ob(\mathbb H)}}
\newcommand{\Osmooth}{{\Omega^{\infty}(X,*)}}
\newcommand{\ghomotop}{{\rho_2^{\square}}}
\newcommand{\gcalp}{{\mathbb G(\mathcal P)}}
\newcommand{\rf}{{R_{\mathcal F}}}
\newcommand{\glob}{{\rm glob}}
\newcommand{\loc}{{\rm loc}}
\newcommand{\TOP}{{\rm TOP}}
\newcommand{\wti}{\widetilde}
\newcommand{\what}{\widehat}
\renewcommand{\a}{\alpha}
\newcommand{\be}{\beta}
\newcommand{\ga}{\gamma}
\newcommand{\Ga}{\Gamma}
\newcommand{\de}{\delta}
\newcommand{\del}{\partial}
\newcommand{\ka}{\kappa}
\newcommand{\si}{\sigma}
\newcommand{\ta}{\tau}
\newcommand{\lra}{{\longrightarrow}}
\newcommand{\ra}{{\rightarrow}}
\newcommand{\rat}{{\rightarrowtail}}
\newcommand{\oset}[1]{\overset {#1}{\ra}}
\newcommand{\osetl}[1]{\overset {#1}{\lra}}
\newcommand{\hr}{{\hookrightarrow}}
\begin{document}
\section{Categories in Physics}

 This is a relatively new area in mathematical and theoretical physics that is concerned with category theory applications to physics, especially non-Abelian
categories and non-Abelian algebraic topology concepts and results in mathematical physics and physical mathematics. Applications range from QFT,
AQFT, non-Abelian gauge theories and Quantum Gravity to complex systems, 
\PMlinkname{categorical dynamics}{CategoricalDynamics}, 
\PMlinkname{complex categorical dynamics}{CategoricalDynamics2}, mathematical biophysics and relational biology. Other applications are related to graph theory approaches to Quantum Chemistry. 

\begin{thebibliography}{99}

\bibitem{BR et al-2011}
Brown, Ronald, Higgins, J.P. and Sivera, R. 2011. ``Nonabelian Algebraic Topology'', 690 pages,  EPS: London and Bruxelles. \PMlinkexternal{Nonabelian Algebraic Topology 2011}{http://pp-wiki.metameso.org/wiki.pl/NAAT_2011_Book_Summary};


\bibitem{BAJ-DJ98B}
Baez, J. and Dolan, J., 1998b, ``Categorification'', {\em Higher Category Theory, Contemporary Mathematics}, 230, Providence: AMS, 1--36.

\bibitem{Baianu-Marinescu68}
Baianu, I. and M. Marinescu: 1968, Organismic Supercategories: Towards a Unitary Theory of Systems. \emph{Bulletin of Mathematical Biophysics} \textbf{30}, 148-159.

\bibitem{Baianu70}
Baianu, I.C.: 1970, Organismic Supercategories: II. On Multistable Systems. \emph{Bulletin of Mathematical Biophysics}, \textbf{32}: 539-561.

\bibitem{Baianu71a}
Baianu,I.C.: 1971a, Organismic Supercategories and Qualitative Dynamics of Systems. \emph{Bulletin of Mathematical Biophysics}, \textbf{33} (3), 339--354.

\bibitem{Baianu71b}
Baianu, I.C.: 1971b, Categories, Functors and Quantum Algebraic Computations, in P. Suppes (ed.), In \emph{Proceed. Fourth Intl. Congress Logic-Mathematics-Philosophy of Science}, 14 pages, September 1--4, 1971. 

\bibitem{ICB04b}
Baianu, I.C. : \L ukasiewicz-Topos Models of Neural Networks, Cell Genome and Interactome Nonlinear Dynamics). CERN Preprint EXT-2004-059. \textit{Health Physics and Radiation Effects} (June 29, 2004).

\bibitem{ICBs5}
Baianu, I.C.  and D. Scripcariu: 1973, On Adjoint Dynamical Systems. \emph{The Bulletin of Mathematical Biophysics}, \textbf{35}(4), 475--486.

\bibitem{BAJ-DJ2k1}
Baez, J. \& Dolan, J., 2001, From Finite Sets to Feynman Diagrams,
in \emph{Mathematics Unlimited -- 2001 and Beyond}, Berlin: Springer, 29--50.

\bibitem{BAJ-DJ97}
Baez, J., 1997, An Introduction to n-Categories,
in \emph{Category Theory and Computer Science, Lecture Notes in Computer Science}, 1290, Berlin: Springer-Verlag, 1--33.

\bibitem{BBGG1}
Baianu I. C., Brown R., Georgescu G. and J. F. Glazebrook: 2006, Complex Nonlinear Biodynamics in Categories, Higher Dimensional Algebra and \L{}ukasiewicz-Moisil Topos: Transformations of Neuronal, Genetic and Neoplastic Networks.,
\emph{Axiomathes}, \textbf{16} Nos. 1-2: 65-122.

\bibitem{ICBDS73}
Baianu, I.C. and D. Scripcariu: 1973, On Adjoint Dynamical Systems.
\emph{Bulletin of Mathematical Biophysics}, \textbf{35}(4): 475-486.

\bibitem{Bgg2}
Baianu, I. C., Glazebrook, J. F. and G. Georgescu: 2004, Categories of Quantum Automata and N-Valued \L ukasiewicz Algebras in Relation to Dynamic Bionetworks, \textbf{(M,R)}-Systems and Their Higher Dimensional Algebra,
\emph{Preprint of Report}.

\bibitem{BHR2}
Brown, R., Higgins, P. J. and R. Sivera,: 2007, \emph{Non-Abelian Algebraic Topology},
\PMlinkexternal{vol.I pdf doc.}{http://www.bangor.ac.uk/~mas010/nonab-t/partI010604.pdf};
\PMlinkexternal{Review of Part I and full contents PDF doc.}{http://planetmath.org/?op=getobj&from=lec&id=75}

\bibitem{RB2k8}
R. Brown. 2008. {\em Higher Dimensional Algebra Preprint as pdf and ps docs. at arXiv:math/0212274v6 [math.AT]}

\bibitem{BP2k3}
Brown R. and T. Porter: 2003, Category theory and higher dimensional algebra: potential descriptive tools in neuroscience, In: {\em Proceedings of the International Conference on Theoretical Neurobiology}, Delhi, February 2003, edited by Nandini Singh, National Brain Research Centre, {\em Conference Proceedings} \textbf{1}: 80-92.


\bibitem{Chaician}
Chaician, M. and A. Demichev. 1996. Introduction to Quantum Groups, World Scientific .

\bibitem{CA94}
Connes, A. 1994. \emph{Noncommutative geometry}. Academic Press: New York.

\bibitem{CL93}
Crane, Louis. 1993. Categorical Physics., 9 pages, \PMlinkexternal{with free download at $arXiv:hep-th/9301061$}{http://arxiv.org/PS_cache/hep-th/pdf/9301/9301061v1.pdf} 
(``A new mathematical form for the quantum theory of gravity coupled to matter. The motivation is from the connection between CSW TQFT and the Ashtekar variables'').

\bibitem{CR-LL63}
Croisot, R. and Lesieur, L. 1963. \emph{Alg\`ebre noeth\'erienne non-commutative.}, Gauthier-Villard: Paris.

\bibitem{DJ-ALEX60-71}
Dieudonn\'e, J. and Grothendieck, A., 1960, [1971], {\em \'El\'ements de G\'eom\'etrie Alg\'ebrique}, Berlin: Springer-Verlag.

\bibitem{Dixmier}
Dixmier, J., 1981, Von Neumann Algebras, Amsterdam: North-Holland Publishing Company. [First published in French in 1957: Les Algebres d'Operateurs dans l'Espace Hilbertien, Paris: Gauthier--Villars.]

\bibitem{Durdevich1}
M. Durdevich : Geometry of quantum principal bundles I, Commun. Math. Phys. \textbf{175} (3) (1996), 457--521.

\bibitem{Durdevich2}
M. Durdevich : Geometry of quantum principal bundles II, Rev.Math. Phys. \textbf{9} (5) (1997), 531-607.

\bibitem{Eh-pseudo}
Ehresmann, C.: 1952, Structures locales et structures infinit\'esimales, \emph{C.R.A.S.} Paris \textbf{274}: 587-589.

\bibitem{Eh}
Ehresmann, C.: 1959, Cat\'egories topologiques et cat\'egories diff\'erentiables, \emph{Coll. G\'eom. Diff. Glob.} Bruxelles, pp.137-150.

\bibitem{EC}
Ehresmann, C.: 1966, Trends Toward Unity in Mathematics., \emph{Cahiers de Topologie et Geometrie Differentielle}
\textbf{8}: 1-7.

\bibitem{EML1}
Eilenberg, S. and S. Mac Lane.: 1942, Natural Isomorphisms in Group Theory., \emph{American Mathematical Society 43}: 757-831.

\bibitem{EL}
Eilenberg, S. and S. Mac Lane: 1945, The General Theory of Natural Equivalences, \emph{Transactions of the American Mathematical Society} \textbf{58}: 231-294.

\bibitem{ETH}
Ezawa,Z.F., G. Tsitsishvilli and K. Hasebe : Noncommutative geometry, extended $W_{\infty}$ algebra and Grassmannian solitons in multicomponent Hall systems, (at arXiv:hep--th/0209198).

\bibitem{GabrielNP}
Gabriel, P. and N. Popescu: 1964, Caract\'{e}risation des cat\'egories ab\'eliennes avec g\'{e}n\'{e}rateurs et limites inductives. , \emph{CRAS Paris} \textbf{258}: 4188-4191.

\bibitem{GA-RG-SM2k}
Galli, A. \& Reyes, G. \& Sagastume, M., 2000, Completeness Theorems via the Double Dual Functor, {\em Studia Logica}, \textbf{64}, no. 1: 61--81.

\bibitem{Alex3}
Grothendieck, A. and J. Dieudon\'{e}.: 1960, El\'{e}ments de geometrie alg\'{e}brique., \emph{Publ. Inst. des Hautes Etudes de Science}, \textbf{4}.

\bibitem{KDM58}
Kan, D. M., 1958, Adjoint Functors, {\em Transactions of the American Mathematical Society} 87, 294-329.

\bibitem{Krips}
H. Krips : Measurement in Quantum Theory, \emph{The Stanford Encyclopedia of Philosophy } (Winter 1999 Edition), Edward N. Zalta (ed.)

\bibitem{Land1}
Landsman, N. P. : Compact quantum groupoids, (at arXiv:math--ph/9912006).

\bibitem{LFW64}
Lawvere, F. W., 1964, An Elementary Theory of the Category of Sets, {\em Proceedings of the National Academy of Sciences U.S.A.}, 52, 1506--1511.

\bibitem{LFW65}
Lawvere, F. W., 1965, Algebraic Theories, Algebraic Categories, and Algebraic Functors, {\em Theory of Models}, Amsterdam: North Holland, 413--418.

\bibitem{LFW66}
Lawvere, F. W.: 1966, The Category of Categories as a Foundation for Mathematics., in \emph{Proc. Conf. Categorical Algebra- La Jolla}., Eilenberg, S. et al., eds. Springer--Verlag: Berlin, Heidelberg and New York., pp. 1-20.

\bibitem{LFW69b}
Lawvere, F. W., 1969b, Adjointness in Foundations, {\em Dialectica}, \textbf{23}: 281--295.

\bibitem{LFW92}
Lawvere, F. W., 1992, Categories of Space and of Quantity, {\em The Space of Mathematics, Foundations of Communication and Cognition}, Berlin: De Gruyter, 14-30.

\bibitem{LFW2k2}
Lawvere, F. W., 2002, Categorical Algebra for Continuum Micro-Physics, {\em Journal of Pure and Applied Algebra}, 175, no. 1--3, 267--287.

\bibitem{LiM-PV97}
Li, M. and P. Vitanyi: 1997, \emph{An introduction to Kolmogorov Complexity and its Applications}, Springer Verlag: New York.

\bibitem{Lofgren68}
L\"{o}fgren, L.: 1968, An Axiomatic Explanation of Complete Self-Reproduction, \emph{Bulletin of Mathematical Biophysics}, \textbf{30}: 317-348.

\bibitem{Mack1}
K. C. H. Mackenzie : {\em Lie Groupoids and Lie Algebroids in Differential Geometry}, LMS Lect. Notes \textbf{124}, Cambridge University Press, 1987

\bibitem{MCLSS48}
MacLane, S.: 1948. Groups, categories, and duality., {\em Proc. Natl. Acad. Sci.U.S.A}, \textbf{34}: 263-267.

\bibitem{MCLSS69}
MacLane, S., 1969, Foundations for Categories and Sets, in {\em Category Theory, Homology Theory and their Applications II}, Berlin: Springer, 146--164.

\bibitem{MCLS71}MacLane, S., 1971, Categorical algebra and Set-Theoretic Foundations, in {\em Axiomatic Set Theory}, Providence: AMS, 231--240.

\bibitem{MLS50}
MacLane, S., 1950, Dualities for Groups, {\em Bulletin of the American Mathematical Society}, 56, 485-516.

\bibitem{MCLS96}
MacLane, S., 1996, Structure in Mathematics. Mathematical Structuralism., {\em Philosophia Mathematica}, 4, 2, 174-183.

\bibitem{MCLS98}
MacLane, S., 1997, Categories for the Working Mathematician, 2nd edition, New York: Springer-Verlag.

\bibitem{Majid1}
Majid, S.: 1995, \emph{Foundations of Quantum Group Theory}, Cambridge Univ. Press: Cambridge, UK.

\bibitem{Majid2}
Majid, S.: 2002, \emph{A Quantum Groups Primer}, Cambridge Univ.Press: Cambridge, UK.

\bibitem{MJP1999}
May, J.P. 1999, \emph{A Concise Course in Algebraic Topology}, The University of Chicago Press: Chicago.

\bibitem{MLC94}
Mc Larty, C., 1994, Category Theory in Real Time, {\em Philosophia Mathematica}, \textbf{2}, no. 1, 36-44.

\bibitem{MLC91}
Mc Larty, C., 1991, Axiomatizing a Category of Categories, {\em Journal of Symbolic Logic}, 56, no. 4, 1243-1260.


\bibitem{Mitchell1}
Mitchell, B.: 1965, \emph{Theory of Categories}, Academic Press:London.

\bibitem{OO31}
Ore, O., 1931, Linear equations on non-commutative fields, {\em Ann. Math.} \textbf{32}: 463-477.


\bibitem{PLR}
Plymen, R.J. and P. L. Robinson: 1994, \emph{Spinors in Hilbert Space}, Cambridge Tracts in Math.
\textbf{114}, Cambridge Univ. Press, Cambridge.

\bibitem{NPop1}
Popescu, N.: 1973, \emph{Abelian Categories with Applications to Rings and Modules.} New York and London: Academic Press., 2nd edn. 1975. \emph{(English translation by I.C. Baianu)}.

\bibitem{PB70}
Pareigis, B., 1970, Categories and Functors, New York: Academic Press.

\bibitem{PMC2k4}
Pedicchio, M. C. and Tholen, W., 2004, Categorical Foundations, Cambridge: Cambridge University Press.

\bibitem{PB91}
Peirce, B., 1991, Basic Category Theory for Computer Scientists, Cambridge: MIT Press.

\bibitem{Pradines1966}
Pradines, J.: 1966, Th\'eorie de Lie pour les groupoides diff\'erentiable, relation entre propri\'etes locales et globales, \emph{C. R. Acad Sci. Paris S\'er. A} \textbf{268}: 907-910.

\bibitem{Raptis1-2k3}
Raptis, I.: 2003, Algebraic quantisation of causal sets, \emph{Int. Jour. Theor. Phys.} \textbf{39}: 1233.

\bibitem{Rieffel}
M. A. Rieffel : Group C*--algebras as compact quantum metric spaces, \emph{Documenta Math.} \textbf{7} (2002), 605-651.

\bibitem{Roberts}
Roberts, J. E.: 2004, More lectures on algebraic quantum field theory, in A. Connes, et al. \emph{Noncommutative Geometry}, Springer: Berlin and New York.

\bibitem{RRosen2}
Rosen, R.: 1958b, The Representation of Biological Systems from the Standpoint of the Theory of Categories., \emph{ Bulletin of Mathematical Biophysics} \textbf{20}: 317-341.

\bibitem{Szabo}
Szabo, R. J.: 2003, Quantum field theory on non-commutative spaces, \emph{Phys. Rep.} \textbf{378}: 207--209.

\bibitem{Varilly}
V\'arilly, J. C.: 1997, An introduction to noncommutative geometry. (at arXiv:physics/9709045)

\bibitem{Weinstein96}
Weinstein, A.: 1996, Groupoids : unifying internal and external symmetry, \emph{Notices of the Amer. Math. Soc.} \textbf{43}: 744--752.

\bibitem{WB83}
Wess J. and J. Bagger: 1983, \emph{Supersymmetry and Supergravity}, Princeton University Press: Princeton, NJ.


\end{thebibliography}

%%%%%
%%%%%
\end{document}
