\documentclass[12pt]{article}
\usepackage{pmmeta}
\pmcanonicalname{CayleyGraphOfS3}
\pmcreated{2013-03-22 17:11:40}
\pmmodified{2013-03-22 17:11:40}
\pmowner{Wkbj79}{1863}
\pmmodifier{Wkbj79}{1863}
\pmtitle{Cayley graph of $S_3$}
\pmrecord{22}{39513}
\pmprivacy{1}
\pmauthor{Wkbj79}{1863}
\pmtype{Example}
\pmcomment{trigger rebuild}
\pmclassification{msc}{20F06}
\pmclassification{msc}{20B30}
\pmclassification{msc}{05C25}
\pmrelated{SymmetricGroup}
\pmrelated{Presentationgroup}

\endmetadata

\usepackage{amssymb}
\usepackage{amsmath}
\usepackage{amsfonts}
\usepackage{pstricks}
\usepackage{psfrag}
\usepackage{graphicx}
\usepackage{amsthm}
%%\usepackage{xypic}

\begin{document}
In this entry, a Cayley graph of $S_3$, the group of permutations of $\{1,2,3\}$, will be investigated.

Let $\varepsilon \in S_3$ denote the identity element, $\tau \in S_3$ be a transposition, and $\gamma \in S_3$ be a \PMlinkname{three-cycle}{SymmetricGroup}.  Then $\{\tau, \gamma \}$ is a generating set of $S_3$ and $\langle\tau, \gamma \vert \tau^2=\varepsilon, \gamma^3=\varepsilon, \tau\gamma =\gamma^2\tau\rangle$ is a presentation of $S_3$.  The corresponding Cayley graph $\Gamma(S_3,\{\tau, \gamma \})$ is:

\begin{center}
\begin{pspicture}(-7,-7)(7,7)
\pscircle(0,0){6}
\rput[a](0,6){.}
\rput[l](-6.05,0){.}
\rput[r](6.05,0){.}
\rput[b](0,-6.5){$\gamma$}
\rput[l](-5.6,3.2){$\gamma$}
\rput[r](5.6,3.2){$\gamma$}
\pscircle(0,4){2}
\rput[l](-2.4,4){$\tau$}
\rput[r](2.4,4){$\tau$}
\psline(-2,4)(-2.1,4.1)
\psline(-2,4)(-1.9,4.1)
\psline(2,4)(1.9,3.9)
\psline(2,4)(2.1,3.9)
\pscircle(-3.4641,-2){2}
\rput[b](-1.6,-3.4){$\tau$}
\rput[a](-3.6,0.24){$\tau$}
\psline(-3.4641,0)(-3.3641,0.1)
\psline(-3.4641,0)(-3.3641,-0.1)
\psline(-1.732,-3)(-1.732,-3.2)
\psline(-1.732,-3)(-1.932,-3)
\pscircle(3.4641,-2){2}
\rput[b](1.6,-3.4){$\tau$}
\rput[a](3.6,0.24){$\tau$}
\psline(3.4641,0)(3.5641,0.1)
\psline(3.4641,0)(3.5641,-0.1)
\psline(1.732,-3)(1.873,-2.859)
\psline(1.732,-3)(1.532,-3)
\pspolygon(0,2)(-1.732,-1)(1.732,-1)
\rput[b](0,-1.5){$\gamma$}
\rput[l](-1.1,0.9){$\gamma$}
\rput[r](1.1,0.9){$\gamma$}
\psdots(0,6)(0,2)(-1.732,-1)(1.732,-1)(5.196,-3)(-5.196,-3)
\psline(-5.196,3)(-5.196,3.2)
\psline(-5.196,3)(-4.996,3)
\psline(0,-6)(-0.1,-6.1)
\psline(0,-6)(-0.1,-5.9)
\psline(5.196,3)(5.055,2.859)
\psline(5.196,3)(5.396,3)
\psline(0,-1)(0.1,-1.1)
\psline(0,-1)(0.1,-0.9)
\psline(-0.866,0.5)(-0.725,0.359)
\psline(-0.866,0.5)(-1.066,0.5)
\psline(0.866,0.5)(1.007,0.641)
\psline(0.866,0.5)(0.666,0.5)
\end{pspicture}
\end{center}

If each of the edges is assigned a \PMlinkname{length}{BasicLength} of one, then the Cayley graph is a hyperbolic metric space, as it is 2 hyperbolic.  Thus, $S_3$ is a hyperbolic group.
%%%%%
%%%%%
\end{document}
