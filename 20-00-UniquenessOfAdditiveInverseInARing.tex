\documentclass[12pt]{article}
\usepackage{pmmeta}
\pmcanonicalname{UniquenessOfAdditiveInverseInARing}
\pmcreated{2013-03-22 14:13:54}
\pmmodified{2013-03-22 14:13:54}
\pmowner{alozano}{2414}
\pmmodifier{alozano}{2414}
\pmtitle{uniqueness of additive inverse in a ring}
\pmrecord{7}{35672}
\pmprivacy{1}
\pmauthor{alozano}{2414}
\pmtype{Theorem}
\pmcomment{trigger rebuild}
\pmclassification{msc}{20-00}
\pmclassification{msc}{16-00}
\pmclassification{msc}{13-00}
\pmrelated{UniquenessOfInverseForGroups}

% this is the default PlanetMath preamble.  as your knowledge
% of TeX increases, you will probably want to edit this, but
% it should be fine as is for beginners.

% almost certainly you want these
\usepackage{amssymb}
\usepackage{amsmath}
\usepackage{amsthm}
\usepackage{amsfonts}

% used for TeXing text within eps files
%\usepackage{psfrag}
% need this for including graphics (\includegraphics)
%\usepackage{graphicx}
% for neatly defining theorems and propositions
%\usepackage{amsthm}
% making logically defined graphics
%%%\usepackage{xypic}

% there are many more packages, add them here as you need them

% define commands here

\newtheorem{thm}{Theorem}
\newtheorem{defn}{Definition}
\newtheorem{prop}{Proposition}
\newtheorem*{lemma}{Lemma}
\newtheorem{cor}{Corollary}

% Some sets
\newcommand{\Nats}{\mathbb{N}}
\newcommand{\Ints}{\mathbb{Z}}
\newcommand{\Reals}{\mathbb{R}}
\newcommand{\Complex}{\mathbb{C}}
\newcommand{\Rats}{\mathbb{Q}}
\begin{document}
\begin{lemma}
Let $R$ be a ring, and let $a$ be any element of $R$. There exists a unique element $b$ of $R$ such that $a+b=0$, i.e. there is a unique \PMlinkname{additive inverse}{Ring} for $a$.
\end{lemma}
\begin{proof}
Let $a$ be an element of $R$. By definition of ring, there exists at least one \PMlinkname{additive inverse}{Ring} of $a$, call it $b_1$, so that $a+b_1=0$. Now, suppose $b_2$ is another additive inverse of $a$, i.e. another element of $R$ such that 
$$a+b_2=0$$
where $0$ is the \PMlinkname{zero element}{Ring} of $R$. Let us show that $b_1=b_2$. Using properties for a ring and the above equations for $b_1$ and $b_2$
yields
\begin{eqnarray*}
b_1 &=& b_1+0 \quad \text{(definition of zero)}\\
 &=& b_1+(a+b_2) \quad (b_2 \text{ is an additive inverse of }a)\\
 &=& (b_1+a)+b_2 \quad (\text{associativity in }R)\\
 &=& 0+b_2 \quad (b_1\text{ is an additive inverse of }a)\\
 &=& b_2 \quad \text{(definition of zero)}.
\end{eqnarray*}
Therefore, there is a unique additive inverse for $a$.
\end{proof}
%%%%%
%%%%%
\end{document}
