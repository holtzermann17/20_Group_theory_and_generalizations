\documentclass[12pt]{article}
\usepackage{pmmeta}
\pmcanonicalname{NormalSubgroupsFormSublatticeOfASubgroupLattice}
\pmcreated{2013-03-22 15:48:24}
\pmmodified{2013-03-22 15:48:24}
\pmowner{CWoo}{3771}
\pmmodifier{CWoo}{3771}
\pmtitle{normal subgroups form sublattice of a subgroup lattice}
\pmrecord{5}{37770}
\pmprivacy{1}
\pmauthor{CWoo}{3771}
\pmtype{Example}
\pmcomment{trigger rebuild}
\pmclassification{msc}{20E15}

\usepackage{amssymb,amscd}
\usepackage{amsmath}
\usepackage{amsfonts}

% used for TeXing text within eps files
%\usepackage{psfrag}
% need this for including graphics (\includegraphics)
%\usepackage{graphicx}
% for neatly defining theorems and propositions
%\usepackage{amsthm}
% making logically defined graphics
%%%\usepackage{xypic}

% define commands here
\begin{document}
Consider $L(G)$, the subgroup lattice of a group $G$.  Let $N(G)$ be the subset of $L(G)$, consisting of all normal subgroups of $G$.  

First, we show that $N(G)$ is closed under $\wedge$.  Suppose $H$ and $K$ are normal subgroups of $G$.  If $x\in H\wedge K=H\cap K$, then for any $g\in G$, $gxg^{-1}\in H$ since $H$ is normal, and $gxg^{-1}\in K$ likewise.  So $gxg^{-1}\in H\cap K=H\wedge K$, implying that $H\wedge K$ is normal in $G$, or $H\wedge K \in N(G)$.


To see that $N(G)$ is closed under $\vee$, let $H,K$ be normal subgroups of $G$, and consider an element $$x=x_1x_2\cdots x_n\in H\vee K,$$
where $x_i\in H$ or $x_i\in K$.  If $g\in G$, then 
$$gxg^{-1}=gx_1x_2\cdots x_ng^{-1}=(gx_1g^{-1})(gx_2g^{-1})\cdots (gx_ng^{-1}),$$
where each $gx_ig^{-1}\in H$ or $K$.  Therefore, $gxg^{-1}\in H\vee K$, so $H\vee K$ is normal in $G$ and $H\vee K\in N(G)$.

Since $N(G)$ is closed under $\wedge$ and $\vee$, $N(G)$ is a sublattice of $L(G)$.

\textbf{Remark}.  If $G$ is finite, it can be shown (Wielandt) that the subnormal subgroups of $G$ form a sublattice of $L(G)$.

\begin{thebibliography}{9}
\bibitem{hw} H. Wielandt \emph{Eine Verallgemeinerung der invarianten Untergruppen}, Math. Zeit. 45, pp. 209-244 (1939)
\end{thebibliography}
%%%%%
%%%%%
\end{document}
