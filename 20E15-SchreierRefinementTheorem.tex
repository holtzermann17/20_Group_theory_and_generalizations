\documentclass[12pt]{article}
\usepackage{pmmeta}
\pmcanonicalname{SchreierRefinementTheorem}
\pmcreated{2013-03-22 14:40:53}
\pmmodified{2013-03-22 14:40:53}
\pmowner{yark}{2760}
\pmmodifier{yark}{2760}
\pmtitle{Schreier refinement theorem}
\pmrecord{8}{36286}
\pmprivacy{1}
\pmauthor{yark}{2760}
\pmtype{Theorem}
\pmcomment{trigger rebuild}
\pmclassification{msc}{20E15}

\usepackage{amssymb}
\usepackage{amsmath}
\usepackage{amsfonts}
\begin{document}
\PMlinkescapeword{equivalent}
\PMlinkescapeword{factors}
\PMlinkescapeword{order}
\PMlinkescapeword{refinements}
\PMlinkescapeword{states}

The \emph{Schreier Refinement Theorem} states that any two subnormal series for a group have equivalent refinements.
Here, two subnormal series are considered equivalent if they have the same factors (up to isomorphism), not necessarily in the same order.

This theorem can be used to prove the \PMlinkname{Jordan-H\"older Theorem}{JordanHolderDecompositionTheorem}, and can also be used to prove that the Hirsch number of a polycyclic group is well-defined.
%%%%%
%%%%%
\end{document}
