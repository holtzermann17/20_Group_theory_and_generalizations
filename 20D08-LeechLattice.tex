\documentclass[12pt]{article}
\usepackage{pmmeta}
\pmcanonicalname{LeechLattice}
\pmcreated{2013-03-22 18:43:23}
\pmmodified{2013-03-22 18:43:23}
\pmowner{monster}{22721}
\pmmodifier{monster}{22721}
\pmtitle{Leech lattice}
\pmrecord{7}{41491}
\pmprivacy{1}
\pmauthor{monster}{22721}
\pmtype{Definition}
\pmcomment{trigger rebuild}
\pmclassification{msc}{20D08}
\pmclassification{msc}{20B25}
\pmclassification{msc}{11H56}
\pmclassification{msc}{11H06}
\pmclassification{msc}{51E10}
\pmrelated{BinaryGolayCode}
\pmrelated{MiracleOctadGenerator}

\endmetadata

% this is the default PlanetMath preamble.  as your knowledge
% of TeX increases, you will probably want to edit this, but
% it should be fine as is for beginners.

% almost certainly you want these
\usepackage{amssymb}
\usepackage{amsmath}
\usepackage{amsfonts}

% used for TeXing text within eps files
%\usepackage{psfrag}
% need this for including graphics (\includegraphics)
%\usepackage{graphicx}
% for neatly defining theorems and propositions
%\usepackage{amsthm}
% making logically defined graphics
%%%\usepackage{xypic}

% there are many more packages, add them here as you need them

% define commands here
\newcommand{\gc}{\mathcal{G}_{24}}

\begin{document}
\PMlinkescapeword{words}

The Leech lattice is the unique \PMlinkname{even}{EvenLattice} unimodular lattice of \PMlinkname{dimension}{Dimension2} 24 having no elements of norm 2.  Its \PMlinkname{automorphism group}{EquivalentCode} is the largest Conway group $Co_0$ (sometimes denoted by $\cdot 0$).  The quotient of $Co_0$ by its center is called $Co_1$, a sporadic simple group.  

The construction of the Leech lattice below depends on the existence of the extended binary Golay code $\gc$ (for a construction of the latter, see miracle octad generator).

\section{Construction of the Leech lattice}

Let $\Omega = \{1,2,\ldots,24\}$ and assume we have constructed the Golay \PMlinkescapetext{code} $\gc$ on $\Omega$.  The Leech lattice $\Lambda$ is the set of all points $$\frac{1}{\sqrt{8}}(x_1, x_2, \ldots, x_{24})$$ in $\mathbb{R}^{24}$ where each $x_i$ is an integer, such that
\begin{itemize}
\item{For some integer $m$, we have $x_i \equiv x_j \equiv m \pmod{2}$ for all $i, j \in \Omega$;}
\item{For any integer $n$, the set of coordinates $\{i \in \Omega: x_i \equiv n \pmod{4}\}$ is in $\gc$;}
\item{$\sum_{i \in\Omega} x_i \equiv 4m \pmod{8}$.}
\end{itemize}

\section{Properties of the Leech lattice}
1. The Leech lattice $\Lambda$ is an \PMlinkescapetext{even} unimodular lattice; in other words:
\begin{itemize}
\item{The set $\Lambda$ spans all of $\mathbb{R}^{24}$ as an $\mathbb{R}$-vector space.}
\item{For any $x,y \in \Lambda$, the scalar product $x \cdot y$ is an integer.}
\item{For any $x \in \Lambda$, the norm $x \cdot x$ is an even integer.}
\item{The volume of the fundamental parallelogram of $\Lambda$ is 1.}
\end{itemize}

2. Let $\Lambda(n) = \{x \in \Lambda: x \cdot x = 2n \}$.  Then $|\Lambda(0)| = 1$, $|\Lambda(1)| = 0$, $|\Lambda(2)| = 196560$, $|\Lambda(3)| = 16773120$, $|\Lambda(4)| = 398034000$.

3. The \PMlinkname{automorphism group}{EquivalentCode} $\mbox{Aut}(\Lambda)$ is the largest Conway group $Co_0$ with order $8\,315\,553\,613\,086\,720\,000 = 2^{22} \cdot 3^9 \cdot 5^4 \cdot 7^2 \cdot 11 \cdot 13 \cdot 23$. 

4. The group $Co_0$ acts \PMlinkname{transitively}{LeftAction} on the sets $\Lambda(2)$, $\Lambda(3)$, $\Lambda(4)$.  For $n=2,3$, the imprimitivity blocks of the action of $Co_0$ on $\Lambda(n)$ are the sets $\{x, -x\}$ where $x \in \Lambda(n)$.  The imprimitivity blocks of the action of $Co_0$ on $\Lambda(4)$ are sets of 48 vectors called \emph{\PMlinkescapetext{coordinate frames}}.  Any two distinct vectors in a \PMlinkescapetext{coordinate frame} are either \PMlinkescapetext{opposite} or orthogonal, and are \PMlinkname{congruent}{QuotientGroup} modulo $2\Lambda$.

5. Any vector in $\Lambda$ is \PMlinkescapetext{congruent} modulo $2\Lambda$ to a vector in $\Lambda(n)$ for one of $n = 0,2,3,4$.  The imprimitivity blocks of the action of $Co_0$ on these sets account for all \PMlinkname{classes}{EquivalenceClass} of $\Lambda / 2 \Lambda$: $$1 + |\Lambda(2)|/2 + |\Lambda(3)|/2 + |\Lambda(4)|/48 = 2^{24} = |\Lambda / 2 \Lambda|.$$ 





\begin{thebibliography}{9}
\bibitem{splag} J. H. Conway and N. J. A. Sloane. Sphere Packings, Lattices, and Groups. Springer-Verlag, 1999.
\end{thebibliography}

%%%%%
%%%%%
\end{document}
