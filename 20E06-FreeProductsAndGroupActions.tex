\documentclass[12pt]{article}
\usepackage{pmmeta}
\pmcanonicalname{FreeProductsAndGroupActions}
\pmcreated{2013-03-22 17:34:56}
\pmmodified{2013-03-22 17:34:56}
\pmowner{rm50}{10146}
\pmmodifier{rm50}{10146}
\pmtitle{free products and group actions}
\pmrecord{5}{39994}
\pmprivacy{1}
\pmauthor{rm50}{10146}
\pmtype{Theorem}
\pmcomment{trigger rebuild}
\pmclassification{msc}{20E06}

% this is the default PlanetMath preamble.  as your knowledge
% of TeX increases, you will probably want to edit this, but
% it should be fine as is for beginners.

% almost certainly you want these
\usepackage{amssymb}
\usepackage{amsmath}
\usepackage{amsfonts}

% used for TeXing text within eps files
%\usepackage{psfrag}
% need this for including graphics (\includegraphics)
%\usepackage{graphicx}
% for neatly defining theorems and propositions
\usepackage{amsthm}
% making logically defined graphics
%%%\usepackage{xypic}

% there are many more packages, add them here as you need them

% define commands here
\newtheorem{thm}{Theorem}

\begin{document}
\begin{thm} (See Lang, Exercise 54 p. 81) Suppose $G_1,\ldots G_n$ are subgroups of $G$ that generate $G$. Suppose further that $G$ acts on a set $S$ and that there are subsets $S_1,S_2,\ldots S_n\subset S$, and some $s\in S-\cup S_i$ such that for each $1\leq i\leq n$, the following holds for each $g\in G_i, g\neq e$:
\begin{itemize}
\item $g(S_j)\subset S_i$ if $j\neq i$, and
\item $g(s)\in S_i$.
\end{itemize}
Then $G=G_1\star\ldots\star G_n$ (where $\star$ denotes the free product).
\end{thm}

\textbf{Proof:}
Any $g\in G$ can be written $g=g_1g_2\ldots g_k$ with $g_i\in G_{j_i}, j_i\neq j_{i+1}, g_i\neq e$, since the $G_i$ generate $G$. Thus there is a surjective homomorphism $\phi:\coprod G_i\twoheadrightarrow G$ (since $\coprod G_i$, as the coproduct, has this universal property). We must show $\ker \phi$ is trivial. Choose $g_1g_2\ldots g_k$ as above. Then $g_k(s)\in S_{j_k}$, $g_{k-1}(g_k(s))\in S_{j_{k-1}}$, and so forth, so that $g_1(g_2(\ldots(g_k(s)\ldots)\in S_{j_1}$. But $e(s)=s\notin S_{j_1}$. Thus $\phi(g_1g_2\ldots g_k) \neq e$, and $\phi$ is injective.

%%%%%
%%%%%
\end{document}
