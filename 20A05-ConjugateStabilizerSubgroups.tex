\documentclass[12pt]{article}
\usepackage{pmmeta}
\pmcanonicalname{ConjugateStabilizerSubgroups}
\pmcreated{2013-03-22 13:21:44}
\pmmodified{2013-03-22 13:21:44}
\pmowner{Thomas Heye}{1234}
\pmmodifier{Thomas Heye}{1234}
\pmtitle{conjugate stabilizer subgroups}
\pmrecord{7}{33888}
\pmprivacy{1}
\pmauthor{Thomas Heye}{1234}
\pmtype{Derivation}
\pmcomment{trigger rebuild}
\pmclassification{msc}{20A05}
\pmrelated{Orbit}

% this is the default PlanetMath preamble.  as your knowledge
% of TeX increases, you will probably want to edit this, but
% it should be fine as is for beginners.

% almost certainly you want these
\usepackage{amssymb}
\usepackage{amsmath}
\usepackage{amsfonts}

% used for TeXing text within eps files
%\usepackage{psfrag}
% need this for including graphics (\includegraphics)
%\usepackage{graphicx}
% for neatly defining theorems and propositions
%\usepackage{amsthm}
% making logically defined graphics
%%%\usepackage{xypic}

% there are many more packages, add them here as you need them

% define commands here
\begin{document}
Let $\operatorname{\cdot}$ be a right group action of $G$ on a set $M$. Then
\begin{displaymath}
G_{\alpha \cdot g} =g^{-1}G_{\alpha}g
\end{displaymath}
for any $\alpha \in M$ and $g \in G$. \footnote{$G_{\alpha}$ is the stabilizer subgroup of $\alpha \in M$.}

Proof:

\[
x \in G_{\alpha\cdot g} \leftrightarrow \alpha\cdot (gx) = \alpha \cdot g
 \leftrightarrow \alpha \cdot (gxg^{-1}) = \alpha \leftrightarrow gxg^{-1} \in
 G_{\alpha} \\ \leftrightarrow x \in g^{-1}\alpha g\]

and therefore $G_{\alpha\cdot g} =g^{-1}G_{\alpha}g$.

Thus all stabilizer subgroups for elements of the orbit $G(\alpha)$ of $\alpha$
are conjugate to $G_{\alpha}$.
%%%%%
%%%%%
\end{document}
