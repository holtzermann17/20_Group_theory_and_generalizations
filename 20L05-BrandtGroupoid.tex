\documentclass[12pt]{article}
\usepackage{pmmeta}
\pmcanonicalname{BrandtGroupoid}
\pmcreated{2013-03-22 18:38:32}
\pmmodified{2013-03-22 18:38:32}
\pmowner{CWoo}{3771}
\pmmodifier{CWoo}{3771}
\pmtitle{Brandt groupoid}
\pmrecord{18}{41383}
\pmprivacy{1}
\pmauthor{CWoo}{3771}
\pmtype{Definition}
\pmcomment{trigger rebuild}
\pmclassification{msc}{20L05}
\pmclassification{msc}{18B40}
\pmrelated{GroupoidCategoryTheoretic}
\pmrelated{ConnectedCategory}

\endmetadata

\usepackage{amssymb,amscd}
\usepackage{amsmath}
\usepackage{amsfonts}
\usepackage{mathrsfs}

% used for TeXing text within eps files
%\usepackage{psfrag}
% need this for including graphics (\includegraphics)
%\usepackage{graphicx}
% for neatly defining theorems and propositions
\usepackage{amsthm}
% making logically defined graphics
%%\usepackage{xypic}
\usepackage{pst-plot}

% define commands here
\newcommand*{\abs}[1]{\left\lvert #1\right\rvert}
\newtheorem{prop}{Proposition}
\newtheorem{thm}{Theorem}
\newtheorem{ex}{Example}
\newcommand{\real}{\mathbb{R}}
\newcommand{\pdiff}[2]{\frac{\partial #1}{\partial #2}}
\newcommand{\mpdiff}[3]{\frac{\partial^#1 #2}{\partial #3^#1}}
\begin{document}
Brandt groupoids, like \PMlinkname{category theoretic groupoids}{GroupoidCategoryTheoretic}, are generalizations of groups, where a multiplication is defined, and inverses with respect to the multiplication exist for each element.  However, unlike elements of a group, each element in a Brandt groupoid behaves like an arrow, with a source and target, and multiplication of two elements only work when the target of the first element coincides with the source of the second element.

\subsubsection*{Definition}

A \emph{Brandt groupoid} is a non-empty set $B$, together with a partial binary operation (called a multiplication) $\cdot$ defined on it (we write $ab$ for $a\cdot b$), such that
\begin{enumerate}
\item For every $a\in B$, there are unique elements $e,f$ such that $ea$ and $af$ are defined, and is equal to $a$.
\item If $ae=a$ or $ea=a$ for some $a,e\in B$, then $ee$ is defined, and is equal to $e$.
\item For $a,b\in B$, $ab$ is defined iff there is an $e\in B$ such that $ae=a$ and $eb=b$.
\item For $a,b,c\in B$ such that $ab$ and $bc$ are defined, then so are $(ab)c$ and $a(bc)$ and they equal.
\item If $ea=af=e$ for some $a,e,f\in B$, then there is a $b\in B$ such that $ab$ and $ba$ are defined and $ab=e$ and $ba=f$.
\item If $ee=e$ and $ff=f$ for some $e,f\in B$, then there is $a\in B$ such that $ea$ and $af$ are defined and are equal to $a$.
\end{enumerate}

In the definition above, we see several instances of elements $e$ such that $e^2=ee=e$.  Such elements are called \emph{idempotents}.  If we let $I$ be the set of all idempotents of $B$, then $I\ne \varnothing$ by conditions 1 and 2.

\subsubsection*{Brandt Groupoids versus Categories}

Brandt groupoids are intimately related to categories, as we will presently discuss.

The first two conditions above imply that there are two surjective functions $s,t:B\to I$, where $t(a)$ and $s(a)$ are the unique idempotents such that $a s(a)=a$ and $t(a) a=a$.  In addition, $s(e)=t(e)=e$ for all $e\in I$.  Call $s$ the source function, $t$ the target function, and for any $a\in B$, $s(a),t(a)$ the source and the target of $a$.

The third condition says that $ab$ is defined iff the source of $a$ is the equal to the target of $b$: $s(a)=t(b)$.  The fourth condition is the associativity law for the multiplication.  An easy consequence of this condition is that if $ab$ exists, then $s(b)=s(ab)$ and $t(a)=t(ab)$.

Altogether, the first four conditions say that a $B$ is a small category, with $I$ its set of objects, and $G$ the set of morphisms, and composition of morphisms is just the multiplication.

A morphism $a$ in $B$ is said to be an \emph{isomorphism} if there is a morphism $b$ in $G$ such that $ab,ba \in I$.  Now, $b$ is uniquely determined by $a$, so that $a$ is an isomorphism in the category theoretic sense.  
\begin{proof}
First notice that $s(b)=s(ab)=ab=t(ab)=t(a)$ and $t(b)=t(ba)=ba=s(ba)=s(a)$.  If $ac, ca \in I$, then $s(c)=t(a)=s(b)$ and $t(c)=s(a)=t(b)$.  So $ab=ac$ and $ba=bc$.  As a result, $c=t(c)c = t(b)c= (ba)c=b(ac)= b(ab)= b s(b)=b$.  
\end{proof}
$b$ is said to be the inverse of $a$, and is often written $a^{-1}$.  Condition 5 says that the category $B$ is in fact a \PMlinkname{category theoretic groupoid}{GroupoidCategoryTheoretic}.  Thus, a Brandt groupoid is a group if the multiplication is everywhere defined.

Finally, condition 6 says that between every pair of objects, there is a morphism from one to the other, this is equivalent to saying that $B$ is strongly connected.  As a result, a Brandt groupoid may be equivalently defined as a small strongly connected groupoid (in the category theoretic sense).

\subsubsection*{An Example}
A Brandt groupoid may be constructed as follows: take a group $G$ and a non-empty set $I$, set $B:=I\times G\times I$, and define multiplication on $B$ as follows:
\begin{displaymath}
(p,x,q)(r,y,s) = \left\{
\begin{array}{ll}
(p,xy,s) & \textrm{if }q=r,\\
\textrm{undefined} & \textrm{otherwise.}
\end{array}
\right.
\end{displaymath}
Then $B$ with the partial multiplication is a Brandt groupoid.  The idempotents in $B$ have the form $(p,e,p)$, where $e\in G$ is the group identity.  And for any $(p,x,q)$, its source, target, and inverse are $(q,e,q)$ and $(p,e,p)$, $(q,x^{-1},p)$ respectively.

In fact, it may be shown that every Brandt groupoid is isomorphic to one constructed above (for a proof, see \PMlinkname{here}{ConstructionOfABrandtGroupoid}).

\textbf{Remark}.  A non-trivial Brandt groupoid can not have a zero element, for if $0a=a0=0$ for all $a\in B$, then $a$ must be the source and target of $0$, but then $a$ would have to be unique by condition 1, which is impossible unless $B$ is trivial.  If we adjoin $0$ to a Brandt groupoid $B$, and call $S:=B\cup \lbrace 0\rbrace$, then $S$ has the structure of a semigroup.  Here's how the multiplication is defined on $S$: 
\begin{displaymath}
ab = \left\{
\begin{array}{ll}
ab & \textrm{if $ab$ is defined in $B$},\\
0 & \textrm{otherwise, or if either $a=0$ or $b=0$}.
\end{array}
\right.
\end{displaymath}
Since the multiplication on $S$ is everywhere defined, $S$ is a groupoid.  To see that $S$ is a semigroup, we must show that associativity of the multiplication applies everywhere.  There are four cases
\begin{itemize}
\item 
If both $ab$ and $bc$ are defined in $B$, they are certainly defined in $S$, and the associativity follows from condition 4.  
\item
If neither $ab$ nor $bc$ is defined in $B$, then $(ab)c=0c=0=a0=a(bc)$ in $S$.  
\item
If $ab$ is not defined in $B$, but $bc$ is, then $s(a)\ne t(b)=t(bc)$, and $(ab)c=0c=0=a(bc)$.  
\item
Similarly, if $ab$ is defined in $B$ but not $bc$, then $(ab)c=0=a(bc)$.
\end{itemize}
Thus, $S$ is a semigroup (with $0$).  In fact, Clifford showed that $S$ is completely simple.

\begin{thebibliography}{9}
\bibitem{hb} H. Brandt, \emph{Uber die Axiome des Gruppoids, Vierteljschr.  naturforsch. Ges. Zurich 85, Beiblatt (Festschrift Rudolph Fueter)}, pp. 95-104, {\bf MR2}, 218, 1940.
\bibitem{rhb} R. H. Bruck, \emph{A Survey on Binary Systems}, Springer-Verlag, New York, 1966.
\bibitem{nj} N. Jacobson, \emph{Theory of Rings}, American Mathematical Society, New York, 1943.
\bibitem{ahc} A. H. Clifford, \emph{Matrix Representations of Completely Simple Semigroups}, Amer. J. Math. 70. pp. 521-526, 1948.
\end{thebibliography}
%%%%%
%%%%%
\end{document}
