\documentclass[12pt]{article}
\usepackage{pmmeta}
\pmcanonicalname{RepresentationRing}
\pmcreated{2013-03-22 19:19:02}
\pmmodified{2013-03-22 19:19:02}
\pmowner{joking}{16130}
\pmmodifier{joking}{16130}
\pmtitle{representation ring}
\pmrecord{7}{42255}
\pmprivacy{1}
\pmauthor{joking}{16130}
\pmtype{Definition}
\pmcomment{trigger rebuild}
\pmclassification{msc}{20C99}

% this is the default PlanetMath preamble.  as your knowledge
% of TeX increases, you will probably want to edit this, but
% it should be fine as is for beginners.

% almost certainly you want these
\usepackage{amssymb}
\usepackage{amsmath}
\usepackage{amsfonts}

% used for TeXing text within eps files
%\usepackage{psfrag}
% need this for including graphics (\includegraphics)
%\usepackage{graphicx}
% for neatly defining theorems and propositions
%\usepackage{amsthm}
% making logically defined graphics
%%%\usepackage{xypic}

% there are many more packages, add them here as you need them

% define commands here

\begin{document}
Let $G$ be a group and $k$ a field. Consider the class
$$\mathcal{R}=\{X\ |\ X\mbox{ is a representation of }G\mbox{ over }k\}$$
and its subclass $\mathcal{R}_f$ consisting of those representations which are finite-dimensional as vector spaces. We consider a special representation
$$\mathcal{F}=(V,\cdot)$$
where $V$ is a fixed vector space with a basis $\mathcal{B}$ which is in bijective correspondence with $G$. If $f:\mathcal{B}\to G$ is a required bijection, then we define ,,$\cdot$'' on basis $\mathcal{B}$ by
$$g\cdot b = gf(b)$$
where on the right side we have a multiplication in $G$. It can be shown that this gives us a well-defined representation and further more, if $X\in\mathcal{R}_f$, then there exists an epimorphism of representations
$$e:\mathcal{F}^n\to X$$
for some $n\in\mathbb{N}$ ($\mathcal{F}$ is a ,,free'' representation). In particular every finite-dimensional representation is a quotient of a direct sum of copies of $\mathcal{F}$. This fact shows that a maximal subclass $\mathcal{X}\subset\mathcal{R}_f$ consisting of pairwise nonisomorphic representations is actually a set (note that $\mathcal{X}$ is never unique). Fix such a set.

\textbf{Definition.} \textbf{The representation semiring} $\overline{R_k(G)}$ of $G$ is defined as a triple $(\mathcal{X}, +, \cdot)$, where $\mathcal{X}$ is a maximal set of pairwise nonisomorphic representations taken from $\mathcal{R}_f$. Addition and multiplication are given by
$$X+Y=Z$$
where $Z$ is a representation in $\mathcal{X}$ isomorphic to the direct sum $X\oplus Y$ and
$$X\cdot Y=Z'$$
where $Z'$ is a representation in $\mathcal{X}$ isomorphic to the tensor product $X\otimes Y$. Note that $\overline{R_k(G)}$ is not a ring, because there are no additive inverses.

\textbf{The representation ring} $R_k(G)$ is defined as \PMlinkname{the Grothendieck ring}{GrothendieckGroup} induced from $\overline{R_k(G)}$. It can be shown that the definition does not depend on the choice of $\mathcal{X}$ (in the sense that it always gives us naturally isomorphic rings).

It is convenient to forget about formal definition which includes the choice of $\mathcal{X}$ and simply write elements of $\overline{R_k(G)}$ as isomorphism classes of representations $[X]$. Thus every element in $R_k(G)$ can be written as a formal difference $[X]-[Y]$. And we can write
$$[X]+[Y]=[X\oplus Y];$$
$$[X][Y] =[X\otimes Y].$$
%%%%%
%%%%%
\end{document}
