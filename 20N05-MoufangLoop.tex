\documentclass[12pt]{article}
\usepackage{pmmeta}
\pmcanonicalname{MoufangLoop}
\pmcreated{2013-03-22 13:50:29}
\pmmodified{2013-03-22 13:50:29}
\pmowner{yark}{2760}
\pmmodifier{yark}{2760}
\pmtitle{Moufang loop}
\pmrecord{12}{34578}
\pmprivacy{1}
\pmauthor{yark}{2760}
\pmtype{Definition}
\pmcomment{trigger rebuild}
\pmclassification{msc}{20N05}

\endmetadata

\usepackage{amssymb}
\usepackage{amsmath}
\usepackage{amsfonts}
\begin{document}
\PMlinkescapeword{algebra}
\PMlinkescapeword{binary}
\PMlinkescapeword{equivalent}
\PMlinkescapeword{implies}
\PMlinkescapeword{proof}
\PMlinkescapeword{proposition}
\PMlinkescapeword{references}
\PMlinkescapeword{theory}
\PMlinkescapeword{unit}
\textbf{Proposition: } Let $Q$ be a nonempty quasigroup.

I) The following conditions are equivalent.
\begin{eqnarray}
(x(yz))x &=& (xy)(zx) \qquad\text{for all } x,y,z\in Q \\
((xy)z)y &=& x(y(zy)) \qquad\text{for all } x,y,z\in Q \\
(xz)(yx) &=& x((zy)x) \qquad\text{for all } x,y,z\in Q \\
((yz)y)x &=& y(z(yx)) \qquad\text{for all } x,y,z\in Q
\end{eqnarray}

II) If $Q$ satisfies those conditions, then $Q$ has an identity element
(i.e., $Q$ is a loop).

For a proof, we refer the reader to the two references.
Kunen in [1] shows that that any of the four conditions implies the
existence of an identity element. And Bol and Bruck [2] show that
the four conditions are equivalent for loops.

\textbf{Definition:} A nonempty quasigroup satisfying the conditions
(1)--(4) is called a Moufang quasigroup or, equivalently, a Moufang
loop (after Ruth Moufang, 1905--1977).

The 16-element set of unit octonions over $\mathbb{Z}$ is an example
of a nonassociative Moufang loop.
Other examples appear in projective geometry, coding theory, and elsewhere.

\textbf{References}

[1] Kenneth Kunen, {\it Moufang Quasigroups}, J. Algebra 83 (1996) 231--234.
(A preprint in PostScript format is available from Kunen's website:
\PMlinkexternal{Moufang Quasigroups}{http://www.math.wisc.edu/~kunen/moufang.ps}.)

[2] R. H. Bruck, {\it A Survey of Binary Systems,} Springer-Verlag, 1958.
%%%%%
%%%%%
\end{document}
