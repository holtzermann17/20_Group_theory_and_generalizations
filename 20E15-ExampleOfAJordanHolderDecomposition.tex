\documentclass[12pt]{article}
\usepackage{pmmeta}
\pmcanonicalname{ExampleOfAJordanHolderDecomposition}
\pmcreated{2013-03-22 14:24:33}
\pmmodified{2013-03-22 14:24:33}
\pmowner{mathcam}{2727}
\pmmodifier{mathcam}{2727}
\pmtitle{example of a Jordan-H\"older decomposition}
\pmrecord{10}{35911}
\pmprivacy{1}
\pmauthor{mathcam}{2727}
\pmtype{Example}
\pmcomment{trigger rebuild}
\pmclassification{msc}{20E15}
\pmsynonym{example of Jordan-Holder decomposition}{ExampleOfAJordanHolderDecomposition}

\usepackage{amssymb}
\usepackage{amsmath}
\usepackage{amsfonts}
\begin{document}
\PMlinkescapeword{equivalent}
\PMlinkescapeword{order}
\PMlinkescapeword{sequence}
\PMlinkescapeword{series}

A group that has a composition series will often have several different composition series.

For example, the cyclic group $C_{12}$ has $(E, C_2, C_6, C_{12})$, and  $(E, C_2, C_4, C_{12})$, and $(E, C_3, C_6, C_{12})$ as different composition series. 
However, the result of the Jordan-H\"older Theorem is that any two composition series of a group are equivalent, in the sense that the sequence of factor groups in each series are the same, up to rearrangement of their order in the sequence $A_{i+1} / A_i$. In the above example, the factor groups are isomorphic to $(C_2, C_3, C_2)$, $(C_2, C_2, C_3)$, and $(C_3, C_2, C_2)$, respectively.

This is taken from the \PMlinkexternal{Wikipedia article on solvable groups}{http://en.wikipedia.org/wiki/Solvable_group}.
%%%%%
%%%%%
\end{document}
