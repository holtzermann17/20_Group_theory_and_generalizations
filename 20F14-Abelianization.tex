\documentclass[12pt]{article}
\usepackage{pmmeta}
\pmcanonicalname{Abelianization}
\pmcreated{2013-03-22 14:52:57}
\pmmodified{2013-03-22 14:52:57}
\pmowner{yark}{2760}
\pmmodifier{yark}{2760}
\pmtitle{abelianization}
\pmrecord{7}{36561}
\pmprivacy{1}
\pmauthor{yark}{2760}
\pmtype{Definition}
\pmcomment{trigger rebuild}
\pmclassification{msc}{20F14}
\pmsynonym{abelianisation}{Abelianization}
\pmrelated{DerivedSubgroup}

\endmetadata

\usepackage{amssymb}
\usepackage{amsmath}
\usepackage{amsfonts}
\begin{document}
\PMlinkescapeword{homomorphism}
\PMlinkescapeword{quotient}

The \emph{abelianization} of a group $G$ is $G/[G,G]$, the \PMlinkname{quotient}{QuotientGroup} of $G$ by its derived subgroup.

The abelianization of $G$ is the largest abelian quotient of $G$, in the sense that if $N$ is a normal subgroup of $G$ then $G/N$ is abelian if and only if $[G,G]\subseteq N$.
In particular, every abelian quotient of $G$ is a homomorphic image of $G/[G,G]$.

If $A$ is an abelian group and $\phi\colon G\to A$ is a \PMlinkname{homomorphism}{GroupHomomorphism},
then there is a unique homomorphism $\psi\colon G/[G,G]\to A$ such that
$\psi\circ\pi=\phi$, where $\pi\colon G\to G/[G,G]$ is the canonical projection.
%%%%%
%%%%%
\end{document}
