\documentclass[12pt]{article}
\usepackage{pmmeta}
\pmcanonicalname{ProofThatANontrivialNormalSubgroupOfAFinitePgroupGAndTheCenterOfGHaveNontrivialIntersection}
\pmcreated{2013-03-22 14:21:07}
\pmmodified{2013-03-22 14:21:07}
\pmowner{rm50}{10146}
\pmmodifier{rm50}{10146}
\pmtitle{proof that a nontrivial normal subgroup of a finite $p$-group $G$ and the center of $G$ have nontrivial intersection}
\pmrecord{9}{35828}
\pmprivacy{1}
\pmauthor{rm50}{10146}
\pmtype{Proof}
\pmcomment{trigger rebuild}
\pmclassification{msc}{20D20}

% this is the default PlanetMath preamble.  as your knowledge
% of TeX increases, you will probably want to edit this, but
% it should be fine as is for beginners.

% almost certainly you want these
\usepackage{amssymb}
\usepackage{amsmath}
\usepackage{amsfonts}

% used for TeXing text within eps files
%\usepackage{psfrag}
% need this for including graphics (\includegraphics)
%\usepackage{graphicx}
% for neatly defining theorems and propositions
%\usepackage{amsthm}
% making logically defined graphics
%%%\usepackage{xypic}

% there are many more packages, add them here as you need them

% define commands here
\begin{document}
Define $G$ to act on $H$ by conjugation; that is, for $g\in G$, $h\in H$, define
\[g \cdot h = ghg^{-1}\]
Note that $g\cdot h\in H$ since $H\triangleleft G$. This is easily seen to be a well-defined group action.

Now, the set of invariants of $H$ under this action are
\[G_H=\{h \in H \  \lvert \  g \cdot h = h \  \forall g\in G\} = \{h \in H \  \lvert \  ghg^{-1} = h \  \forall g\in G\}=H \cap Z(G)\]

The class equation theorem states that
\[\lvert H\rvert = \lvert G_H\rvert + \sum_{i=1}^{r}[G:G_{x_{i}}]\]
where the $G_{x_{i}}$ are proper subgroups of $G$, and thus that
\[\lvert G_H\rvert = \lvert H\rvert - \sum_{i=1}^{r}[G:G_{x_{i}}]\]

We now use elementary group theory to show that $p$ divides each term on the right, and conclude as a result that $p$ divides $\lvert G_H\rvert$, so that $G_H=H\cap Z(G)$ cannot be trivial.

As $G$ is a nontrivial finite $p$-group, it is obvious from Cauchy's theorem that $|G|=p^{n}$ for $n>0$. Since $H$ and the $G_{x_i}$ are subgroups of $G$, each either is trivial or has order a power of $p$, by Lagrange's theorem. Since $H$ is nontrivial, its order is a nonzero power of $p$. Since each $G_{x_i}$ is a proper subgroup of $G$ and has order a power of $p$, it follows that $[G:G_{x_i}]$ also has order a nonzero power of $p$.
%%%%%
%%%%%
\end{document}
