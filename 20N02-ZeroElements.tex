\documentclass[12pt]{article}
\usepackage{pmmeta}
\pmcanonicalname{ZeroElements}
\pmcreated{2013-03-22 13:02:19}
\pmmodified{2013-03-22 13:02:19}
\pmowner{mclase}{549}
\pmmodifier{mclase}{549}
\pmtitle{zero elements}
\pmrecord{6}{33440}
\pmprivacy{1}
\pmauthor{mclase}{549}
\pmtype{Definition}
\pmcomment{trigger rebuild}
\pmclassification{msc}{20N02}
\pmclassification{msc}{20M99}
\pmrelated{Semigroup}
\pmrelated{NullSemigroup}
\pmrelated{AbsorbingElement}
\pmdefines{zero}
\pmdefines{zero element}
\pmdefines{right zero}
\pmdefines{left zero}

\endmetadata

% this is the default PlanetMath preamble.  as your knowledge
% of TeX increases, you will probably want to edit this, but
% it should be fine as is for beginners.

% almost certainly you want these
\usepackage{amssymb}
\usepackage{amsmath}
\usepackage{amsfonts}

% used for TeXing text within eps files
%\usepackage{psfrag}
% need this for including graphics (\includegraphics)
%\usepackage{graphicx}
% for neatly defining theorems and propositions
\usepackage{amsthm}
% making logically defined graphics
%%%\usepackage{xypic}

% there are many more packages, add them here as you need them

% define commands here
\newtheorem{prop}{Proposition}
\begin{document}
Let $S$ be a semigroup.  An element $z$ is called a \emph{right zero} [resp. \emph{left zero}] if $xz = z$ [resp. $zx = z$] for all $x \in S$.  

An element which is both a left and a right zero is called a \emph{zero element}.

A semigroup may have many left zeros or right zeros, but if it has at least one of each, then they are necessarily equal, giving a unique (two-sided) zero element.

More generally, these definitions and statements are valid for a groupoid.

It is customary to use the symbol $\theta$ for the zero element of a semigroup.

\begin{prop} If a groupoid has a left zero $0_L$ and a right zero $0_R$, then $0_L = 0_R$. \end{prop}
\begin{proof} $0_L=0_L 0_R = 0_R$. \end{proof}

\begin{prop} If $0$ is a left zero in a semigroup $S$, then so is $x0$ for every $x\in S$.  \end{prop}
\begin{proof}
For any $y\in S$, $(x0)y=x(0y)=x0$.  As a result, $x0$ is a left zero of $S$.
\end{proof}

\begin{prop} If $0$ is the unique left zero in a semigroup $S$, then it is also the zero element.  \end{prop}
\begin{proof}
By assumption and the previous proposition, $x0$ is a left zero for every $x\in S$.  But $0$ is the unique left zero in $S$, we must have $x0=0$, which means that $0$ is a right zero element, and hence a zero element by the first proposition.
\end{proof}
%%%%%
%%%%%
\end{document}
