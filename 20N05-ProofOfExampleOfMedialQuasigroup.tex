\documentclass[12pt]{article}
\usepackage{pmmeta}
\pmcanonicalname{ProofOfExampleOfMedialQuasigroup}
\pmcreated{2013-03-22 16:27:35}
\pmmodified{2013-03-22 16:27:35}
\pmowner{rspuzio}{6075}
\pmmodifier{rspuzio}{6075}
\pmtitle{proof of example of medial quasigroup}
\pmrecord{8}{38618}
\pmprivacy{1}
\pmauthor{rspuzio}{6075}
\pmtype{Proof}
\pmcomment{trigger rebuild}
\pmclassification{msc}{20N05}

\endmetadata

% this is the default PlanetMath preamble.  as your knowledge
% of TeX increases, you will probably want to edit this, but
% it should be fine as is for beginners.

% almost certainly you want these
\usepackage{amssymb}
\usepackage{amsmath}
\usepackage{amsfonts}

% used for TeXing text within eps files
%\usepackage{psfrag}
% need this for including graphics (\includegraphics)
%\usepackage{graphicx}
% for neatly defining theorems and propositions
%\usepackage{amsthm}
% making logically defined graphics
%%%\usepackage{xypic}

% there are many more packages, add them here as you need them

% define commands here

\begin{document}
We shall proceed by first showing that the algebraic systems defined in the \PMlinkname{parent entry}{MedialQuasigroup} are quasigroups and then showing that the medial property is satisfied.

To show that the system is a quasigroup, we need to check the solubility of equations.  Let $x$ and $y$ be two elements of $G$.  Then, by definition of $\cdot$, the equation $x \cdot z = y$ is equivalent to 
 \[ f(x) + g(z) + c = y .\]
This is equivalent to 
 \[g(z) = y - c - f(x).\]
Since $g$ is an automorphism, there will exist a unique solution $z$ to this equation.

Likewise, the equation $z \cdot x = y$ is equivalent to
 \[ f(z) + g(x) + c = y \]
which, in turn is equivalent to 
 \[f(z) = y - c - g(x),\]
so we may also find a unique $z$ such that $z \cdot x = y$.  Hence, $(G, \cdot)$ is a quasigroup.

To check the medial property, we use the definition of $\cdot$ to conclude that
\begin{eqnarray*} (x \cdot y) \cdot (z \cdot w) &=& (f(x) + g(y) + c) \cdot (f(z) + g(w) + c) \\ &=& f (f(x) + g(y) + c) + g (f(z) + g(w) + c) + c 
\end{eqnarray*}
Since $f$ and $g$ are automorphisms and the group is commutative, this equals
 \[ f(f(x)) + f(g(y)) + g(f(z)) + g(g(w)) + f(c) + g(c) + c .\]
Since $f$ and $g$ commute this, in turn, equals
 \[ f(f(x)) + g(f(y)) + f(g(z)) + g(g(w)) + f(c) + g(c) + c .\]
Using the commutative and associative laws, we may regroup this expression as follows:
 \[ (f(f(x)) + f(g(z)) + f(c)) + (g(f(y)) + g(g(w)) + g(c)) + c \]
Because $f$ and $g$ are automorphisms, this equals
 \[ f (f(x) + g(z) + c) + g( f(y) + g(w) + c) + c \]
By defintion of $\cdot$, this equals
 \[ f (x \cdot z) + g (y \cdot z) + c ,\]
which equals $(x \cdot z) \cdot (y \cdot z)$, so we have
 \[ (x \cdot y) \cdot (z \cdot w) = (x \cdot z) \cdot (y \cdot z). \]
Thus, the medial property is satisfied, so we have a medial quasigroup.
%%%%%
%%%%%
\end{document}
