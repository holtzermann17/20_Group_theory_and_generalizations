\documentclass[12pt]{article}
\usepackage{pmmeta}
\pmcanonicalname{BaumslagSolitarGroup}
\pmcreated{2013-03-22 17:25:12}
\pmmodified{2013-03-22 17:25:12}
\pmowner{yark}{2760}
\pmmodifier{yark}{2760}
\pmtitle{Baumslag-Solitar group}
\pmrecord{5}{39794}
\pmprivacy{1}
\pmauthor{yark}{2760}
\pmtype{Definition}
\pmcomment{trigger rebuild}
\pmclassification{msc}{20F05}
\pmsynonym{Solitar-Baumslag group}{BaumslagSolitarGroup}

\endmetadata

\usepackage{amssymb}
\usepackage{amsmath}
\usepackage{amsfonts}

\begin{document}
A \emph{Baumslag-Solitar group} is a group with presentation 
\[
  \langle b,t\mid t^{-1}b^mt=b^n\rangle
\]
for some non-zero integers $m$ and $n$.

These groups were studied by Baumslag and Solitar\cite{bs},
who were interested in finding examples
of finitely generated \PMlinkname{non-Hopfian}{HopfianGroup} groups.
In particular, the group
\[
  \langle b,t\mid t^{-1}b^2t=b^3\rangle
\]
is non-Hopfian.

\begin{thebibliography}{9}
\bibitem{bs}
 G.~Baumslag, D.~Solitar,
 {\it Some two-generator one-relator non-Hopfian groups},
 Bull.\ Amer.\ Math.\ Soc., 68 (1962) 199--201.
 (This paper is \PMlinkexternal{available on-line from Project Euclid}{http://projecteuclid.org/euclid.bams/1183524561}.)
\bibitem{eom}
 D.~J.~Collins,
 {\it \PMlinkexternal{Baumslag--Solitar group}{http://eom.springer.de/b/b130070.htm}},
 in the {\it Online Encyclopaedia of Mathematics}.
\end{thebibliography}

%%%%%
%%%%%
\end{document}
