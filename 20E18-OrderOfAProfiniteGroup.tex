\documentclass[12pt]{article}
\usepackage{pmmeta}
\pmcanonicalname{OrderOfAProfiniteGroup}
\pmcreated{2013-03-22 15:23:39}
\pmmodified{2013-03-22 15:23:39}
\pmowner{mathcam}{2727}
\pmmodifier{mathcam}{2727}
\pmtitle{order of a profinite group}
\pmrecord{5}{37228}
\pmprivacy{1}
\pmauthor{mathcam}{2727}
\pmtype{Definition}
\pmcomment{trigger rebuild}
\pmclassification{msc}{20E18}
\pmdefines{index of a profinite subgroup}
\pmdefines{index of a profinite group}

% this is the default PlanetMath preamble.  as your knowledge
% of TeX increases, you will probably want to edit this, but
% it should be fine as is for beginners.

% almost certainly you want these
\usepackage{amssymb}
\usepackage{amsmath}
\usepackage{amsfonts}
\usepackage{amsthm}

% used for TeXing text within eps files
%\usepackage{psfrag}
% need this for including graphics (\includegraphics)
%\usepackage{graphicx}
% for neatly defining theorems and propositions
%\usepackage{amsthm}
% making logically defined graphics
%%%\usepackage{xypic}

% there are many more packages, add them here as you need them

% define commands here

\newcommand{\mc}{\mathcal}
\newcommand{\mb}{\mathbb}
\newcommand{\mf}{\mathfrak}
\newcommand{\ol}{\overline}
\newcommand{\ra}{\rightarrow}
\newcommand{\la}{\leftarrow}
\newcommand{\La}{\Leftarrow}
\newcommand{\Ra}{\Rightarrow}
\newcommand{\nor}{\vartriangleleft}
\newcommand{\Gal}{\text{Gal}}
\newcommand{\GL}{\text{GL}}
\newcommand{\Z}{\mb{Z}}
\newcommand{\R}{\mb{R}}
\newcommand{\Q}{\mb{Q}}
\newcommand{\C}{\mb{C}}
\newcommand{\<}{\langle}
\renewcommand{\>}{\rangle}
\begin{document}
Let $G$ be a profinite group, and let $H$ be any closed subgroup.  We define the \emph{\PMlinkescapetext{index} of $H$ in $G$} by

\begin{align*}
[G:H]=\operatorname{lcm}(\{[G/N:HN/N]\}),
\end{align*}
where $N$ runs over all open (and hence of finite index) subgroups of $G$, and where $\operatorname{lcm}$ is taken in the sense of the least common multiple of supernatural numbers.

In particular, we can define the \emph{order of a profinite group} to be the index of the identity subgroup in $G$:
\begin{align*}
|G|:=[G:\{e\}].
\end{align*}

Some examples of orders of profinite groups:
\begin{itemize}
\item $G=\mathbb{Z}_p$, the ring of $p$-adic integers.  Since every finite quotient of $\mathbb{Z}_p$ is cyclic of $p^n$ elements (for some $n$), and every such group occurs as a quotient, we have $|G|=\operatorname{lcm}(p^n),$ where $n$ runs over all natural numbers.  Thus $|G|=p^\infty$. 
\item $G=\widehat{\mathbb{Z}}.$  Since $G\approx\prod_p\mathbb{Z}_p$, we have $|G|=\prod_p |\mathbb{Z}_p|=\prod_p p^\infty$.  This example illustrates the limitations of this concept:  Despite being ``relatively small'' in \PMlinkescapetext{terms} of profinite groups, $\widehat{\mathbb{Z}}$ has the largest possible profinite order.
\end{itemize}


\begin{thebibliography}{9}
\bibitem{Ram}[Ram] Ramakrishnan, Dinikara and Valenza, Robert. \emph{Fourier Analysis on Number Fields}. Graduate Texts in Mathematics, volume 186. Springer-Verlag, New York, NY. 1989.
\bibitem{Ser}[Ser] Serre, J.-P. (Ion, P., translator) \emph{Galois Cohomology}. Springer, New York, NY. 1997
\end{thebibliography}
%%%%%
%%%%%
\end{document}
