\documentclass[12pt]{article}
\usepackage{pmmeta}
\pmcanonicalname{Exponent}
\pmcreated{2013-03-22 13:30:21}
\pmmodified{2013-03-22 13:30:21}
\pmowner{Wkbj79}{1863}
\pmmodifier{Wkbj79}{1863}
\pmtitle{exponent}
\pmrecord{23}{34087}
\pmprivacy{1}
\pmauthor{Wkbj79}{1863}
\pmtype{Definition}
\pmcomment{trigger rebuild}
\pmclassification{msc}{20A99}
\pmrelated{KummerTheory}

\endmetadata

% this is the default PlanetMath preamble.  as your knowledge
% of TeX increases, you will probably want to edit this, but
% it should be fine as is for beginners.

% almost certainly you want these
\usepackage{amssymb}
\usepackage{amsmath}
\usepackage{amsfonts}

% used for TeXing text within eps files
%\usepackage{psfrag}
% need this for including graphics (\includegraphics)
%\usepackage{graphicx}
% for neatly defining theorems and propositions
%\usepackage{amsthm}
% making logically defined graphics
%%%\usepackage{xypic}

% there are many more packages, add them here as you need them

% define commands here
\begin{document}
Let $G$ be a group with the \PMlinkescapetext{property} that there exists a positive integer $n$ such that, for every $g \in G$, $g^n=e_G$.  The {\sl exponent} of $G$, denoted $\operatorname{exp}~G$, is the smallest positive integer $m$ such that, for every $g \in G$, $g^m=e_G$.  Thus, for every finite group $G$, $\operatorname{exp}~G$ divides $|G|$.  Also, for every group $G$ that has an exponent and for every $g \in G$, $|g|$ divides $\operatorname{exp}~G$.

The concept of exponent for finite groups is \PMlinkescapetext{similar} to that of \PMlinkname{characterisic}{Characteristic} for finite fields.

If $G$ is a finite abelian group, then there exists $g \in G$ with $|g|=\operatorname{exp}~G$.  As a result of the \PMlinkname{fundamental theorem of finite abelian groups}{FundamentalTheoremOfFinitelyGeneratedAbelianGroups}, there exist $a_1, \ldots , a_n$ with $a_i$ dividing $a_{i+1}$ for every integer $i$ between 1 and $n$ such that $G \cong {\mathbb Z}_{a_1} \oplus \cdots \oplus {\mathbb Z}_{a_n}$.  Since, for every $c \in G$, $c^{a_n}=e_G$, then $\operatorname{exp}~G \le a_n$.  Since $|(0, \ldots , 0, 1)|=a_n$, it follows that $\operatorname{exp}~G=a_n$.

Following are some examples of exponents of finite nonabelian groups.

Since $|(1 2)|=2$, $|(1 2 3)|=3$, and $|S_3|=6$, it follows that $\operatorname{exp}~S_3=6$.

In $Q_8=\{1, -1, i, -i, j, -j, k, -k\}$, the ring of quaternions of order eight, since $|i|=|-i|=|j|=|-j|=|k|=|-k|=4$ and $1^4=(-1)^4=1$, it follows that $\operatorname{exp}~Q_8=4$.

Since the order of a product of two disjoint transpositions is 2, the order of a three cycle is 3, and the only nonidentity elements of $A_4$ are three cycles and products of two disjoint transpositions, it follows that $\operatorname{exp}~A_4=6$.

Since $|(1 2 3)|=3$ and $|(1 2 3 4)|=4$, $\operatorname{exp}~S_4 \ge 12$.  Since $S_4$ has no elements of order 8, it cannot have an element of order 24.  It follows that $\operatorname{exp}~S_4=12$.

Following are some examples of exponents of infinite groups.

Clearly, any infinite group that has an element of infinite order does not have an exponent.  On the other hand, just because an infinite group has the \PMlinkescapetext{property} that every element has finite order does not \PMlinkescapetext{mean} that the group has an exponent.  As an example, consider $G=\mathbb{Q}/\mathbb{Z}$, which is a group under addition.  Despite that all of its elements have finite order, $G$ does not have an exponent.  This is because, for every positive integer $n$, $G$ has an element of order $n$, namely $\displaystyle \frac{1}{n}+\mathbb{Z}$.

On the other hand, some infinite groups have exponents.  For example, let $\mathbb{F}_2$ denote the field having two elements.  Then $\mathbb{F}_2[x]$, the ring of all polynomials in $x$ with coefficients in $\mathbb{F}_2$, is an abelian group under addition.  Moreover, it is an infinite group; however, every nonzero element has order $2$.  Thus, $\operatorname{exp}~\mathbb{F}_2[x]=2$.
%%%%%
%%%%%
\end{document}
