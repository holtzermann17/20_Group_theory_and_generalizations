\documentclass[12pt]{article}
\usepackage{pmmeta}
\pmcanonicalname{DirectProductsOfHomomorphisms}
\pmcreated{2013-03-22 18:36:00}
\pmmodified{2013-03-22 18:36:00}
\pmowner{joking}{16130}
\pmmodifier{joking}{16130}
\pmtitle{direct products of homomorphisms}
\pmrecord{4}{41330}
\pmprivacy{1}
\pmauthor{joking}{16130}
\pmtype{Definition}
\pmcomment{trigger rebuild}
\pmclassification{msc}{20A99}

\endmetadata

% this is the default PlanetMath preamble.  as your knowledge
% of TeX increases, you will probably want to edit this, but
% it should be fine as is for beginners.

% almost certainly you want these
\usepackage{amssymb}
\usepackage{amsmath}
\usepackage{amsfonts}

% used for TeXing text within eps files
%\usepackage{psfrag}
% need this for including graphics (\includegraphics)
%\usepackage{graphicx}
% for neatly defining theorems and propositions
%\usepackage{amsthm}
% making logically defined graphics
%%%\usepackage{xypic}

% there are many more packages, add them here as you need them

% define commands here

\begin{document}
Assume that $\{f_i:G_i\to H_i\}_{i\in I}$ is a family of homomorphisms between groups. Then we can define the \textit{Cartesian product} (or \textit{unrestricted direct product}) of this family as a homomorphism $$\prod_{i\in I}f_i:\prod_{i\in I} G_i\to \prod_{i\in I}H_i$$ such that $$\bigg( \prod_{i\in I}f_i \bigg) \big( g\big) (j)=f_j(g(j))$$ for each $g\in \prod_{i\in I}G_i$ and $j\in I$.

One can easily show that $\prod_{i\in I}f_i$ is a group homomorphism. Moreover it is clear that $$\bigg( \prod_{i\in I}f_i \bigg) \big( \bigoplus_{i\in I} G_i \big) \subseteq \bigoplus_{i\in I} H_i,$$ so $\prod_{i\in I}f_i$ induces a homomorphism $$\bigoplus_{i\in I}f_i:\bigoplus_{i\in I}G_i\to \bigoplus_{i\in I}H_i,$$ which is a restriction of $\prod_{i\in I}f_i$ to $\bigoplus_{i\in I}G_i$. This homomorphism is called the \textit{direct product} (or \textit{restricted direct product}) of $\{f_i:G_i\to H_i\}_{i\in I}$.
%%%%%
%%%%%
\end{document}
