\documentclass[12pt]{article}
\usepackage{pmmeta}
\pmcanonicalname{NormalityOfSubgroupsIsNotTransitive}
\pmcreated{2013-03-22 12:49:27}
\pmmodified{2013-03-22 12:49:27}
\pmowner{yark}{2760}
\pmmodifier{yark}{2760}
\pmtitle{normality of subgroups is not transitive}
\pmrecord{13}{33147}
\pmprivacy{1}
\pmauthor{yark}{2760}
\pmtype{Example}
\pmcomment{trigger rebuild}
\pmclassification{msc}{20A05}
\pmrelated{NormalIsNotTransitive}

\endmetadata

\usepackage{amssymb}
\usepackage{amsmath}
\usepackage{amsfonts}
\usepackage{psfrag}
\usepackage{graphicx}

\newcommand{\Prob}[2]{\mathbb{P}_{#1}\left\{#2\right\}}
\newcommand{\Expect}{\mathbb{E}}
\newcommand{\norm}[1]{\left\|#1\right\|}

% some sets
\newcommand{\Nats}{\mathbb{N}}
\newcommand{\Ints}{\mathbb{Z}}
\newcommand{\Reals}{\mathbb{R}}
\newcommand{\Complex}{\mathbb{C}}

% normal subgroup notation
\newcommand{\normal}{\trianglelefteq}
\newcommand{\notnormal}{\ntrianglelefteq}

% psfrag stuff
\psfrag{a}{$a$}
\psfrag{b}{$b$}
\psfrag{c}{$c$}
\psfrag{d}{$d$}
\psfrag{r}{$r$}
\psfrag{f}{$f$}
\psfrag{r1}{$r^{-1}$}
\begin{document}
\PMlinkescapeword{even}
\PMlinkescapeword{isometries}
\PMlinkescapeword{isometry}
\PMlinkescapeword{similar}

Let $G$ be a group.
A subgroup $K$ of a subgroup $H$ of $G$ is obviously a subgroup of $G$.
It seems plausible that a similar situation would also hold for normal subgroups, but in fact it does not:
even when $K\normal H$ and $H\normal G$, it is possible that $K\notnormal G$.  Here are two examples:
\begin{enumerate}
\item
Let $G$ be the subgroup of orientation-preserving \PMlinkname{isometries}{Isometry} of the plane $\Reals^2$ ($G$ is just all rotations and translations), let $H$ be the subgroup of $G$ of translations, and let $K$ be the subgroup of $H$ of integer translations $\tau_{i,j}(x,y)=(x+i,y+j)$, where $i,j\in\Ints$.

Any element $g\in G$ may be represented as $g=r_1\circ t_1=t_2\circ r_2$, where $r_{1,2}$ are rotations and $t_{1,2}$ are translations.  So for any translation $t\in H$ we may write
$$
g^{-1}\circ t\circ g = r^{-1}\circ t'\circ r,
$$
where $t'\in H$ is some other translation and $r$ is some rotation.  But this is an orientation-preserving isometry of the plane that does not rotate, so it too must be a translation.  Thus $G^{-1}HG=H$, and $H\normal G$.

$H$ is an abelian group, so all its subgroups, $K$ included, are normal.

We claim that $K\notnormal G$.  Indeed, if $\rho\in G$ is rotation by $45^{\circ}$ about the origin, then $\rho^{-1}\circ \tau_{1,0}\circ \rho$ is not an integer translation.

\item
A related example uses finite subgroups.  Let $G=D_4$ be the dihedral group with eight elements (the group of automorphisms of the graph of the square).  Then
$$
D_4=\left\langle r,f \mid f^2=1, r^4=1, fr=r^{-1}f \right\rangle
$$
is generated by $r$, rotation, and $f$, flipping.

\begin{center}
\includegraphics{c2,2_in_d4}
\end{center}

The subgroup
$$
H = \langle rf, fr\rangle = \left\{ 1, rf, r^2, fr \right\} \cong C_2\times C_2
$$
is isomorphic to the Klein 4-group -- an identity and 3 elements of order 2.  $H\normal G$ since $[G:H] = 2$.  Finally, take
$$
K = \langle rf \rangle = \left\{ 1, rf \right\} \normal H.
$$

We claim that $K\notnormal G$.  And indeed,
$$
f\circ rf \circ f = fr \notin K.
$$
\end{enumerate}
%%%%%
%%%%%
\end{document}
