\documentclass[12pt]{article}
\usepackage{pmmeta}
\pmcanonicalname{EmpiricalProofThatSolvingOpposingFacesOfARubiksCubeDoesNotNecessarilySolveTheMiddleLayer}
\pmcreated{2013-03-22 15:36:56}
\pmmodified{2013-03-22 15:36:56}
\pmowner{PrimeFan}{13766}
\pmmodifier{PrimeFan}{13766}
\pmtitle{empirical proof that solving opposing faces of a Rubik's cube does not necessarily solve the middle layer}
\pmrecord{9}{37539}
\pmprivacy{1}
\pmauthor{PrimeFan}{13766}
\pmtype{Proof}
\pmcomment{trigger rebuild}
\pmclassification{msc}{20B99}
\pmclassification{msc}{00A08}

\endmetadata

% this is the default PlanetMath preamble.  as your knowledge
% of TeX increases, you will probably want to edit this, but
% it should be fine as is for beginners.

% almost certainly you want these
\usepackage{amssymb}
\usepackage{amsmath}
\usepackage{amsfonts}

% used for TeXing text within eps files
%\usepackage{psfrag}
% need this for including graphics (\includegraphics)
%\usepackage{graphicx}
% for neatly defining theorems and propositions
%\usepackage{amsthm}
% making logically defined graphics
%%%\usepackage{xypic}

% there are many more packages, add them here as you need them

% define commands here
\begin{document}
``Is it true that if two opposite faces of a Rubik's cube are completed the middle layer is completed too?"

Not necessarily, even though this is usually the case. A cube with two opposing faces completed but the middle layer scrambled does not arise from normal play, but can be contrived easily enough.

Given a Rubik's cube and a solve-by-layers algorithm, start with a scrambled cube.

1. Choose a color to be the top face. Let's say you choose white and the opposing face is blue.

2. Solve the top layer (i.e., making sure the colors of all the pieces on the top layer are in the right place).

3. Inspect the middle layer for any instance of the color you want on the bottom face (blue, in this example). Use Edge Mover 1 and Edge Mover 2 (moves that take a piece from the bottom layer and place it on the middle layer) to remove any blue pieces from the middle layer, but leave any other pieces in the middle layer so that each face's middle layer has two or three different colors.

4. Use Two Corner Swap (changes position of bottom corners), Three Corner Swap (changes orientation of bottom corners), Edge Flipper Left and Edge Flipper Right (change position of edge pieces in the bottom layer) to solve the bottom face, or if desired, the bottom layer. Each of these moves, upon completion, leaves the top and middle layer as it was before beginning the move, but makes the appropriate change to the bottom layer.

Now you should have a cube with white on the top face, blue on the bottom face and some``wrong" colors on the middle layer.

Hopefully this empirical proof can lead to a more mathematically rigorous proof.

\begin{thebibliography}{2}
\bibitem{ds} D. Singmaster, "Notes on Rubik's Magic Cube" (1980)
\bibitem{mw} Winning Moves, Inc., {\it Solutions Hints! Booklet}, (1996)
\end{thebibliography}
%%%%%
%%%%%
\end{document}
