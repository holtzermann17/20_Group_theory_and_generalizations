\documentclass[12pt]{article}
\usepackage{pmmeta}
\pmcanonicalname{OrderOfProducts}
\pmcreated{2013-03-22 18:56:43}
\pmmodified{2013-03-22 18:56:43}
\pmowner{pahio}{2872}
\pmmodifier{pahio}{2872}
\pmtitle{order of products}
\pmrecord{5}{41802}
\pmprivacy{1}
\pmauthor{pahio}{2872}
\pmtype{Theorem}
\pmcomment{trigger rebuild}
\pmclassification{msc}{20A05}
\pmrelated{InverseFormingInProportionToGroupOperation}

% this is the default PlanetMath preamble.  as your knowledge
% of TeX increases, you will probably want to edit this, but
% it should be fine as is for beginners.

% almost certainly you want these
\usepackage{amssymb}
\usepackage{amsmath}
\usepackage{amsfonts}

% used for TeXing text within eps files
%\usepackage{psfrag}
% need this for including graphics (\includegraphics)
%\usepackage{graphicx}
% for neatly defining theorems and propositions
 \usepackage{amsthm}
% making logically defined graphics
%%%\usepackage{xypic}

% there are many more packages, add them here as you need them

% define commands here

\theoremstyle{definition}
\newtheorem*{thmplain}{Theorem}

\begin{document}
If $a$ and $b$ are elements of a group, then both $ab$ and $ba$ have always the same order.\\

\emph{Proof.}\, Let $e$ be the indentity element of the group.\, For\, $n > 1$,\, we have the 
\PMlinkname{equivalent}{Equivalent3} conditions
$$e \;=\; (ab)^n \;=\; \underbrace{(ab)(ab)\cdots(ab)}_{n} \;=\; a(ba)^{n-1}b,$$
$$a^{-1}b^{-1} \;=\; (ba)^{n-1},$$
$$(ba)^{-1} \;=\; (ba)^{n-1},$$
$$e \;=\; (ba)^n.$$
As for the infinite order, it makes the conditions false.\\


\textbf{Note.}\, More generally, all elements of any conjugacy class have the same order.
%%%%%
%%%%%
\end{document}
