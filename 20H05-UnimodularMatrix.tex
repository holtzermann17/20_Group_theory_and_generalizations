\documentclass[12pt]{article}
\usepackage{pmmeta}
\pmcanonicalname{UnimodularMatrix}
\pmcreated{2013-03-22 14:57:50}
\pmmodified{2013-03-22 14:57:50}
\pmowner{CWoo}{3771}
\pmmodifier{CWoo}{3771}
\pmtitle{unimodular matrix}
\pmrecord{13}{36662}
\pmprivacy{1}
\pmauthor{CWoo}{3771}
\pmtype{Definition}
\pmcomment{trigger rebuild}
\pmclassification{msc}{20H05}
\pmclassification{msc}{15A04}
\pmclassification{msc}{15A09}
%\pmkeywords{unimodular}
%\pmkeywords{unimodularity}
\pmrelated{SpecialLinearGroup}
\pmdefines{unimodular linear transformation}
\pmdefines{unimodular row}
\pmdefines{unimodular column}
\pmdefines{unimodular group}
\pmdefines{unimodular vector}

\endmetadata

% this is the default PlanetMath preamble.  as your knowledge
% of TeX increases, you will probably want to edit this, but
% it should be fine as is for beginners.

% almost certainly you want these
\usepackage{amssymb,amscd}
\usepackage{amsmath}
\usepackage{amsfonts}

% used for TeXing text within eps files
%\usepackage{psfrag}
% need this for including graphics (\includegraphics)
%\usepackage{graphicx}
% for neatly defining theorems and propositions
%\usepackage{amsthm}
% making logically defined graphics
%%%\usepackage{xypic}

% there are many more packages, add them here as you need them

% define commands here
\begin{document}
\PMlinkescapeword{mean}
\PMlinkescapeword{even}
\PMlinkescapeword{similar}
\PMlinkescapeword{unimodular}

An $n\times n$ square matrix over a field is \emph{unimodular} if its determinant is 1.  The set of all $n\times n$ unimodular matrices forms a group under the usual matrix multiplication.  This group is known as the special linear group.  Any of its subgroup is simply called a \emph{unimodular group}.  Furthermore, unimodularity is preserved under similarity transformations: if $S$ any $n\times n$ invertible matrix and $U$ is unimodular, then $S^{-1}US$ is unimodular.  In view of the last statement, the special linear group is a normal subgroup of the group of all invertible matrices, known as the general linear group.  

A linear transformation $T$ over an $n$-dimensional vector space $V$ (over a field $F$) is \emph{unimodular} if it can be represented by a unimodular matrix.

The concept of the unimodularity of a square matrix over a field can be readily extended to that of a square matrix over a commutative ring.  Unimodularity in square matrices can even be extended to arbitrary finite-dimensional matrices: suppose $R$ is a commutative ring with 1, and $M$ is an $m\times n$ matrix over $R$ (entries are elements of $R$) with $m\leq n$.  Then $M$ is said to be \emph{unimodular} if it can be ``completed'' to a $n\times n$ square unimodular matrix $N$ over $R$.  By completion of $M$ to $N$ we mean that $m$ of the $n$ rows in $N$ are exactly the rows of $M$.  Of course, the operation of completion from a matrix to a square matrix can be done via columns too.

Let $M$ is an $m\times n$ matrix and $v$ is any row of $M$.  If $M$ is unimodular, then $v$ is unimodular viewed as a $1\times n$ matrix.  A $1\times n$ unimodular matrix is called a \emph{unimodular row}, or a \emph{unimodular vector}.  A $n\times 1$ \emph{unimodular column} can be defined via a similar procedure.  Let $v=(v_1,\ldots,v_n)$ be a $1\times n$ matrix over $R$.  Then the unimodularity of $v$ means that $$v_1R+\cdots+v_nR=R.$$  To see this, let $U$ be a completion of $v$ with $\operatorname{det}(U)=1$.  Since $\operatorname{det}$ is a multilinear operator over the rows (or columns) of $U$, we see that $$1=\operatorname{det}(U)=v_1r_1+\cdots+v_nr_n.$$
%%%%%
%%%%%
\end{document}
