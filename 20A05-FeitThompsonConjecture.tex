\documentclass[12pt]{article}
\usepackage{pmmeta}
\pmcanonicalname{FeitThompsonConjecture}
\pmcreated{2013-03-22 17:55:38}
\pmmodified{2013-03-22 17:55:38}
\pmowner{PrimeFan}{13766}
\pmmodifier{PrimeFan}{13766}
\pmtitle{Feit-Thompson conjecture}
\pmrecord{8}{40421}
\pmprivacy{1}
\pmauthor{PrimeFan}{13766}
\pmtype{Conjecture}
\pmcomment{trigger rebuild}
\pmclassification{msc}{20A05}
\pmclassification{msc}{20E32}

% this is the default PlanetMath preamble.  as your knowledge
% of TeX increases, you will probably want to edit this, but
% it should be fine as is for beginners.

% almost certainly you want these
\usepackage{amssymb}
\usepackage{amsmath}
\usepackage{amsfonts}

% used for TeXing text within eps files
%\usepackage{psfrag}
% need this for including graphics (\includegraphics)
%\usepackage{graphicx}
% for neatly defining theorems and propositions
%\usepackage{amsthm}
% making logically defined graphics
%%%\usepackage{xypic}

% there are many more packages, add them here as you need them

% define commands here

\begin{document}
Conjecture (Walter Feit \& John Thompson). There are no prime numbers $p$ and $q$ (with $p \neq q$) such that $\displaystyle \frac{p^q - 1}{p - 1}$ is divisible by $\displaystyle \frac{q^p - 1}{q - 1}$.

Feit and Thompson, in regards to the Feit-Thompson theorem, have said that proving this conjecture would simplify their proof of their theorem, ``rendering unnecessary the detailed use of generators and relations.'' In 1971, Stephens strengthened the conjecture to state that $$\gcd\left(\frac{p^q - 1}{p - 1}, \frac{q^p - 1}{q - 1}\right) = 1$$ always, and then found the counterexample $p = 17$, $q = 3313$. The numbers $\displaystyle \frac{17^{3313} - 1}{17 - 1}$ and $\displaystyle \frac{3313^{17} - 1}{3313 - 1}$ do have 112643 as their greatest common divisor, but dividing the former by the latter leaves a remainder of 149073454345008273252753518779212742886488244343395482423. No other counterexamples have been found to Stephen's stronger version of the conjecture.

\begin{thebibliography}{1}
\bibitem{ns} N. M. Stephens, ``On the Feit-Thompson Conjecture'' {\it Math. of Computation} {\bf 25} 115 (1971): 625
\end{thebibliography}
%%%%%
%%%%%
\end{document}
