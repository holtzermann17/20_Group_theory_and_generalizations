\documentclass[12pt]{article}
\usepackage{pmmeta}
\pmcanonicalname{CancellativeSemigroup}
\pmcreated{2013-03-22 14:25:09}
\pmmodified{2013-03-22 14:25:09}
\pmowner{yark}{2760}
\pmmodifier{yark}{2760}
\pmtitle{cancellative semigroup}
\pmrecord{9}{35926}
\pmprivacy{1}
\pmauthor{yark}{2760}
\pmtype{Definition}
\pmcomment{trigger rebuild}
\pmclassification{msc}{20M10}
\pmsynonym{cancellation semigroup}{CancellativeSemigroup}
\pmrelated{CancellationIdeal}
\pmdefines{cancellative}
\pmdefines{weakly cancellative}
\pmdefines{left cancellative}
\pmdefines{right cancellative}
\pmdefines{weakly cancellative semigroup}
\pmdefines{left cancellative semigroup}
\pmdefines{right cancellative semigroup}

\usepackage{amssymb}
\usepackage{amsmath}
\usepackage{amsfonts}
\usepackage{amsthm}

\begin{document}
Let $S$ be a semigroup.

$S$ is \emph{left cancellative} if, for all $a,b,c\in S$, $ab=ac\Rightarrow b=c$\\
$S$ is \emph{right cancellative} if, for all $a,b,c\in S$, $ba=ca\Rightarrow b=c$

$S$ is \emph{cancellative} if it is both left and right cancellative.

\section{Relationship to some other types of semigroup}
This is a generalisation of groups, and in fact being cancellative is a necessary condition for a semigroup to be embeddable in a group.

Note that a non-empty semigroup is a group if and only if it is cancellative and regular.

$S$ is \emph{weakly cancellative} if, for all $a,b,c\in S$, $(ab=ac~\&~ba=ca)\Rightarrow b=c$

A semigroup is completely simple if and only if it is weakly cancellative and regular.

\section{Individual elements}
An element $x\in S$ is called \emph{left cancellative} if, for all $b,c\in S$, $xb=xc\Rightarrow b=c$\\
An element $x\in S$ is called \emph{right cancellative} if, for all $b,c\in S$, $bx=cx\Rightarrow b=c$
%%%%%
%%%%%
\end{document}
