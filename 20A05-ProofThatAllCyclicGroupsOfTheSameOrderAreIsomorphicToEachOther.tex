\documentclass[12pt]{article}
\usepackage{pmmeta}
\pmcanonicalname{ProofThatAllCyclicGroupsOfTheSameOrderAreIsomorphicToEachOther}
\pmcreated{2013-03-22 13:30:41}
\pmmodified{2013-03-22 13:30:41}
\pmowner{Wkbj79}{1863}
\pmmodifier{Wkbj79}{1863}
\pmtitle{proof that all cyclic groups of the same order are isomorphic to each other}
\pmrecord{9}{34095}
\pmprivacy{1}
\pmauthor{Wkbj79}{1863}
\pmtype{Proof}
\pmcomment{trigger rebuild}
\pmclassification{msc}{20A05}

\usepackage{amssymb}
\usepackage{amsmath}
\usepackage{amsfonts}

\usepackage{psfrag}
\usepackage{graphicx}
\usepackage{amsthm}
%%\usepackage{xypic}
\begin{document}
The following is a proof that all cyclic groups of the same order are isomorphic to each other.

\begin{proof}
Let $G$ be a cyclic group and $g$ be a generator of $G$.  Define $\varphi \colon {\mathbb Z} \to G$ by $\varphi (c)=g^c$.  Since $\varphi(a+b)=g^{a+b}=g^ag^b=\varphi(a)\varphi(b)$, $\varphi$ is a group homomorphism.  If $h \in G$, then there exists $x \in {\mathbb Z}$ such that $h=g^x$.  Since $\varphi (x)=g^x=h$, $\varphi$ is surjective.

Note that $\ker \varphi =\{c \in {\mathbb Z}\, : \, \varphi(c)=e_G\}=\{c \in {\mathbb Z}\, : \, g^c=e_G\}$.

If $G$ is infinite, then $\ker \varphi = \{0\}$, and $\varphi$ is injective.  Hence, $\varphi$ is a group isomorphism, and $G \cong {\mathbb Z}$.

If $G$ is finite, then let $|G|=n$.  Thus, $|g|=|\langle g \rangle |=|G|=n$.  If $g^c=e_G$, then $n$ divides $c$.  Therefore, $\ker \varphi =n{\mathbb Z}$.  By the first isomorphism theorem, $G \cong \mathbb{Z}/n\mathbb{Z}=\mathbb{Z}_n$.

Let $H$ and $K$ be cyclic groups of the same order.  If $H$ and $K$ are infinite, then, by the above \PMlinkescapetext{argument}, $H \cong {\mathbb Z}$ and $K \cong {\mathbb Z}$.  If $H$ and $K$ are finite of order $n$, then, by the above \PMlinkescapetext{argument}, $H \cong {\mathbb Z}_n$ and $K \cong {\mathbb Z}_n$.  In any case, it follows that $H \cong K$.
\end{proof}
%%%%%
%%%%%
\end{document}
