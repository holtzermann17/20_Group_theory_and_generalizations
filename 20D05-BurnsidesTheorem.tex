\documentclass[12pt]{article}
\usepackage{pmmeta}
\pmcanonicalname{BurnsidesTheorem}
\pmcreated{2013-03-22 16:38:14}
\pmmodified{2013-03-22 16:38:14}
\pmowner{rm50}{10146}
\pmmodifier{rm50}{10146}
\pmtitle{Burnside's Theorem}
\pmrecord{4}{38839}
\pmprivacy{1}
\pmauthor{rm50}{10146}
\pmtype{Theorem}
\pmcomment{trigger rebuild}
\pmclassification{msc}{20D05}

% this is the default PlanetMath preamble.  as your knowledge
% of TeX increases, you will probably want to edit this, but
% it should be fine as is for beginners.

% almost certainly you want these
\usepackage{amssymb}
\usepackage{amsmath}
\usepackage{amsfonts}

% used for TeXing text within eps files
%\usepackage{psfrag}
% need this for including graphics (\includegraphics)
%\usepackage{graphicx}
% for neatly defining theorems and propositions
\usepackage{amsthm}
% making logically defined graphics
%%%\usepackage{xypic}

% there are many more packages, add them here as you need them

% define commands here
\newtheorem{thm}{Theorem}
\newtheorem{cor}{Corollary}
\begin{document}
\begin{thm}[Burnside's Theorem]

Let $G$ be a simple group, $\sigma\in G$. Then the number of conjugates of $\sigma$ is not a prime power (unless $\sigma$ is its own conjugacy class).
\end{thm}

Proofs of this theorem are quite difficult and rely on representation theory.

From this we immediately get
\begin{cor} A group $G$ of order $p^a q^b$, where $p,q$ are prime, cannot be a nonabelian simple group.
\end{cor}
\begin{proof}
Suppose it is. Then the center of $G$ is trivial, $\{e\}$, since the center is a normal subgroup and $G$ is simple nonabelian. So if $C_i$ are the nontrivial conjugacy classes, we have from the class equation that
\[\lvert G\rvert = 1+\sum\lvert C_i\rvert\]
Now, each $\lvert C_i\rvert$ divides $\lvert G\rvert$, but cannot be $1$ since the center is trivial. It cannot be a power of either $p$ or $q$ by Burnside's theorem. Thus $pq \mid \lvert C_i\rvert$ for each $i$ and thus $\lvert G\rvert\equiv 1\pmod {pq}$, which is a contradiction.
\end{proof}

Finally, a corollary of the above is known as the \PMlinkname{Burnside $p$-$q$ Theorem}{BurnsidePQTheorem}.
\begin{cor} A group of order $p^aq^b$ is solvable.
\end{cor}
%%%%%
%%%%%
\end{document}
