\documentclass[12pt]{article}
\usepackage{pmmeta}
\pmcanonicalname{BruhatDecomposition}
\pmcreated{2013-03-22 15:43:15}
\pmmodified{2013-03-22 15:43:15}
\pmowner{nerdy2}{62}
\pmmodifier{nerdy2}{62}
\pmtitle{Bruhat decomposition}
\pmrecord{9}{37669}
\pmprivacy{1}
\pmauthor{nerdy2}{62}
\pmtype{Theorem}
\pmcomment{trigger rebuild}
\pmclassification{msc}{20-00}

\endmetadata

% this is the default PlanetMath preamble.  as your knowledge
% of TeX increases, you will probably want to edit this, but
% it should be fine as is for beginners.

% almost certainly you want these
\usepackage{amssymb}
\usepackage{amsmath}
\usepackage{amsfonts}

% used for TeXing text within eps files
%\usepackage{psfrag}
% need this for including graphics (\includegraphics)
%\usepackage{graphicx}
% for neatly defining theorems and propositions
%\usepackage{amsthm}
% making logically defined graphics
%%%\usepackage{xypic} 

% there are many more packages, add them here as you need them

% define commands here
\begin{document}
Bruhat decomposition is the name for the fact that $B\backslash G/B = W$, where $G$ is a reductive group, $B$ a Borel subgroup, and $W$ the Weyl group.  Less canonically, one can write $G = BWB$.

In the case of the general linear group $G = GL_n$, $B$ is the group of nonsingular upper triangular matrices, and $W$ is the collection of permutation matrices (and is isomorphic to $S_n$).  Any nonsingular matrix can thus be written uniquely as a product of an upper triangular matrix, a permutation matrix, and another upper triangular matrix.
%%%%%
%%%%%
\end{document}
