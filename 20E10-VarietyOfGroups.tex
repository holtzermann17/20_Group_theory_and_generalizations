\documentclass[12pt]{article}
\usepackage{pmmeta}
\pmcanonicalname{VarietyOfGroups}
\pmcreated{2013-03-22 13:12:02}
\pmmodified{2013-03-22 13:12:02}
\pmowner{yark}{2760}
\pmmodifier{yark}{2760}
\pmtitle{variety of groups}
\pmrecord{27}{33662}
\pmprivacy{1}
\pmauthor{yark}{2760}
\pmtype{Definition}
\pmcomment{trigger rebuild}
\pmclassification{msc}{20E10}
\pmclassification{msc}{20J15}
\pmsynonym{variety}{VarietyOfGroups}
\pmrelated{GroupVariety}
\pmrelated{EquationalClass}
\pmdefines{Burnside group}

\endmetadata

\usepackage{amssymb}
\usepackage{amsmath}
\usepackage{amsfonts}

% The following lines should work as the command
% \renewcommand{\bibname}{References}
% without creating havoc when rendering an entry in 
% the page-image mode.
\makeatletter
\@ifundefined{bibname}{}{\renewcommand{\bibname}{References}}
\makeatother

\begin{document}
\PMlinkescapephrase{closed under}
\PMlinkescapephrase{free group}
\PMlinkescapeword{length}
\PMlinkescapeword{satisfies}
\PMlinkescapeword{satisfy}
\PMlinkescapeword{theorem}

\section*{Definition}

A \emph{variety} of groups is the class of groups $G$
that satisfy a given set of equationally defined relations 
\[
  \{\, r_i(x_1,\ldots,x_{n_i})=1 \mid i\in I \,\}
\]
for all elements $x_1,x_2,x_3,\ldots$ of $G$,
where $I$ is some index set.

\section*{Examples}

Abelian groups are a variety defined by the equations
$$\{[x_1,x_2]=1\},$$
where $[x,y]=xyx^{-1}y^{-1}$.

Nilpotent groups of class less than $c$ form a variety defined by
$$\{[[\cdots[[x_1,x_2],x_3]\cdots],x_c]=1\}.$$
Similarly, solvable groups of length less than $c$ form a variety.
(Abelian groups are a special case of both of these.)
Note, however, that the class of all nilpotent groups is not a variety,
nor is the class of all solvable groups.

For any positive integer $n$,
the variety defined by $\{x_1^n=1\}$
consists of all groups of finite exponent dividing $n$.
For $n=1$ this gives the variety containing only the trivial groups,
which is the smallest variety.

The largest variety is the variety of all groups,
given by an empty set of relations.

\section*{Notes}

By a theorem of Birkhoff\cite{birkhoff},
a class of groups is a variety if and only if
it is closed under taking subgroups, homomorphic images
and unrestricted direct products
(that is,
every unrestricted direct product of members
of the class $\cal C$ is in $\cal C$,
and all subgroups and homomorphic images of members of $\cal C$
are also in $\cal C$).

A variety of groups is a full subcategory of the category of groups,
and there is a free group on any set of elements in the variety,
which is the usual \PMlinkname{free group}{FreeGroup}
modulo the relations of the variety applied to all elements.
This satisfies the usual universal property
of the free group on groups in the variety,
and is thus \PMlinkname{adjoint}{AdjointFunctor} to the forgetful functor
in the category of sets.
In the variety of abelian groups,
the free groups are the usual free abelian groups.
In the variety of groups satisfying $x^n=1$,
the free groups are called \emph{Burnside groups},
and are commonly denoted by $B(m,n)$, where $m$ is the number of generators.

\begin{thebibliography}{9}
\bibitem{birkhoff}
 G.\ Birkhoff,
 {\it On the structure of abstract algebras},
 Proc.\ Cambridge Philos.\ Soc., 31 (1935), 433--454.
\end{thebibliography}
%%%%%
%%%%%
\end{document}
