\documentclass[12pt]{article}
\usepackage{pmmeta}
\pmcanonicalname{ReesFactor}
\pmcreated{2013-03-22 13:05:46}
\pmmodified{2013-03-22 13:05:46}
\pmowner{mclase}{549}
\pmmodifier{mclase}{549}
\pmtitle{Rees factor}
\pmrecord{4}{33517}
\pmprivacy{1}
\pmauthor{mclase}{549}
\pmtype{Definition}
\pmcomment{trigger rebuild}
\pmclassification{msc}{20M12}
\pmclassification{msc}{20M10}
\pmrelated{Ideal3}

\endmetadata

% this is the default PlanetMath preamble.  as your knowledge
% of TeX increases, you will probably want to edit this, but
% it should be fine as is for beginners.

% almost certainly you want these
\usepackage{amssymb}
\usepackage{amsmath}
\usepackage{amsfonts}

% used for TeXing text within eps files
%\usepackage{psfrag}
% need this for including graphics (\includegraphics)
%\usepackage{graphicx}
% for neatly defining theorems and propositions
%\usepackage{amsthm}
% making logically defined graphics
%%%\usepackage{xypic}

% there are many more packages, add them here as you need them

% define commands here
\begin{document}
Let $I$ be an ideal of a semigroup $S$.  Define a congruence $\sim$ by $x \sim y$ iff $x = y$ or $x, y \in I$.

Then the \emph{Rees factor} of $S$ by $I$ is the quotient $S/\sim$.  As a matter of notation, the congruence $\sim$ is normally suppressed, and the quotient is simply written $S/I$.

Note that a Rees factor always has a zero element.  Intuitively, the quotient identifies all element in $I$ and the resulting element is a zero element.
%%%%%
%%%%%
\end{document}
