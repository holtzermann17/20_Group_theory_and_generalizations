\documentclass[12pt]{article}
\usepackage{pmmeta}
\pmcanonicalname{pqUnshuffle}
\pmcreated{2013-03-22 16:47:45}
\pmmodified{2013-03-22 16:47:45}
\pmowner{Karid}{16341}
\pmmodifier{Karid}{16341}
\pmtitle{$(p,q)$ unshuffle}
\pmrecord{10}{39029}
\pmprivacy{1}
\pmauthor{Karid}{16341}
\pmtype{Definition}
\pmcomment{trigger rebuild}
\pmclassification{msc}{20B99}
\pmclassification{msc}{05A05}

% this is the default PlanetMath preamble.  as your knowledge
% of TeX increases, you will probably want to edit this, but
% it should be fine as is for beginners.

% almost certainly you want these
\usepackage{amssymb}
\usepackage{amsmath}
\usepackage{amsfonts}

% used for TeXing text within eps files
%\usepackage{psfrag}
% need this for including graphics (\includegraphics)
%\usepackage{graphicx}
% for neatly defining theorems and propositions
%\usepackage{amsthm}
% making logically defined graphics
%%%\usepackage{xypic}

% there are many more packages, add them here as you need them

% define commands here

\begin{document}
Let $p$ and $ q$ be positive natural numbers. Further, let $S(k)$ be the symmetric group on the numbers $ \{1,\ldots, k\}$. A permutation $ \tau\in S(p+q)$ is a $ (p,q)$ \emph{unshuffle} if there exist $i_1 < \cdots < i_p $ and $j_1 < \cdots <  j_q $ s.t.
\[
\tau(i_1)=1, \ldots ,\tau(i_p)=p
\]
and
\[
\tau(j_1)=p+1 \ldots ,  \tau(j_q)=p+q.
\]

Alternatively a $(p,q)$ \emph{unshuffle} is a permutation $ \tau\in S(p+q)$ s.t.~$\tau^{-1}$ is a $(p,q)$ shuffle.

Since a $ (p,q)$ unshuffle is completely determined by $\{i_1,\ldots ,i_p \}$, the cardinality of $\{\sigma\in S(p+q)|\mbox{$\sigma$ is an unshuffle}\}$ is $\binom{p+q}{q}$.
%%%%%
%%%%%
\end{document}
