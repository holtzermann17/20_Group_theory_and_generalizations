\documentclass[12pt]{article}
\usepackage{pmmeta}
\pmcanonicalname{Immanent}
\pmcreated{2013-03-22 14:05:43}
\pmmodified{2013-03-22 14:05:43}
\pmowner{Mathprof}{13753}
\pmmodifier{Mathprof}{13753}
\pmtitle{immanent}
\pmrecord{17}{35477}
\pmprivacy{1}
\pmauthor{Mathprof}{13753}
\pmtype{Definition}
\pmcomment{trigger rebuild}
\pmclassification{msc}{20C30}
%\pmkeywords{permanent}
%\pmkeywords{determinant}
%\pmkeywords{character}
%\pmkeywords{trace}
\pmrelated{permanent}
\pmrelated{character}

\endmetadata

% this is the default PlanetMath preamble.  as your knowledge
% of TeX increases, you will probably want to edit this, but
% it should be fine as is for beginners.

% almost certainly you want these
\usepackage{amssymb}
\usepackage{amsmath}
\usepackage{amsfonts}

% used for TeXing text within eps files
%\usepackage{psfrag}
% need this for including graphics (\includegraphics)
%\usepackage{graphicx}
% for neatly defining theorems and propositions
%\usepackage{amsthm}
% making logically defined graphics
%%%\usepackage{xypic}

% there are many more packages, add them here as you need them

% define commands here
\DeclareMathOperator{\imm}{Imm}
\begin{document}
Let $S_n$ denote the symmetric group on $n$ elements.
Let $\chi:S_n\to\mathbb C$ be a complex character.
For any $n\times n$ matrix $A=(a_{ij})_{i,j=1}^n$ define the \emph{immanent} of $A$ as
\[
\imm_{\chi} (A)=\sum_{\sigma\in {S_n}} \chi(\sigma) \prod_{j=1}^n A_{j \, \sigma( j)}.\]

Special cases of immanents are determinants and permanents --- in the case where $\chi$ is the constant character ($\chi (x) = 1$ for all $x \in S_n$), $\imm_{\chi} (A)$ is the permanent of $A$.  In the case where $\chi$ is the sign of the permutation (which is the character of the permutation group associated to the (non-trivial) one-dimensional representation), $\imm_{\chi} (A)$ is the determinant of $A$.
%%%%%
%%%%%
\end{document}
