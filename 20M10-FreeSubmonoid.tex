\documentclass[12pt]{article}
\usepackage{pmmeta}
\pmcanonicalname{FreeSubmonoid}
\pmcreated{2013-03-22 18:21:36}
\pmmodified{2013-03-22 18:21:36}
\pmowner{Ziosilvio}{18733}
\pmmodifier{Ziosilvio}{18733}
\pmtitle{free submonoid}
\pmrecord{5}{40998}
\pmprivacy{1}
\pmauthor{Ziosilvio}{18733}
\pmtype{Definition}
\pmcomment{trigger rebuild}
\pmclassification{msc}{20M10}
\pmclassification{msc}{20M05}
\pmdefines{minimal generating set of a submonoid}

\endmetadata

% this is the default PlanetMath preamble.  as your knowledge
% of TeX increases, you will probably want to edit this, but
% it should be fine as is for beginners.

% almost certainly you want these
\usepackage{amssymb}
\usepackage{amsmath}
\usepackage{amsfonts}

% used for TeXing text within eps files
%\usepackage{psfrag}
% need this for including graphics (\includegraphics)
%\usepackage{graphicx}
% for neatly defining theorems and propositions
%\usepackage{amsthm}
% making logically defined graphics
%%%\usepackage{xypic}

% there are many more packages, add them here as you need them

% define commands here

\begin{document}
\newcommand{\mgs}{\ensuremath{\mathrm{mgs}}}

Let $A$ be an arbitrary set,
let $A^\ast$ be the free monoid on $A$,
and let $e$ be the identity element (empty word) of $A^\ast$.

Let $M$ be a submonoid of $A^\ast$.
The \emph{minimal generating set} of $M$ is
\begin{equation} \label{eq:mgs}
\mgs(M) = (M\setminus\{e\})\setminus(M\setminus\{e\})^2\;.
\end{equation}
Shortly, $\mgs(M)$ is the set of all the nontrivial elements of $M$
that cannot be ``reconstructed'' as products of elements of $M$.
It is straightforward that
\begin{enumerate}
\item $(\mgs(M))^\ast=M$, and
\item if $S\subseteq A^\ast$ and $M\subseteq S^\ast$,
then $\mgs(M)\subseteq S$.
\end{enumerate}
We say that $M$ is a \emph{free submonoid} of $A^\ast$
if it is isomorphic (as a monoid)
to a free monoid $B^\ast$ for some set $B$.
A set $K\subseteq A^\ast$ such that $K=\mgs(M)$
for some free submonoid $M$ of $A^\ast$
is also called a \emph{code}.

%%%%%
%%%%%
\end{document}
