\documentclass[12pt]{article}
\usepackage{pmmeta}
\pmcanonicalname{AlternatingGroupHasIndex2InTheSymmetricGroupThe}
\pmcreated{2013-03-22 16:48:49}
\pmmodified{2013-03-22 16:48:49}
\pmowner{yesitis}{13730}
\pmmodifier{yesitis}{13730}
\pmtitle{alternating group has index 2 in the symmetric group, the}
\pmrecord{8}{39048}
\pmprivacy{1}
\pmauthor{yesitis}{13730}
\pmtype{Proof}
\pmcomment{trigger rebuild}
\pmclassification{msc}{20-00}

% this is the default PlanetMath preamble.  as your knowledge
% of TeX increases, you will probably want to edit this, but
% it should be fine as is for beginners.

% almost certainly you want these
\usepackage{amssymb}
\usepackage{amsmath}
\usepackage{amsfonts}

% used for TeXing text within eps files
%\usepackage{psfrag}
% need this for including graphics (\includegraphics)
%\usepackage{graphicx}
% for neatly defining theorems and propositions
%\usepackage{amsthm}
% making logically defined graphics
%%%\usepackage{xypic}

% there are many more packages, add them here as you need them

% define commands here

\begin{document}
We prove that the alternating group $A_n$ has index 2 in the symmetric group $S_n$, i.e., $A_n$ has the same cardinality as its complement $S_n\setminus A_n$. The proof is function-theoretic. Its idea is similar to the proof in the parent topic, but the focus is less on algebraic aspect.


Let $\pi\in S_n\setminus A_n$. Define $\pi:S_n \setminus A_n\rightarrow A_n$ by $\pi(\sigma)=\pi\sigma$, where $\pi\sigma$ is the product of $\pi$ and $\sigma$.

One-to-one:
\begin{equation*}
\pi(\sigma)=\pi(\delta) \Longrightarrow \sigma=\delta
\end{equation*}
since $\pi^{-1}$ exists and $\pi^{-1}\pi\sigma=\pi^{-1}\pi\delta$.

Onto:
Given $\alpha\in A_n$, there exists an element in $S_n\setminus A_n$, namely $\lambda=\pi^{-1}\alpha$, such that
\begin{equation*}
\pi(\alpha)=\lambda.
\end{equation*}
(The element $\lambda$ is in $S_n\setminus A_n$ because $\pi^{-1}$ is and the product of an odd permutation and an even permutation is odd.)

The function $\pi:S_n \setminus A_n\rightarrow A_n$ is, therefore, a one-to-one correspondence, so both sets $S_n \setminus A_n$ and $A_n$ have the same cardinality. 
%%%%%
%%%%%
\end{document}
