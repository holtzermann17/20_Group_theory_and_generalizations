\documentclass[12pt]{article}
\usepackage{pmmeta}
\pmcanonicalname{SimpleGroup}
\pmcreated{2013-03-22 12:23:35}
\pmmodified{2013-03-22 12:23:35}
\pmowner{yark}{2760}
\pmmodifier{yark}{2760}
\pmtitle{simple group}
\pmrecord{9}{32189}
\pmprivacy{1}
\pmauthor{yark}{2760}
\pmtype{Definition}
\pmcomment{trigger rebuild}
\pmclassification{msc}{20E32}
\pmrelated{Group}
\pmrelated{NormalSubgroup}
\pmdefines{simple}

\endmetadata


\begin{document}
A non-trivial group $G$ is said to be \emph{simple}
if the only normal subgroups of $G$ are $\{1\}$ and $G$ itself.

Equivalently, a simple group is a group in which the trivial subgroup is a maximal normal subgroup.
%%%%%
%%%%%
\end{document}
