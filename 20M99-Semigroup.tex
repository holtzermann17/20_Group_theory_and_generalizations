\documentclass[12pt]{article}
\usepackage{pmmeta}
\pmcanonicalname{Semigroup}
\pmcreated{2013-03-22 11:50:08}
\pmmodified{2013-03-22 11:50:08}
\pmowner{djao}{24}
\pmmodifier{djao}{24}
\pmtitle{semigroup}
\pmrecord{11}{30388}
\pmprivacy{1}
\pmauthor{djao}{24}
\pmtype{Definition}
\pmcomment{trigger rebuild}
\pmclassification{msc}{20M99}
\pmsynonym{homomorphism}{Semigroup}
\pmrelated{groupoid}
\pmrelated{Band2}
\pmrelated{SubmonoidSubsemigroup}
\pmrelated{NullSemigroup}
\pmrelated{ZeroElements}
\pmrelated{Monoid}
\pmdefines{semigroup homomorphism}

\endmetadata

\usepackage{amssymb}
\usepackage{amsmath}
\usepackage{amsfonts}
\usepackage{graphicx}
%%%%\usepackage{xypic}
\begin{document}
A {\em semigroup} $G$ is a set together with a binary operation $\cdot: G \times G \longrightarrow G$ which satisfies the associative property: $(a \cdot b) \cdot c = a \cdot (b \cdot c)$ for all $a,b,c \in G$.

The set $G$ is not required to be nonempty.

Let $G,H$ be two semigroups.  A \emph{semigroup homomorphism} from $G$ to $H$ is a function $f:G\to H$ such that $f(ab)=f(a)f(b)$.
%%%%%
%%%%%
%%%%%
%%%%%
\end{document}
