\documentclass[12pt]{article}
\usepackage{pmmeta}
\pmcanonicalname{HopfianGroup}
\pmcreated{2013-03-22 15:36:03}
\pmmodified{2013-03-22 15:36:03}
\pmowner{yark}{2760}
\pmmodifier{yark}{2760}
\pmtitle{Hopfian group}
\pmrecord{11}{37516}
\pmprivacy{1}
\pmauthor{yark}{2760}
\pmtype{Definition}
\pmcomment{trigger rebuild}
\pmclassification{msc}{20F99}
\pmrelated{HopfianModule}
\pmdefines{Hopfian}
\pmdefines{co-Hopfian}
\pmdefines{cohopfian}
\pmdefines{co-Hopfian group}
\pmdefines{cohopfian group}

\usepackage{amssymb}
\usepackage{amsmath}
\usepackage{amsfonts}
\begin{document}
\PMlinkescapeword{finite}
\PMlinkescapephrase{finite rank}
\PMlinkescapeword{infinite}
\PMlinkescapeword{theorem}

A group is said to be \emph{Hopfian} if it is not isomorphic to any of its proper \PMlinkname{quotients}{QuotientGroup}.
A group $G$ is Hopfian if and only if every surjective endomorphism $G\to G$ is an automorphism.

A group is said to be \emph{co-Hopfian} if it is not isomorphic to any of its proper subgroups.
A group $G$ is co-Hopfian if and only if every injective endomorphism $G\to G$ is an automorphism.

\section*{Examples}

Every finite group is obviously Hopfian and co-Hopfian.

The group of rationals is an example of an infinite group that is both Hopfian and co-Hopfian.

The group of integers is Hopfian, but not co-Hopfian.
More generally, every finitely generated abelian group is Hopfian,
but is not co-Hopfian unless it is finite.

Quasicyclic groups are co-Hopfian, but not Hopfian.

Free groups of infinite rank are neither Hopfian nor co-Hopfian.
By contrast, free groups of finite rank are Hopfian (though not co-Hopfian unless of rank zero).

By a theorem of Mal'cev, every finitely generated residually finite group is Hopfian.

The Baumslag-Solitar group with presentation $\langle b,t\mid t^{-1}b^2t=b^3\rangle$ is an example of a finitely generated group that is not Hopfian.
%%%%%
%%%%%
\end{document}
