\documentclass[12pt]{article}
\usepackage{pmmeta}
\pmcanonicalname{Pgroup}
\pmcreated{2013-03-22 14:53:08}
\pmmodified{2013-03-22 14:53:08}
\pmowner{yark}{2760}
\pmmodifier{yark}{2760}
\pmtitle{$p$-group}
\pmrecord{13}{36565}
\pmprivacy{1}
\pmauthor{yark}{2760}
\pmtype{Definition}
\pmcomment{trigger rebuild}
\pmclassification{msc}{20F50}
\pmsynonym{p-group}{Pgroup}
\pmsynonym{p-primary group}{Pgroup}
\pmsynonym{primary group}{Pgroup}
\pmrelated{PGroup}
\pmrelated{PExtension}
\pmrelated{ProPGroup}
\pmrelated{QuasicyclicGroup}
\pmrelated{Subgroup}
\pmdefines{p-subgroup}
\pmdefines{primary component}
\pmdefines{p-primary}
\pmdefines{p-primary subgroup}
\pmdefines{primary subgroup}

\usepackage{amssymb}
\usepackage{amsmath}
\usepackage{amsfonts}
\begin{document}
\PMlinkescapeword{component}
\PMlinkescapeword{maximal}
\PMlinkescapeword{subgroup}

\section*{Primary groups}

Let $p$ be a prime number.
A \emph{$p$-group} (or \emph{$p$-primary group}) is a group in which the order of every element is a power of $p$.
A group that is a $p$-group for some prime $p$ is also called a \emph{primary group}.

Using Lagrange's Theorem and Cauchy's Theorem one may show that a finite group $G$ is a $p$-group if and only if $|G|$ is a power of $p$.

\section*{Primary subgroups}

A \emph{$p$-subgroup} (or \emph{$p$-primary subgroup}) of a group $G$ is a \PMlinkname{subgroup}{Subgroup} $H$ of $G$ such that $H$ is also a $p$-group.
A group that is a $p$-subgroup for some prime $p$ is also called a \emph{primary subgroup}.

It follows from Zorn's Lemma that every group has a maximal $p$-subgroup, for every prime $p$. The maximal $p$-subgroup need not be unique (though for abelian groups it is always unique, and is called the \emph{$p$-primary component} of the abelian group). A maximal $p$-subgroup may, of course, be trivial. Non-trivial maximal $p$-subgroups of finite groups are called \PMlinkname{Sylow $p$-subgroups}{SylowPSubgroups}.
%%%%%
%%%%%
\end{document}
