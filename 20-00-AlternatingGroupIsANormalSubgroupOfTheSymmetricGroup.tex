\documentclass[12pt]{article}
\usepackage{pmmeta}
\pmcanonicalname{AlternatingGroupIsANormalSubgroupOfTheSymmetricGroup}
\pmcreated{2013-03-22 13:42:32}
\pmmodified{2013-03-22 13:42:32}
\pmowner{CWoo}{3771}
\pmmodifier{CWoo}{3771}
\pmtitle{alternating group is a normal subgroup of the symmetric group}
\pmrecord{8}{34387}
\pmprivacy{1}
\pmauthor{CWoo}{3771}
\pmtype{Theorem}
\pmcomment{trigger rebuild}
\pmclassification{msc}{20-00}

\endmetadata

\usepackage{latexsym}
\usepackage{amssymb}
\usepackage{amsfonts}
\usepackage[centertags]{amsmath}
\usepackage{amsfonts}
\usepackage{amssymb}
\usepackage{amsthm}
\usepackage{color}
\usepackage{graphicx}
\usepackage{latexsym}
\usepackage{amsfonts}
\newtheorem{theorem}{Theorem}
\begin{document}
\begin{theorem}
 The alternating group $A_{n}$ is a normal subgroup of the symmetric group $S_{n}$
  \end{theorem}

\begin{proof}
Define the epimorphism $f:S_{n}\rightarrow\mathbb{Z}_2$ by
$:\sigma\mapsto 0$ if $\sigma$ is an even permutation and
$:\sigma\mapsto 1$ if $\sigma$ is an odd permutation.  Hence,
$A_{n}$ is the kernel of $f$ and so it is a normal subgroup of the
domain $S_{n}$. Furthermore $S_{n}/A_{n}\cong\mathbb{Z}_2$ by
the first isomorphism theorem. So by Lagrange's theorem
$$
\vert S_{n} \vert=\vert A_{n} \vert\vert S_{n}/A_{n}\vert.
$$
Therefore, $\vert A_{n}\vert=n!/2$. That is, there are $n!/2$ many
elements in $A_{n}$
\end{proof}

\textbf{Remark}.  What we have shown in the theorem is that, in fact, $A_n$ has index $2$ in $S_n$.  In general, if a subgroup $H$ of $G$ has index $2$, then $H$ is normal in $G$. (Since $[G:H]=2$, there is an element $g\in G-H$, so that $gH\cap H=\varnothing$ and thus $gH=Hg$).  
%%%%%
%%%%%
\end{document}
