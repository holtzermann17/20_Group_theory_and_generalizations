\documentclass[12pt]{article}
\usepackage{pmmeta}
\pmcanonicalname{CayleysTheoremForSemigroups}
\pmcreated{2013-03-22 19:04:37}
\pmmodified{2013-03-22 19:04:37}
\pmowner{Ziosilvio}{18733}
\pmmodifier{Ziosilvio}{18733}
\pmtitle{Cayley's theorem for semigroups}
\pmrecord{8}{41964}
\pmprivacy{1}
\pmauthor{Ziosilvio}{18733}
\pmtype{Theorem}
\pmcomment{trigger rebuild}
\pmclassification{msc}{20M20}
\pmclassification{msc}{20M15}
\pmrelated{CayleysTheorem}

% this is the default PlanetMath preamble.  as your knowledge
% of TeX increases, you will probably want to edit this, but
% it should be fine as is for beginners.

% almost certainly you want these
\usepackage{amssymb}
\usepackage{amsmath}
\usepackage{amsfonts}

% used for TeXing text within eps files
%\usepackage{psfrag}
% need this for including graphics (\includegraphics)
%\usepackage{graphicx}
% for neatly defining theorems and propositions
%\usepackage{amsthm}
% making logically defined graphics
%%%\usepackage{xypic}

% there are many more packages, add them here as you need them

% define commands here
\newcommand{\restrict}[2]{\left.{#1}\right|_{#2}}
\newtheorem{theorem}{Theorem}
\begin{document}
Let $X$ be a set.
We can define on $X^X$,
the set of functions from $X$ to itself,
a structure of semigroup by putting
$f\otimes g=g\circ f$.
Such semigroup is actually a monoid,
whose identity element is the identity function of $X$.
\begin{theorem}[Cayley's theorem for semigroups] \label{thm:cay-sg}
For every semigroup $(S,\cdot)$
there exist a set $X$
and an injective map $\phi:S\to X^X$
which is a morphism of semigroups from $(S,\cdot)$ to $(X^X,\otimes)$.
\end{theorem}
In other words,
every semigroup is isomorphic to
a semigroup of transformations of some set.
This is an extension of Cayley's theorem on groups,
which states that every group is isomorphic to
a group of \emph{invertible} transformations of some set.

\textit{Proof of Theorem~\ref{thm:cay-sg}.}
The argument is similar to the one for Cayley's theorem on groups.
Let $X=S$, the set of elements of the semigroup.

First, suppose $(S,\cdot)$ is a monoid with unit $e$.
For $s\in S$ define $f_s:S\to S$ as
\begin{equation} \label{eq:iso}
f_s(x)=x\cdot s\;\forall x\in S\,.
\end{equation}
Then for every $s,t,x\in S$ we have
\begin{eqnarray*}
f_{s\cdot t}(x)
& = & x\cdot(s\cdot t) \\
& = & (x\cdot s)\cdot t \\
& = & f_t(x\cdot s) \\
& = & f_t(f_s(x)) \\
& = & (f_t\circ f_s)(x) \\
& = & (f_s\otimes f_t)(x)\,,
\end{eqnarray*}
so $\phi(s)=f_s$ is a homomorphism of monoids,
with $f_e=\mathrm{id}_S$.
This homomorphism is injective,
because if $f_s=f_t$,
then $s=f_s(e)=f_t(e)=t$.

Next, suppose $(S,\cdot)$ is a semigroup but not a monoid.
Let $e\not\in S$.
Construct a monoid $(M,\ast)$ by putting $M=S\cup\{e\}$ and defining
\begin{displaymath}
s\ast t=\left\{\begin{array}{ll}
s\cdot t & \mathrm{if}\;s,t\in S, \\
s & \mathrm{if}\;s\in S,t=e, \\
t & \mathrm{if}\;s=e,t\in S, \\
e & \mathrm{if}\;s=t=e. \\
\end{array}\right.
\end{displaymath}
Then $(M,\ast)$ is isomorphic to a submonoid of $(M^M,\otimes)$
as by (\ref{eq:iso}).
For $s\in S$ put $g_s=\restrict{f_s}{S}$:
then $g_s\in S^S$ for every $s$,
$g_{s\cdot t}=\restrict{f_{s\ast t}}{S}$,
and $(S,\cdot)$ is isomorphic to $(\Sigma,\otimes)$
with $\Sigma=\{g_s\mid s\in S\}$.
$\Box$

Observe that the theorem remains valid
if $f\otimes g$ is defined as $f\circ g$.
In this case, the morphism $\phi$ is defined by
$f_s(x)=s\cdot x\,\forall x\in S$.
%%%%%
%%%%%
\end{document}
