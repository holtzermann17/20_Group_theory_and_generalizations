\documentclass[12pt]{article}
\usepackage{pmmeta}
\pmcanonicalname{Difference}
\pmcreated{2013-03-22 17:33:35}
\pmmodified{2013-03-22 17:33:35}
\pmowner{pahio}{2872}
\pmmodifier{pahio}{2872}
\pmtitle{difference}
\pmrecord{16}{39966}
\pmprivacy{1}
\pmauthor{pahio}{2872}
\pmtype{Definition}
\pmcomment{trigger rebuild}
\pmclassification{msc}{20K99}
\pmclassification{msc}{00A05}
\pmclassification{msc}{11B25}
\pmrelated{VectorDifference}
\pmrelated{SetDifference}
\pmrelated{Multiple}
\pmrelated{GeneralAssociativity}
\pmrelated{Quotient}
\pmrelated{DifferenceOfVectors}
\pmdefines{minuend}
\pmdefines{subtrahend}

\endmetadata

% this is the default PlanetMath preamble.  as your knowledge
% of TeX increases, you will probably want to edit this, but
% it should be fine as is for beginners.

% almost certainly you want these
\usepackage{amssymb}
\usepackage{amsmath}
\usepackage{amsfonts}

% used for TeXing text within eps files
%\usepackage{psfrag}
% need this for including graphics (\includegraphics)
%\usepackage{graphicx}
% for neatly defining theorems and propositions
 \usepackage{amsthm}
% making logically defined graphics
%%%\usepackage{xypic}

% there are many more packages, add them here as you need them

% define commands here

\theoremstyle{definition}
\newtheorem*{thmplain}{Theorem}

\begin{document}
The {\em difference of} two numbers $a$ and $b$ is a number $d$ such that 
$$b\!+\!d \;=\; a.$$
The difference of $a$ (the {\em minuend}) and $b$ (the {\em subtrahend}) is denoted by $a\!-\!b$.

The definition is \PMlinkescapetext{similar} for the elements $a,\,b$ of any \PMlinkescapetext{additive} Abelian group (e.g. of a vector space).\,The difference of them is always unique.\\

\textbf{Note 1.}\, Forming the difference of numbers (resp. elements), i.e. subtraction, is in a certain sense converse to the addition operation:
$$(x\!+\!y)\!-\!y \;=\; x$$

\textbf{Note 2.}\, As for real numbers, one may say that the difference \emph{between} $a$ and $b$ is $|a\!-\!b|$ (which is the same as $|b\!-\!a|$); then it is always nonnegative.\, For all complex numbers, such a phrase would be nonsense.\\


\textbf{Some \PMlinkescapetext{identities}}
\begin{itemize}
\item $b\!+\!(a\!-\!b) \;=\; a$
\item $a\!-\!b \;=\; a\!+\!(-b)$
\item $-(a\!-\!b) \;=\; b\!-\!a$
\item $n(a\!-\!b) \;=\; na\!-\!nb \quad (n\in\mathbb{Z})$
\item $a\!-\!a \;=\; 0$
\end{itemize}




%%%%%
%%%%%
\end{document}
