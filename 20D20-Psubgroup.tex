\documentclass[12pt]{article}
\usepackage{pmmeta}
\pmcanonicalname{Psubgroup}
\pmcreated{2013-03-22 14:02:14}
\pmmodified{2013-03-22 14:02:14}
\pmowner{drini}{3}
\pmmodifier{drini}{3}
\pmtitle{$p$-subgroup}
\pmrecord{8}{35184}
\pmprivacy{1}
\pmauthor{drini}{3}
\pmtype{Definition}
\pmcomment{trigger rebuild}
\pmclassification{msc}{20D20}
\pmrelated{PGroup4}
\pmdefines{Sylow $p$-subgroup}
\pmdefines{Sylow p-subgroup}
\pmdefines{$p$-group}
\pmdefines{p-group}

\usepackage{graphicx}
%%%\usepackage{xypic} 
\usepackage{bbm}
\newcommand{\Z}{\mathbbmss{Z}}
\newcommand{\C}{\mathbbmss{C}}
\newcommand{\R}{\mathbbmss{R}}
\newcommand{\Q}{\mathbbmss{Q}}
\newcommand{\mathbb}[1]{\mathbbmss{#1}}
\newcommand{\figura}[1]{\begin{center}\includegraphics{#1}\end{center}}
\newcommand{\figuraex}[2]{\begin{center}\includegraphics[#2]{#1}\end{center}}
\begin{document}
Let $G$ be a finite group with order $n$, and let $p$ be a prime integer.
We can write $n=p^k m$ for some $k,m$ integers, such that $k$ and $m$ are coprimes (that is, $p^k$ is  the highest power of $p$ that divides $n$).
Any subgroup of $G$ whose order is $p^k$ is called a Sylow $p$-subgroup.

While there is no reason for Sylow $p$-subgroups to exist for any finite group, the fact is that all groups have Sylow $p$-subgroups for every prime $p$ that divides $|G|$. This statement is the First Sylow theorem
	
When $|G|=p^k$ we simply say that $G$ is a $p$-group.
%%%%%
%%%%%
\end{document}
