\documentclass[12pt]{article}
\usepackage{pmmeta}
\pmcanonicalname{HNNExtension}
\pmcreated{2013-03-22 16:04:03}
\pmmodified{2013-03-22 16:04:03}
\pmowner{juanman}{12619}
\pmmodifier{juanman}{12619}
\pmtitle{HNN extension}
\pmrecord{8}{38122}
\pmprivacy{1}
\pmauthor{juanman}{12619}
\pmtype{Definition}
\pmcomment{trigger rebuild}
\pmclassification{msc}{20E06}
\pmrelated{GroupExtension}

\endmetadata

% this is the default PlanetMath preamble.  as your knowledge
% of TeX increases, you will probably want to edit this, but
% it should be fine as is for beginners.

% almost certainly you want these
\usepackage{amssymb}
\usepackage{amsmath}
\usepackage{amsfonts}

% used for TeXing text within eps files
%\usepackage{psfrag}
% need this for including graphics (\includegraphics)
%\usepackage{graphicx}
% for neatly defining theorems and propositions
%\usepackage{amsthm}
% making logically defined graphics
%%%\usepackage{xypic}

% there are many more packages, add them here as you need them

% define commands here

\begin{document}
The \emph{HNN extension group} $G$ for a group $A$, is constructed from a pair of isomorphic subgroups $B\stackrel{\phi}\cong C$ in $A$, according to formula 
$$G=\frac{A*\langle t|- \rangle}{N}$$
where 
$\langle t|- \rangle$ is a cyclic free group, $*$ is the free product and $N$ is the normal closure of $\{tbt^{-1}\phi(b)^{-1}\colon b\in B\}$. 

As an example take a surface bundle $F\subset E\to S^1$, hence the homotopy long exact sequence of this bundle implies that the fundamental group $\pi_1(E)$ is given by
$$\pi_1(E)=\langle x_1,...,x_k,t|\Pi=1,tx_it^{-1}=\phi(x_i)\rangle$$
where $k$ is the genus of the surface and the relation $\Pi$ is $[x_1,x_2][x_3,x_4]\cdots[x_{k-1},x_k]$ for an orientable surface or $x_1^2x_2^2\cdots x_k^2$ is for a non-orientable one. $\phi$ is an isomorphism induced by a self homeomorphism of $F$. 
%%%%%
%%%%%
\end{document}
