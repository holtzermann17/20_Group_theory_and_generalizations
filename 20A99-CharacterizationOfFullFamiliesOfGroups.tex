\documentclass[12pt]{article}
\usepackage{pmmeta}
\pmcanonicalname{CharacterizationOfFullFamiliesOfGroups}
\pmcreated{2013-03-22 18:36:08}
\pmmodified{2013-03-22 18:36:08}
\pmowner{joking}{16130}
\pmmodifier{joking}{16130}
\pmtitle{characterization of full families of groups}
\pmrecord{9}{41333}
\pmprivacy{1}
\pmauthor{joking}{16130}
\pmtype{Derivation}
\pmcomment{trigger rebuild}
\pmclassification{msc}{20A99}

\endmetadata

% this is the default PlanetMath preamble.  as your knowledge
% of TeX increases, you will probably want to edit this, but
% it should be fine as is for beginners.

% almost certainly you want these
\usepackage{amssymb}
\usepackage{amsmath}
\usepackage{amsfonts}

% used for TeXing text within eps files
%\usepackage{psfrag}
% need this for including graphics (\includegraphics)
%\usepackage{graphicx}
% for neatly defining theorems and propositions
%\usepackage{amsthm}
% making logically defined graphics
%%%\usepackage{xypic}

% there are many more packages, add them here as you need them

% define commands here

\begin{document}
\textbf{Proposition.} Let $\mathcal{G}=\{G_k\}_{k\in I}$ be a family of groups. Then $\mathcal{G}$ is full if and only if for any $i,j\in I$ such that $i\neq j$ we have that any homomorphism $f:G_i\to G_j$ is trivial.

\textit{Proof.} ,,$\Rightarrow$'' Assume that $f:G_i\to G_j$ is a nontrivial group homomorphism. Then define $$h:\bigoplus_{k\in I} G_k\to \bigoplus_{k\in I} G_k$$ as follows: if $t\in I$ is such that $t\neq i$ and $g\in\bigoplus_{k\in I}G_k$ is such that $g\in G_t$, then $h(g)=g$. If $g\in\bigoplus_{k\in I}G_k$ is such that $g\in G_i$, then $h(g)(j)=f(g(i))$ and $h(g)(k)=0$ for $k\neq j$. This values uniquely define $h$ and one can easily check that $h$ is not decomposable. $\square$

,,$\Leftarrow$'' Assume that for any $i,j\in I$ such that $i\neq j$ we have that any homomorphism $f:G_i\to G_j$ is trivial. Let $$h:\bigoplus_{k\in I} G_k\to \bigoplus_{k\in I} G_k$$ be any homomorphism. Moreover, let $i\in I$ and $g\in\bigoplus_{k\in I} G_k$ be such that $g\in G_i$. We wish to show that $h(g)\in G_i$.

So assume that $h(g)\not\in G_i$. Then there exists $j\neq i$ such that $0\neq h(g)(j)\in G_j$. Let $$\pi:\bigoplus_{k\in I} G_k\to G_j$$ be the projection and let $$u:G_i\to \bigoplus_{k\in I} G_k$$ be the natural inclusion homomorphism. Then $\pi\circ u:G_i\to G_j$ is a nontrivial group homomorphism. Contradiction. $\square$


\textbf{Corollary.} Assume that $\{G_k\}_{k\in I}$ is a family of nontrivial groups such that $G_i$ is periodic for each $i\in I$. Moreover assume that for any $i,j\in I$ such that $i\neq j$ and any $g\in G_i$, $h\in G_j$ orders $|g|$ and $|h|$ are realitvely prime (which implies that $I$ is countable). Then $\{G_k\}_{k\in I}$ is full.

\textit{Proof.} Assume that $i\neq j$ and $f:G_i\to G_j$ is a group homomorphism. Then $|f(g)|$ divides $|g|$ for any $g\in G_i$. But $f(g)\in G_j$, so $|g|$ and $|f(g)|$ are relatively prime. Thus $|f(g)|=1$, so $f(g)=0$. Therefore $f$ is trivial, which (due to proposition) completes the proof. $\square$
%%%%%
%%%%%
\end{document}
