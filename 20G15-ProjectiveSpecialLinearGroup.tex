\documentclass[12pt]{article}
\usepackage{pmmeta}
\pmcanonicalname{ProjectiveSpecialLinearGroup}
\pmcreated{2013-03-22 15:09:46}
\pmmodified{2013-03-22 15:09:46}
\pmowner{alozano}{2414}
\pmmodifier{alozano}{2414}
\pmtitle{projective special linear group}
\pmrecord{4}{36912}
\pmprivacy{1}
\pmauthor{alozano}{2414}
\pmtype{Definition}
\pmcomment{trigger rebuild}
\pmclassification{msc}{20G15}
\pmsynonym{PSL}{ProjectiveSpecialLinearGroup}
\pmrelated{TheoremsOfSpecialLinearGroupOverAFiniteField}

% this is the default PlanetMath preamble.  as your knowledge
% of TeX increases, you will probably want to edit this, but
% it should be fine as is for beginners.

% almost certainly you want these
\usepackage{amssymb}
\usepackage{amsmath}
\usepackage{amsthm}
\usepackage{amsfonts}

% used for TeXing text within eps files
%\usepackage{psfrag}
% need this for including graphics (\includegraphics)
%\usepackage{graphicx}
% for neatly defining theorems and propositions
%\usepackage{amsthm}
% making logically defined graphics
%%%\usepackage{xypic}

% there are many more packages, add them here as you need them

% define commands here

\newtheorem{thm}{Theorem}
\newtheorem*{defn}{Definition}
\newtheorem{prop}{Proposition}
\newtheorem{lemma}{Lemma}
\newtheorem{cor}{Corollary}

\theoremstyle{definition}
\newtheorem{exa}{Example}

% Some sets
\newcommand{\Nats}{\mathbb{N}}
\newcommand{\Ints}{\mathbb{Z}}
\newcommand{\Reals}{\mathbb{R}}
\newcommand{\Complex}{\mathbb{C}}
\newcommand{\Rats}{\mathbb{Q}}
\newcommand{\Gal}{\operatorname{Gal}}
\newcommand{\Cl}{\operatorname{Cl}}
\newcommand{\SL}{\operatorname{SL}}
\newcommand{\PSL}{\operatorname{PSL}}
\begin{document}
\begin{defn}
Let $V$ be a vector space over a field $F$ and let $\SL(V)$ be the special linear group. Let $Z$ be the center of $\SL(V)$. The {\bf projective special linear group} associated to $V$ is the quotient group $\SL(V)/Z$ and is usually denoted by $\PSL(V)$. 

When $V$ is a finite dimensional vector space over $F$ (of dimension $n$) then we write $\PSL(n,F)$ or $\PSL_n(F)$. We also identify the linear transformations of $V$ with $n\times n$ matrices, so $\PSL$ may be regarded as a quotient of the group of matrices $\SL(n,F)$ by its center.
\end{defn}

Note: see the entry on projective space for the origin of the terminology.

\begin{thm}
The center $Z$ of $\SL(n,F)$ is the group of all scalar matrices $\lambda\cdot \operatorname{Id}$ where $\lambda$ is an $n$th root of unity in $F$.

In particular, for $n=2$, $Z=\{ \pm \operatorname{Id} \}$ and:
$$\PSL(2,F)=\SL(2,F)/\{ \pm \operatorname{Id} \}.$$
\end{thm}

As a consequence of the previous theorem, we obtain:

\begin{thm}
For $n\geq 3$, $\PSL(n,F)$ is a simple group. 
Furthermore, if $\mathbb{F}$ is a finite field then the groups 
$$\PSL(n,\mathbb{F})=\SL(n,\mathbb{F})/Z,\quad n\geq 2$$
are all finite simple groups, except for $n=2$ and $\mathbb{F}=\mathbb{F}_2,\mathbb{F}_3$.
\end{thm}

\begin{thebibliography}{00}
\bibitem{lang} S. Lang, {\em Algebra}, Springer-Verlag, New York.
\bibitem{dummit} D. Dummit, R. Foote, {\em Abstract Algebra},
Second Edition, Wiley.
\end{thebibliography}
%%%%%
%%%%%
\end{document}
