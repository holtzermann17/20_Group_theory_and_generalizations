\documentclass[12pt]{article}
\usepackage{pmmeta}
\pmcanonicalname{NormalityOfSubgroupsOfPrimeIndex}
\pmcreated{2013-03-22 17:26:38}
\pmmodified{2013-03-22 17:26:38}
\pmowner{azdbacks4234}{14155}
\pmmodifier{azdbacks4234}{14155}
\pmtitle{normality of subgroups of prime index}
\pmrecord{13}{39825}
\pmprivacy{1}
\pmauthor{azdbacks4234}{14155}
\pmtype{Theorem}
\pmcomment{trigger rebuild}
\pmclassification{msc}{20A05}
%\pmkeywords{prime}
%\pmkeywords{index}
%\pmkeywords{normal}
%\pmkeywords{subgroup}
%\pmkeywords{group action}
\pmrelated{Coset}
\pmrelated{GroupAction}
\pmrelated{ASubgroupOfIndex2IsNormal}

\endmetadata

%packages
\usepackage{amsmath,mathrsfs,amsfonts,amsthm}
%theorem environments
\theoremstyle{plain}
\newtheorem*{thm*}{Theorem}
\newtheorem*{lem*}{Lemma}
\newtheorem*{cor*}{Corollary}
\newtheorem*{prop*}{Proposition}
%delimiters
\newcommand{\set}[1]{\{#1\}}
\newcommand{\medset}[1]{\big\{#1\big\}}
\newcommand{\bigset}[1]{\bigg\{#1\bigg\}}
\newcommand{\Bigset}[1]{\Bigg\{#1\Bigg\}}
\newcommand{\abs}[1]{\vert#1\vert}
\newcommand{\medabs}[1]{\big\vert#1\big\vert}
\newcommand{\bigabs}[1]{\bigg\vert#1\bigg\vert}
\newcommand{\Bigabs}[1]{\Bigg\vert#1\Bigg\vert}
\newcommand{\norm}[1]{\Vert#1\Vert}
\newcommand{\mednorm}[1]{\big\Vert#1\big\Vert}
\newcommand{\bignorm}[1]{\bigg\Vert#1\bigg\Vert}
\newcommand{\Bignorm}[1]{\Bigg\Vert#1\Bigg\Vert}
\newcommand{\vbrack}[1]{\langle#1\rangle}
\newcommand{\medvbrack}[1]{\big\langle#1\big\rangle}
\newcommand{\bigvbrack}[1]{\bigg\langle#1\bigg\rangle}
\newcommand{\Bigvbrack}[1]{\Bigg\langle#1\Bigg\rangle}
\newcommand{\sbrack}[1]{[#1]}
\newcommand{\medsbrack}[1]{\big[#1\big]}
\newcommand{\bigsbrack}[1]{\bigg[#1\bigg]}
\newcommand{\Bigsbrack}[1]{\Bigg[#1\Bigg]}
%operators
\DeclareMathOperator{\Hom}{Hom}
\DeclareMathOperator{\Tor}{Tor}
\DeclareMathOperator{\Ext}{Ext}
\DeclareMathOperator{\Aut}{Aut}
\DeclareMathOperator{\End}{End}
\DeclareMathOperator{\Inn}{Inn}
\DeclareMathOperator{\lcm}{lcm}
\DeclareMathOperator{\ord}{ord}
\DeclareMathOperator{\rank}{rank}
\DeclareMathOperator{\tr}{tr}
\DeclareMathOperator{\Mat}{Mat}
\DeclareMathOperator{\Gal}{Gal}
\DeclareMathOperator{\GL}{GL}
\DeclareMathOperator{\SL}{SL}
\DeclareMathOperator{\SO}{SO}
\DeclareMathOperator{\ann}{ann}
\DeclareMathOperator{\im}{im}
\DeclareMathOperator{\Char}{char}
\DeclareMathOperator{\Spec}{Spec}
\DeclareMathOperator{\supp}{supp}
\DeclareMathOperator{\diam}{diam}
\DeclareMathOperator{\Ind}{Ind}
\DeclareMathOperator{\vol}{vol}

\begin{document}
\begin{prop*}
If $H$ is a subgroup of a finite group $G$ of index $p$, where $p$ is the smallest prime dividing the order of $G$, then $H$ is normal in $G$.
\end{prop*}
\begin{proof}
Suppose $H\leq G$ with $\abs{G}$ finite and $\abs{G:H}=p$, where $p$ is the smallest prime divisor of $\abs{G}$, let $G$ act on the set $L$ of left cosets of $H$ in $G$ by left \PMlinkescapetext{translation}, and let $\varphi:G\rightarrow S_p$ be the \PMlinkid{homomorphism induced}{3820} by this action. Now, if $g\in\ker\varphi$, then $gxH=xH$ for each $x\in G$, and in particular, $gH=H$, whence $g\in H$. Thus $K=\ker\varphi$ is a normal subgroup of $H$ (being contained in $H$ and normal in $G$). By the First Isomorphism Theorem, $G/K$ is isomorphic to a subgroup of $S_p$, and consequently $\abs{G/K}=\abs{G:K}$ must \PMlinkid{divide}{923} $p!$; moreover, any divisor of $\abs{G:K}$ must also \PMlinkescapetext{divide} $\abs{G}=\abs{G:K}\abs{K}$, and because $p$ is the smallest divisor of $\abs{G}$ different from $1$, the only possibilities are $\abs{G:K}=p$ or $\abs{G:K}=1$. But $\abs{G:K}=\abs{G:H}\abs{H:K}=p\abs{H:K}\geq p$, which \PMlinkescapetext{forces} $\abs{G:K}=p$, and consequently $\abs{H:K}=1$, so that $H=K$, from which it follows that $H$ is normal in $G$.
\end{proof}

%%%%%
%%%%%
\end{document}
