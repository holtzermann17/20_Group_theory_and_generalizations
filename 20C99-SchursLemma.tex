\documentclass[12pt]{article}
\usepackage{pmmeta}
\pmcanonicalname{SchursLemma}
\pmcreated{2013-03-22 13:08:01}
\pmmodified{2013-03-22 13:08:01}
\pmowner{rmilson}{146}
\pmmodifier{rmilson}{146}
\pmtitle{Schur's lemma}
\pmrecord{22}{33570}
\pmprivacy{1}
\pmauthor{rmilson}{146}
\pmtype{Theorem}
\pmcomment{trigger rebuild}
\pmclassification{msc}{20C99}
\pmclassification{msc}{20C15}
\pmrelated{GroupRepresentation}
\pmrelated{DenseRingOfLinearTransformations}

\usepackage{amsmath,amsfonts,amssymb,amsthm}
\newtheorem*{lemma}{Lemma}
\newtheorem*{corollary}{Corollary}


\newcommand{\image}{\operatorname{im}}
\begin{document}
\PMlinkescapeword{theory}
\PMlinkescapeword{implications}
\PMlinkescapeword{language}
\PMlinkescapeword{irreducible}

Schur's lemma is a fundamental result in representation theory,
an elementary observation about irreducible modules, which is nonetheless
noteworthy because of its profound applications.

\begin{lemma}[Schur's lemma]
  Let $G$ be a finite group and let $V$ and $W$ be irreducible
  $G$-modules. Then, every $G$-module homomorphism $f: V \to W$ is
  either invertible or the trivial zero map.
\end{lemma}
\begin{proof}
  Note that both the kernel, $\ker f$, and the image, $\image f$, are $G$-submodules of $V$ and
  $W$, respectively.  Since $V$ is irreducible, $\ker f$ is either
  trivial or all of $V$. In the former case, $\image f$ is all of $W$
  --- also because $W$ is irreducible --- and  hence $f$ is invertible. In
  the latter case, $f$ is the zero map.
\end{proof}

One of the most important consequences of Schur's lemma is the following.

\begin{corollary}
  Let $V$ be a finite-dimensional, irreducible $G$-module taken over
  an algebraically closed field. Then, every $G$-module homomorphism
  $f: V \to V$ is equal to a scalar multiplication.
\end{corollary}
\begin{proof}
  Since the ground field is algebraically closed, the linear
  transformation $f: V\to V$ has an eigenvalue; call it $\lambda$.
  By definition, $f - \lambda 1$ is not invertible, and hence equal to
  zero by Schur's lemma. In other words, $f = \lambda$, a scalar.
\end{proof}
%%%%%
%%%%%
\end{document}
