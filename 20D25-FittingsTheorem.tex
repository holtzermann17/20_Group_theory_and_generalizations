\documentclass[12pt]{article}
\usepackage{pmmeta}
\pmcanonicalname{FittingsTheorem}
\pmcreated{2013-03-22 13:51:39}
\pmmodified{2013-03-22 13:51:39}
\pmowner{yark}{2760}
\pmmodifier{yark}{2760}
\pmtitle{Fitting's theorem}
\pmrecord{12}{34600}
\pmprivacy{1}
\pmauthor{yark}{2760}
\pmtype{Theorem}
\pmcomment{trigger rebuild}
\pmclassification{msc}{20D25}
\pmdefines{Fitting subgroup}
\pmdefines{Fitting group}

\endmetadata


\begin{document}
\PMlinkescapeword{subgroup}
\PMlinkescapeword{subgroups}
\PMlinkescapeword{states}
\PMlinkescapeword{sum}

\emph{Fitting's Theorem} states that if $G$ is a group
and $M$ and $N$ are normal nilpotent \PMlinkname{subgroups}{Subgroup} of $G$,
then $MN$ is also a normal nilpotent subgroup
(of nilpotency class less than or equal to
the sum of the nilpotency classes of $M$ and $N$).

Thus, any finite group has a unique largest normal nilpotent subgroup, called its \emph{Fitting subgroup}.
More generally, the Fitting subgroup of a group $G$ is defined to be the subgroup of $G$ generated by the normal nilpotent subgroups of $G$;
Fitting's Theorem shows that the Fitting subgroup is always locally nilpotent.
A group that is equal to its own Fitting subgroup is sometimes called a \emph{Fitting group}.
%%%%%
%%%%%
\end{document}
