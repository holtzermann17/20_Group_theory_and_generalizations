\documentclass[12pt]{article}
\usepackage{pmmeta}
\pmcanonicalname{SemigroupWithInvolution}
\pmcreated{2013-03-22 16:11:24}
\pmmodified{2013-03-22 16:11:24}
\pmowner{Mazzu}{14365}
\pmmodifier{Mazzu}{14365}
\pmtitle{semigroup with involution}
\pmrecord{7}{38281}
\pmprivacy{1}
\pmauthor{Mazzu}{14365}
\pmtype{Definition}
\pmcomment{trigger rebuild}
\pmclassification{msc}{20M10}
\pmsynonym{*-semigroup}{SemigroupWithInvolution}
%\pmkeywords{semigroup}
\pmrelated{ISemigroup}
\pmrelated{ThueSystem}
\pmdefines{involution}
\pmdefines{semigroup with involution}
\pmdefines{monoid with involution}

\endmetadata

% this is the default PlanetMath preamble.  as your knowledge
% of TeX increases, you will probably want to edit this, but
% it should be fine as is for beginners.

% almost certainly you want these
\usepackage{amssymb}
\usepackage{amsmath}
\usepackage{amsfonts}

% used for TeXing text within eps files
%\usepackage{psfrag}
% need this for including graphics (\includegraphics)
%\usepackage{graphicx}
% for neatly defining theorems and propositions
%\usepackage{amsthm}
% making logically defined graphics
%%%\usepackage{xypic}

% there are many more packages, add them here as you need them

% define commands here

\begin{document}
An \emph{involution} on a semigroup $S$ [on a monoid $M$] is a unary operation $x\mapsto x^*$ defined on $S$ [resp. on $M$] such that for each $x,y\in S$ [resp. for each $x,y\in M$] $$(x^*)^*=x,\ \ \ (xy)^*=y^*x^*.$$
With this added internal operation, the semigroup $S$ [resp. the monoid $M$] becomes a \emph{semigroup with involution} or \emph{$*$-semigroup}  [resp. a \emph{monoid with involution} or \emph{$*$-monoid}].

\begin{thebibliography}{9}
\bibitem{b:howie} J.M. Howie, \emph{Fundamentals of Semigroup Theory}, Oxford University Press, Oxford, 1991.
\end{thebibliography}
%%%%%
%%%%%
\end{document}
