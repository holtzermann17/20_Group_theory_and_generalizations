\documentclass[12pt]{article}
\usepackage{pmmeta}
\pmcanonicalname{NormalIsNotTransitive}
\pmcreated{2013-03-22 16:00:34}
\pmmodified{2013-03-22 16:00:34}
\pmowner{Wkbj79}{1863}
\pmmodifier{Wkbj79}{1863}
\pmtitle{normal is not transitive}
\pmrecord{9}{38043}
\pmprivacy{1}
\pmauthor{Wkbj79}{1863}
\pmtype{Definition}
\pmcomment{trigger rebuild}
\pmclassification{msc}{20A05}
\pmclassification{msc}{12F10}
\pmrelated{ExampleOfNormalExtension}
\pmrelated{NormalityOfSubgroupsIsNotTransitive}

\endmetadata

% this is the default PlanetMath preamble.  as your knowledge
% of TeX increases, you will probably want to edit this, but
% it should be fine as is for beginners.

% almost certainly you want these
\usepackage{amssymb}
\usepackage{amsmath}
\usepackage{amsfonts}

% used for TeXing text within eps files
%\usepackage{psfrag}
% need this for including graphics (\includegraphics)
%\usepackage{graphicx}
% for neatly defining theorems and propositions
%\usepackage{amsthm}
% making logically defined graphics
%%%\usepackage{xypic}

% there are many more packages, add them here as you need them

% define commands here

\begin{document}
The phrase ``normal is not transitive'' can be used as a mnemonic for two statements.

The first is: ``The relation `is a normal subgroup of' is not transitive.''  This means that, if $H \triangleleft N \triangleleft G$, it does not follow that $H \triangleleft G$.  See normality of subgroups is not transitive for more details.

The second is: ``The relation `is a normal extension of' is not transitive.''  This means that, if $K/F$ and $L/K$ are normal extensions, it does not follow that $L/F$ is normal.  See example of normal extension for more details.
%%%%%
%%%%%
\end{document}
