\documentclass[12pt]{article}
\usepackage{pmmeta}
\pmcanonicalname{MetacyclicGroup}
\pmcreated{2013-03-22 15:36:39}
\pmmodified{2013-03-22 15:36:39}
\pmowner{yark}{2760}
\pmmodifier{yark}{2760}
\pmtitle{metacyclic group}
\pmrecord{5}{37531}
\pmprivacy{1}
\pmauthor{yark}{2760}
\pmtype{Definition}
\pmcomment{trigger rebuild}
\pmclassification{msc}{20F16}
\pmdefines{metacyclic}

\endmetadata

\usepackage{amssymb}
\usepackage{amsmath}
\usepackage{amsfonts}

%\usepackage{psfrag}
%\usepackage{graphicx}
%\usepackage{amsthm}
%%%\usepackage{xypic}

\renewcommand{\le}{\leqslant}
\renewcommand{\ge}{\geqslant}
\renewcommand{\leq}{\leqslant}
\renewcommand{\geq}{\geqslant}
\begin{document}
\PMlinkescapeword{subgroup}
\PMlinkescapeword{subgroups}

\section*{Definition}

A \emph{metacyclic group} is a group $G$ that possesses a normal subgroup $N$ such that $N$ and $G/N$ are both cyclic.

\section*{Examples}

\begin{itemize}
\item All cyclic groups, and direct products of two cyclic groups.
\item All dihedral groups (including the infinite dihedral group).
\item All finite groups whose Sylow subgroups are cyclic (and so, in particular, all finite groups of \PMlinkname{squarefree}{SquareFreeNumber} order).
\end{itemize}

\section*{Properties}

\PMlinkname{Subgroups}{Subgroup} and \PMlinkname{quotients}{QuotientGroup} of metacyclic groups are also metacyclic.

Metacyclic groups are obviously supersolvable, with Hirsch length at most $2$.
%%%%%
%%%%%
\end{document}
