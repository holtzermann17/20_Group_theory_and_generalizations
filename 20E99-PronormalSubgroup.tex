\documentclass[12pt]{article}
\usepackage{pmmeta}
\pmcanonicalname{PronormalSubgroup}
\pmcreated{2013-03-22 16:28:13}
\pmmodified{2013-03-22 16:28:13}
\pmowner{yark}{2760}
\pmmodifier{yark}{2760}
\pmtitle{pronormal subgroup}
\pmrecord{6}{38634}
\pmprivacy{1}
\pmauthor{yark}{2760}
\pmtype{Definition}
\pmcomment{trigger rebuild}
\pmclassification{msc}{20E99}
\pmdefines{pronormal}
\pmdefines{pronormality}

\def\genby#1{{\left\langle #1\right\rangle}}

\begin{document}
A subgroup $H$ of a group $G$ is called a \emph{pronormal subgroup}
if for all $x\in G$ the subgroups $H$ and $xHx^{-1}$
are conjugate in $\genby{H,xHx^{-1}}$.

Some facts about pronormal subgroups:
\begin{itemize}
\item Normal subgroups are pronormal.
\item Maximal subgroups are pronormal.
\item Abnormal subgroups are pronormal.
\item Sylow subgroups of finite groups are pronormal.
\item The normalizer of a pronormal subgroup is abnormal.
\item A pronormal subgroup is normal if and only if it is subnormal.
\end{itemize}

%%%%%
%%%%%
\end{document}
