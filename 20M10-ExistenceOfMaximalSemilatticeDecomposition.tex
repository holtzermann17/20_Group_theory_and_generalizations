\documentclass[12pt]{article}
\usepackage{pmmeta}
\pmcanonicalname{ExistenceOfMaximalSemilatticeDecomposition}
\pmcreated{2013-03-22 13:07:12}
\pmmodified{2013-03-22 13:07:12}
\pmowner{mclase}{549}
\pmmodifier{mclase}{549}
\pmtitle{existence of maximal semilattice decomposition}
\pmrecord{5}{33553}
\pmprivacy{1}
\pmauthor{mclase}{549}
\pmtype{Result}
\pmcomment{trigger rebuild}
\pmclassification{msc}{20M10}
\pmdefines{minimal semilattice congruence}
\pmdefines{maximal semilattice decomposition}

% this is the default PlanetMath preamble.  as your knowledge
% of TeX increases, you will probably want to edit this, but
% it should be fine as is for beginners.

% almost certainly you want these
\usepackage{amssymb}
\usepackage{amsmath}
\usepackage{amsfonts}

% used for TeXing text within eps files
%\usepackage{psfrag}
% need this for including graphics (\includegraphics)
%\usepackage{graphicx}
% for neatly defining theorems and propositions
\usepackage{amsthm}
% making logically defined graphics
%%\usepackage{xypic}

% there are many more packages, add them here as you need them

% define commands here
\newtheorem*{proposition}{Proposition}
\begin{document}
Let $S$ be a semigroup.  A \emph{maximal semilattice decomposition} for $S$ is a surjective homomorphism $\phi \colon S \to \Gamma$ onto a semilattice $\Gamma$ with the property that any other semilattice decomposition factors through $\phi$.  So if $\phi' \colon S \to \Gamma'$ is any other semilattice decomposition of $S$, then there is a homomorphism $\Gamma \to \Gamma'$ such that the following diagram commutes:
$$\xymatrix{
S \ar[r]^\phi \ar[dr]_{\phi'} & \Gamma \ar@{-->}[d] \\ & \Gamma'
}$$

\begin{proposition}
Every semigroup has a maximal semilattice decomposition.
\end{proposition}

\begin{proof}
Recall that each semilattice decompostion determines a semilattice congruence.  If $\{ \rho_i \mid i \in I \}$ is the family of all semilattice congruences on $S$, then define $\rho = \bigcap_{i \in I} \rho_i$.  (Here, we consider the congruences as subsets of $S \times S$, and take their intersection as sets.)

It is easy to see that $\rho$ is also a semilattice congruence, which is contained in all other semilattice congruences.

Therefore each of the homomorphisms $S \to S/\rho_i$ factors through $S \to S/\rho$.
\end{proof}
%%%%%
%%%%%
\end{document}
