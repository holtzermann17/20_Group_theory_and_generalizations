\documentclass[12pt]{article}
\usepackage{pmmeta}
\pmcanonicalname{ExampleOfANonfullyInvariantSubgroup}
\pmcreated{2013-03-22 16:06:26}
\pmmodified{2013-03-22 16:06:26}
\pmowner{Algeboy}{12884}
\pmmodifier{Algeboy}{12884}
\pmtitle{example of a non-fully invariant subgroup}
\pmrecord{7}{38172}
\pmprivacy{1}
\pmauthor{Algeboy}{12884}
\pmtype{Example}
\pmcomment{trigger rebuild}
\pmclassification{msc}{20D99}

\endmetadata

\usepackage{latexsym}
\usepackage{amssymb}
\usepackage{amsmath}
\usepackage{amsfonts}
\usepackage{amsthm}

%%\usepackage{xypic}

%-----------------------------------------------------

%       Standard theoremlike environments.

%       Stolen directly from AMSLaTeX sample

%-----------------------------------------------------

%% \theoremstyle{plain} %% This is the default

\newtheorem{thm}{Theorem}

\newtheorem{coro}[thm]{Corollary}

\newtheorem{lem}[thm]{Lemma}

\newtheorem{lemma}[thm]{Lemma}

\newtheorem{prop}[thm]{Proposition}

\newtheorem{conjecture}[thm]{Conjecture}

\newtheorem{conj}[thm]{Conjecture}

\newtheorem{defn}[thm]{Definition}

\newtheorem{remark}[thm]{Remark}

\newtheorem{ex}[thm]{Example}



%\countstyle[equation]{thm}



%--------------------------------------------------

%       Item references.

%--------------------------------------------------


\newcommand{\exref}[1]{Example-\ref{#1}}

\newcommand{\thmref}[1]{Theorem-\ref{#1}}

\newcommand{\defref}[1]{Definition-\ref{#1}}

\newcommand{\eqnref}[1]{(\ref{#1})}

\newcommand{\secref}[1]{Section-\ref{#1}}

\newcommand{\lemref}[1]{Lemma-\ref{#1}}

\newcommand{\propref}[1]{Prop\-o\-si\-tion-\ref{#1}}

\newcommand{\corref}[1]{Cor\-ol\-lary-\ref{#1}}

\newcommand{\figref}[1]{Fig\-ure-\ref{#1}}

\newcommand{\conjref}[1]{Conjecture-\ref{#1}}


% Normal subgroup or equal.

\providecommand{\normaleq}{\unlhd}

% Normal subgroup.

\providecommand{\normal}{\lhd}

\providecommand{\rnormal}{\rhd}
% Divides, does not divide.

\providecommand{\divides}{\mid}

\providecommand{\ndivides}{\nmid}


\providecommand{\union}{\cup}

\providecommand{\bigunion}{\bigcup}

\providecommand{\intersect}{\cap}

\providecommand{\bigintersect}{\bigcap}










\begin{document}
Every fully invariant subgroup is characteristic, but some characteristic subgroups need not be fully invariant.  For example, the center of a group is characteristic but not always fully invariant.  We pursue a single example.

Recall the dihedral group of order $2n$, denoted $D_{2n}$, can be considered as the symmetries of a regular $n$-gon.  If we consider a regular hexagon, so $n=6$, and label the vertices counterclockwise from 1 to 6 we can then encode each symmetry as a permutation on 6 points.  So a rotation by $\pi/3$ can be encoded as the permutation $\rho = (123456)$ and the reflection fixing the axis through the vertices 1 and 4 can be encoded as $\phi = (26)(35)$.  Indeed these two permutations generate a permutation group isomorphic to $D_{12}$.

The center of a dihedral group of order $2n$ is trivial if $n$ is odd, and of order 2 if $n>2$ is even (if $n=2$ it is the entire group $D_4\cong \mathbb{Z}_2\oplus\mathbb{Z}_2$, see the remark below).  Specifically, if $\rho$ is a rotation of order $n$, and $n=2m$, then $\langle \rho^m\rangle$ is the center of $D_{2n}$.  (Note this is the only rotation or order 2, and in particular it is always a rotation by $\pi$.)
So when $n=6$, the center is $\langle (14)(25)(36)\rangle$.

Now fix $n=6$ and note the following assignment of generators determines an endomorphism $f:D_{12}\rightarrow D_{12}$:
\[(123456)\mapsto (26)(35),\quad (26)(35)\mapsto (14)(25)(36).\]
Note that image $K:=\langle (26)(35),(14)(25)(36)\rangle\cong \mathbb{Z}_2\oplus \mathbb{Z}_2$, as $(14)(25)(36)$ is central in $D_{12}$ and the generators of
$K$ are distinct elements of order $2$.  [This can be proved with the relations of the dihedral group.]

\begin{remark}
Geometrically we note that the kernel of the homomorphism is $\langle \rho^2\rangle$ -- the group of rotations of order 3.  So if we quotient by the kernel we are identifying the three inscribed (non-square) rectangles of the hexagon (1245, 2356 and 3461).  The symmetry group of a non-square rectangle is none other than $\mathbb{Z}_2\oplus\mathbb{Z}_2$, sometimes called $D_{4}$.
\end{remark}

Now the center is mapped via $f$ to the subgroup $\langle (26)(35)\rangle$ so 
$f(Z(D_{12}))$ is not contained in $Z(D_{12})$ proving $Z(D_{12})$ is not fully-invariant.

Of course the example applies without serious modification to the dihedral groups on $2m$-gons, where $m>1$ is odd.  Here a generally offending endomorphism may be described with a composition of maps (the first leaves the center invariant, the second swaps the basis of the image of the first thus moving the image of the center):
\[\rho\mapsto \rho^m\mapsto \phi,\qquad \phi\mapsto \phi\mapsto \rho^m.\]
As $m$ is odd and the center, $\langle \rho^m\rangle$, has order 2, it follows $\langle\rho^m\rangle$ maps to $\langle \rho^m\rangle$ under the first map, and then can be interchanged with a reflection to violate the condition of full invariance.  If $m$ is even then the center lies in the kernel of the first map so no such trick can be played.
%%%%%
%%%%%
\end{document}
