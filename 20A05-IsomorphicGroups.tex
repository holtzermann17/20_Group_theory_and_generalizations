\documentclass[12pt]{article}
\usepackage{pmmeta}
\pmcanonicalname{IsomorphicGroups}
\pmcreated{2013-03-22 14:01:58}
\pmmodified{2013-03-22 14:01:58}
\pmowner{alozano}{2414}
\pmmodifier{alozano}{2414}
\pmtitle{isomorphic groups}
\pmrecord{10}{35127}
\pmprivacy{1}
\pmauthor{alozano}{2414}
\pmtype{Definition}
\pmcomment{trigger rebuild}
\pmclassification{msc}{20A05}
\pmdefines{isomorphic}
\pmdefines{abstractly identical}

\usepackage{amssymb}
\usepackage{amsmath}
\usepackage{amsfonts}
\begin{document}
\PMlinkescapeword{isomorphic}
Two groups $(X_1,\,*_1)$ and $(X_2,\,*_2)$ are said to be {\em isomorphic} if there is a group isomorphism $\psi\colon X_1\to X_2$.

Next we name a few necessary conditions for two groups $X_1,\,X_2$ to be isomorphic (with isomorphism $\psi$ as above). 

\begin{enumerate}
\item {\it If two groups are isomorphic, then they have the same cardinality}. Indeed, an isomorphism is in particular a bijection of sets.

\item {\it If the group $X_1$ has an element $g$ of order $n$, then the group $X_2$ must have an element of the same order}. If there is an isomorphism $\psi$ then $\psi(g)\in X_2$ and $(\psi(g))^n=\psi(g^n)=\psi(e_1)=e_2$ where $e_i$ is the identity elements of $X_i$. Moreover, if $(\psi(g))^m=e_2$ then $\psi(g^m)=e_2$ and by the injectivity of $\psi$ we must have $g^m=e_1$ so $n$ divides $m$. Therefore the order of $\psi(g)$ is $n$.

\item {\it If one group is cyclic, the other one must be cyclic too}. Suppose $X_1$ is cyclic generated by an element $g$. Then it is easy to see that $X_2$ is generated by the element $\psi(g)$. Also {\it if $X_1$ is finitely generated, then $X_2$ is finitely generated as well}.

\item {\it If one group is abelian, the other one must be abelian as well}. Indeed, suppose $X_2$ is abelian. Then 
$$\psi(g*_1 h)=\psi(g)*_2 \psi(h)=\psi(h)*_2 \psi(g) =\psi(h*_1 g)$$
and using the injectivity of $\psi$ we conclude \,$g*_1 h=h*_1 g$. 
\end{enumerate}

\textbf{Note.}\, Isomorphic groups are sometimes said to be {\em abstractly identical}, because their ``abstract'' \PMlinkescapetext{structures} are completely similar --- one may think that their elements are the same but have only different names.
%%%%%
%%%%%
\end{document}
