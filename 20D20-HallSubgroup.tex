\documentclass[12pt]{article}
\usepackage{pmmeta}
\pmcanonicalname{HallSubgroup}
\pmcreated{2013-03-22 14:02:02}
\pmmodified{2013-03-22 14:02:02}
\pmowner{Algeboy}{12884}
\pmmodifier{Algeboy}{12884}
\pmtitle{Hall subgroup}
\pmrecord{22}{35135}
\pmprivacy{1}
\pmauthor{Algeboy}{12884}
\pmtype{Definition}
\pmcomment{trigger rebuild}
\pmclassification{msc}{20D20}
\pmrelated{VeeCuhininsTheorem}
\pmrelated{SylowTheorems}
\pmdefines{Hall's theorem}
\pmdefines{Hall $\pi$-subgroup}

\endmetadata

\usepackage{amssymb}
\usepackage{amsfonts}
\usepackage[mathscr]{euscript}
\usepackage[T1]{fontenc}
% graphics
%\usepackage{pstricks}
% math
\usepackage{amsmath}
\usepackage{amsopn}
\usepackage{amstext}
\usepackage{amsthm}

\newtheorem{thm}{Theorem}

\newtheorem{ex}[thm]{Example}
\begin{document}
Let $G$ be a finite group.  A subgroup $H$ of $G$ is said to 
be a \emph{Hall subgroup} if $$\gcd (|H|,|G/H|)=1.$$  In other 
words, $H$ is a Hall subgroup if the order of $H$ and its index 
in $G$ are coprime.  These subgroups are 
name after Philip Hall who used them to characterize solvable groups.

Hall subgroups are a generalization of Sylow subgroups.  Indeed, 
every Sylow subgroup is a Hall subgroup.  According to Sylow's 
theorem, this means that any group of order $p^k m$, $\gcd(p,m)=1$, 
has a Hall subgroup (of order $p^k$).

A common notation used with Hall subgroups is to use the notion of
\PMlinkname{$\pi$-groups}{PiGroupsAndPiGroups}.  Here $\pi$ is a set
of primes and a Hall $\pi$-subgroup of a group is a subgroup which is
also a $\pi$-group, and maximal with this property.

\begin{thm}[Hall (1928)] 
A finite group $G$ is solvable iff $G$ has a Hall $\pi$-subgroup for
any set of primes $\pi$.
\end{thm}

The sets of primes $\pi$ in Hall's theorem can be restricted to
the subsets of primes which divide $|G|$.
However, this result fails for non-solvable groups.  

\begin{ex}
The group $A_5$ has no Hall $\{2,5\}$-subgroup.  That is,
$A_5$ has no subgroup of order $20$.
\end{ex}
\begin{proof}
Suppose that $A_5$ has a Hall $\{2,5\}$-subgroup $H$.  
As $|A_5|=60$, it follows that $|H|=20$.  Thus, there
are three cosets of $H$.  Since a group always acts
on the cosets of a subgroup, it follows that $A_5$ acts
on the three member set $C$ of cosets of $H$.  This induces
a non-trivial homomorphism from $A_5$ to $S_C\cong S_3$ 
(here, $S_C$ is the symmetric group on $C$, see 
\PMlinkname{this}{GroupActionsAndHomomorphisms} for more detail).
Since $A_5$ is simple, this homomorphism must be one-to-one, 
implying that its image must have order at most $6$, an 
impossibility.
\end{proof}

This example can also be proved by direct inspection of the subgroups of $A_5$.
In any case, $A_5$ is non-abelian simple and therefore it is not a
solvable group.  Thus, Hall's theorem does not apply to $A_5$.



%%%%%
%%%%%
\end{document}
