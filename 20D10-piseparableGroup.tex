\documentclass[12pt]{article}
\usepackage{pmmeta}
\pmcanonicalname{piseparableGroup}
\pmcreated{2013-03-22 13:17:48}
\pmmodified{2013-03-22 13:17:48}
\pmowner{yark}{2760}
\pmmodifier{yark}{2760}
\pmtitle{$\pi$-separable group}
\pmrecord{7}{33796}
\pmprivacy{1}
\pmauthor{yark}{2760}
\pmtype{Definition}
\pmcomment{trigger rebuild}
\pmclassification{msc}{20D10}
\pmdefines{$\pi$-separable}
\pmdefines{$p$-separable}
\pmdefines{$\pi$-separability}
\pmdefines{$p$-separability}
\pmdefines{$p$-separable group}

\endmetadata

\usepackage{amssymb}
\usepackage{amsmath}
\usepackage{amsfonts}

\begin{document}
Let $\pi$ be a set of prime numbers.
A finite group $G$ is called \emph{$\pi$-separable}
if there exists a composition series
\[
  \{1\}=G_0\lhd\cdots\lhd G_n=G
\]
such that each $G_{i+1}/G_i$ is either
a \PMlinkname{$\pi$-group}{PiGroupsAndPiGroups}
or a \PMlinkname{$\pi'$-group}{PiGroupsAndPiGroups}.

A $\{p\}$-separable group, where $p$ is a prime number,
is usually called a \emph{$p$-separable} group.

$\pi$-separability can be thought of as
a generalization of solvability for finite groups;
a finite group is $\pi$-separable for all sets of primes
if and only it is solvable.
%%%%%
%%%%%
\end{document}
