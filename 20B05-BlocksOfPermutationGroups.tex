\documentclass[12pt]{article}
\usepackage{pmmeta}
\pmcanonicalname{BlocksOfPermutationGroups}
\pmcreated{2013-03-22 17:19:05}
\pmmodified{2013-03-22 17:19:05}
\pmowner{rm50}{10146}
\pmmodifier{rm50}{10146}
\pmtitle{blocks of permutation groups}
\pmrecord{15}{39668}
\pmprivacy{1}
\pmauthor{rm50}{10146}
\pmtype{Topic}
\pmcomment{trigger rebuild}
\pmclassification{msc}{20B05}
\pmdefines{trivial block}
\pmdefines{block}
\pmdefines{block system}
\pmdefines{conjugate block}

\endmetadata

% this is the default PlanetMath preamble.  as your knowledge
% of TeX increases, you will probably want to edit this, but
% it should be fine as is for beginners.

% almost certainly you want these
\usepackage{amssymb}
\usepackage{amsmath}
\usepackage{amsfonts}

% used for TeXing text within eps files
%\usepackage{psfrag}
% need this for including graphics (\includegraphics)
%\usepackage{graphicx}
% for neatly defining theorems and propositions
\usepackage{amsthm}
% making logically defined graphics
%%%\usepackage{xypic}

% there are many more packages, add them here as you need them

% define commands here
\newtheorem*{thm}{Theorem}
\newtheorem*{cor}{Corollary}
\theoremstyle{definition}
\newtheorem*{defn}{Definition}

\begin{document}
Throughout this article, $A$ is a set and $G$ is a permutation group on $A$.

A \emph{block} is a subset $B$ of $A$ such that for each $\sigma\in G$, either $\sigma\cdot B=B$ or $(\sigma\cdot B)\cap B=\emptyset$, where $\sigma\cdot B=\{\sigma(b)\ \mid\ b\in B\}$. In other words, if $\sigma\cdot B$ intersects $B$, then $\sigma\cdot B = B$.

Note that for any such permutation group, each of $\emptyset$, $A$, and every element of $A$ is a block. These are called \emph{trivial blocks}.

It is obvious that if $H\subset G$ are permutation groups on $A$, then any block of $G$ is also a block of $H$.

Blocks are closed under finite intersection:
\begin{thm} If $B_1, B_2\subset A$ are blocks of $G$, then $B=B_1\cap B_2$ is a block of $G$.
\end{thm}
\begin{proof} Choose $\sigma\in G$.
Note that $\sigma\cdot(B_1\cap B_2)=(\sigma\cdot B_1)\cap (\sigma\cdot B_2)$. Thus if $(\sigma\cdot B)\cap B\neq\emptyset$, then
\[(\sigma\cdot B)\cap B = (\sigma\cdot (B_1\cap B_2))\cap(B_1\cap B_2)=(\sigma\cdot B_1 \cap B_1)\cap (\sigma\cdot B_2\cap B_2)\]
is nonempty, and thus $\sigma\cdot B_i \cap B_i\neq\emptyset$ for $i=1,2$. But $B_1$ and $B_2$ are blocks, so that $\sigma\cdot B_i=B_i$ for $i=1,2$. Thus 
\[\sigma\cdot B  = \sigma\cdot (B_1\cap B_2) = (\sigma\cdot B_1)\cap (\sigma\cdot B_2)=B_1\cap B_2=B\]
and $B$ is a block.
\end{proof}

We show, as a corollary to the following theorem, that blocks themselves are permuted by the action of the group.
\begin{thm} If $H\subset G$ are permutation groups on $A$, $B\subset A$ is a block of $H$, and $
\sigma\in G$, then $\sigma\cdot B$ is a block of $\sigma H\sigma^{-1}$.
\end{thm}
\begin{proof}
Choose $\tau\in H$ and assume that
\[((\sigma \tau\sigma^{-1})\sigma\cdot B)\cap \sigma\cdot B\neq\emptyset\]
Then, applying $\sigma^{-1}$ to this equation, we see that
\[(\tau\cdot B)\cap B\neq\emptyset\]
But $B$ is a block of $H$, so $\tau\cdot B=B$. Multiplying by $\sigma$, we see that
\[\sigma\cdot(\tau\cdot B)=\sigma\cdot B\]
and thus
\[(\sigma\tau\sigma^{-1})\sigma\cdot B=\sigma\cdot B\]
and the result follows.
\end{proof}

\begin{cor} If $B$ is a block of $G$, $\sigma\in G$, then $\sigma\cdot B$ is also a block of $G$.
\end{cor}
\begin{proof}
Set $G=H$ in the above theorem.
\end{proof}

\begin{defn} If $B$ is a block of $G$, $\sigma\in G$, then $B$ and $\sigma\cdot B$ are \emph{conjugate blocks}. The set of all blocks conjugate to a given block is a \emph{block system}.
\end{defn}

It is clear from the fact that $B$ is a block that conjugate blocks are either equal or disjoint, so the action of $G$ permutes the blocks of $G$. Then if $G$ acts transitively on $A$, the union of any nontrivial block and its conjugates is $A$.

\begin{thm} If $G$ is finite and $G$ acts transitively on $A$, then the size of a nonempty block divides the order of $G$.
\end{thm}
\begin{proof} Since $G$ acts transitively, $A$ is finite as well. All conjugates of the block have the same size; since the action is transitive, the union of the block and all its conjugates is $A$. Thus the size of the block divides the size of $A$. Finally, by the orbit-stabilizer theorem, the order of $G$ is divisible by the size of $A$.
\end{proof}
%%%%%
%%%%%
\end{document}
