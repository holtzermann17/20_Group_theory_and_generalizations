\documentclass[12pt]{article}
\usepackage{pmmeta}
\pmcanonicalname{CorrespondenceOfNormalSubgroupsAndGroupCongruences}
\pmcreated{2013-03-22 15:32:52}
\pmmodified{2013-03-22 15:32:52}
\pmowner{Dr_Absentius}{537}
\pmmodifier{Dr_Absentius}{537}
\pmtitle{correspondence of normal subgroups and group congruences}
\pmrecord{7}{37445}
\pmprivacy{1}
\pmauthor{Dr_Absentius}{537}
\pmtype{Theorem}
\pmcomment{trigger rebuild}
\pmclassification{msc}{20-00}
\pmdefines{group congruence}

\endmetadata

%\documentclass{amsart}
\usepackage{amsmath}
%\usepackage[all,poly,knot,dvips]{xy}
%\usepackage{pstricks,pst-poly,pst-node}
\usepackage{amssymb}
\usepackage{amsthm}
\usepackage{eucal,latexsym}

% THEOREM Environments --------------------------------------------------

\newtheorem{thm}{Theorem}
 \newtheorem*{mainthm}{Main~Theorem}
 \newtheorem{cor}[thm]{Corollary}
 \newtheorem{lem}[thm]{Lemma}
 \newtheorem{prop}[thm]{Proposition}
 \newtheorem{claim}[thm]{Claim}
 \theoremstyle{definition}
 \newtheorem{defn}[thm]{Definition}
 \theoremstyle{remark}
 \newtheorem{rem}[thm]{Remark}
 \numberwithin{equation}{subsection}


%---------------------  Greek letters, etc ------------------------- 

\newcommand{\CA}{\mathcal{A}}
\newcommand{\CC}{\mathcal{C}}
\newcommand{\CM}{\mathcal{M}}
\newcommand{\CP}{\mathcal{P}}
\newcommand{\CS}{\mathcal{S}}
\newcommand{\BC}{\mathbb{C}}
\newcommand{\BN}{\mathbb{N}}
\newcommand{\BR}{\mathbb{R}}
\newcommand{\BZ}{\mathbb{Z}}
\newcommand{\FF}{\mathfrak{F}}
\newcommand{\FL}{\mathfrak{L}}
\newcommand{\FM}{\mathfrak{M}}
\newcommand{\Ga}{\alpha}
\newcommand{\Gb}{\beta}
\newcommand{\Gg}{\gamma}
\newcommand{\GG}{\Gamma}
\newcommand{\Gd}{\delta}
\newcommand{\GD}{\Delta}
\newcommand{\Ge}{\varepsilon}
\newcommand{\Gz}{\zeta}
\newcommand{\Gh}{\eta}
\newcommand{\Gq}{\theta}
\newcommand{\GQ}{\Theta}
\newcommand{\Gi}{\iota}
\newcommand{\Gk}{\kappa}
\newcommand{\Gl}{\lambda}
\newcommand{\GL}{\Lamda}
\newcommand{\Gm}{\mu}
\newcommand{\Gn}{\nu}
\newcommand{\Gx}{\xi}
\newcommand{\GX}{\Xi}
\newcommand{\Gp}{\pi}
\newcommand{\GP}{\Pi}
\newcommand{\Gr}{\rho}
\newcommand{\Gs}{\sigma}
\newcommand{\GS}{\Sigma}
\newcommand{\Gt}{\tau}
\newcommand{\Gu}{\upsilon}
\newcommand{\GU}{\Upsilon}
\newcommand{\Gf}{\varphi}
\newcommand{\GF}{\Phi}
\newcommand{\Gc}{\chi}
\newcommand{\Gy}{\psi}
\newcommand{\GY}{\Psi}
\newcommand{\Gw}{\omega}
\newcommand{\GW}{\Omega}
\newcommand{\Gee}{\epsilon}
\newcommand{\Gpp}{\varpi}
\newcommand{\Grr}{\varrho}
\newcommand{\Gff}{\phi}
\newcommand{\Gss}{\varsigma}

\def\co{\colon\thinspace}
\begin{document}
We start with a definition.  

\begin{defn}
  Let $G$ be a group.  An equivalence relation $\sim$ on $G$ is
  called a \emph{group congruence} if it is compatible with the 
  group structure, ie. when the following holds
  \begin{itemize}
  \item $\forall a,b,a',b'\in G,\quad (a\sim a'\,\,\, \text{and}
    \,\,\, b\sim b')  \Rightarrow ab \sim a'b'$
  \item $\forall a,b\in G,\quad a\sim b \Rightarrow a^{-1}\sim b^{-1}\,. $
  \end{itemize}
\end{defn}

So a group congruence is a \PMlinkid{semigroup congruence}{3403}
that additionally preserves  the unary operation of taking inverse. 

It turns out that group congruences correspond to normal subgroups:
\begin{thm}
  An equivalence relation $\sim$ is a group congruence 
  if and only if there is a normal subgroup such that 
$$\forall a,b \in G, \quad a\sim b \Longleftrightarrow ab^{-1}\in H\,. $$
\end{thm}


\begin{proof}

  Let $H$ be a \emph{normal} subgroup of $G$ and let $\sim_H$ be the
  equivalence relation $H$ defines in $G$.  To see that this
  equivalence relation is compatible with the group operation note
  that if $a'\sim_H a$ and $b'\sim_H b$ then there are elements $h_1$
  and $h_2$ of $H$ such that $a' = ah_1$ and $b' = b
  h_2$. Furthermore since $H$ is normal in $G$ there is an element
  $h_3\in H$ such that $h_1b = bh_3$.  Then we have
\begin{align*}
  a'b'&=ah_1bh_2 \\
      &=abh_3h_2 
\end{align*}
which gives that $a'b'\sim ab$.

  To prove the converse,  assume that $\sim$ is an
  equivalence relation compatible with the group operation and let
  $H$ be the equivalence class of the identity $e$.  We will prove
  that $\sim \,= \,\sim_H$. We first prove
  that $H$ is a normal subgroup of $G$.  Indeed if $a\sim e$ and $b
  \sim e$ then by the compatibility we have that $ab^{1} \sim ee^{-1}$, that
  is $ab^{-1}\sim e$; so that $H$ is a subgroup of $G$.  Now if
  $g\in G$ and $h\in H$ we have 
  \begin{align*}
    h\sim e &\Rightarrow ghg^{-1} \sim geg^{-1}\\
            &\Rightarrow ghg^{-1} \sim e\\            
            &\Rightarrow ghg^{-1} \in H\,.
  \end{align*}
 Therefore $H$ is a normal subgroup of $G$.  Now consider two
 elements $a$ and $b$ of $G$.  To finish the proof observe that for
 $a,b\in G$ we have
 \begin{align*}
   a\sim_H b   &\Rightarrow ab^{-1}\in H\\ 
               &\Rightarrow ab^{-1} \sim e\\
               &\Rightarrow (ab^{-1})b \sim eb\\ 
               &\Rightarrow a \sim b 
 \end{align*}
and
 \begin{align*}
   a\sim b &\Rightarrow ab^{-1} \sim bb^{-1}\\
               &\Rightarrow ab^{-1} \sim e\\
               &\Rightarrow a\sim_H b\,.
 \end{align*}
\end{proof}
%%%%%
%%%%%
\end{document}
