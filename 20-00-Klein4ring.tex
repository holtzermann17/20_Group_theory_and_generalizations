\documentclass[12pt]{article}
\usepackage{pmmeta}
\pmcanonicalname{Klein4ring}
\pmcreated{2015-06-04 16:11:18}
\pmmodified{2015-06-04 16:11:18}
\pmowner{pahio}{2872}
\pmmodifier{pahio}{2872}
\pmtitle{Klein 4-ring}
\pmrecord{16}{36720}
\pmprivacy{1}
\pmauthor{pahio}{2872}
\pmtype{Definition}
\pmcomment{trigger rebuild}
\pmclassification{msc}{20-00}
\pmclassification{msc}{16B99}
\pmsynonym{Klein's four-ring}{Klein4ring}
\pmsynonym{Klein four-ring}{Klein4ring}
%\pmkeywords{non-commutative}
\pmrelated{Klein4Group}
\pmrelated{InversesInRings}
\pmrelated{NonCommutativeRingsOfOrderFour}
\pmrelated{GroupsInField}
\pmrelated{Subcommutative}

\endmetadata

% this is the default PlanetMath preamble.  as your knowledge
% of TeX increases, you will probably want to edit this, but
% it should be fine as is for beginners.

% almost certainly you want these
\usepackage{amssymb}
\usepackage{amsmath}
\usepackage{amsfonts}

% used for TeXing text within eps files
%\usepackage{psfrag}
% need this for including graphics (\includegraphics)
%\usepackage{graphicx}
% for neatly defining theorems and propositions
%\usepackage{amsthm}
% making logically defined graphics
%%%\usepackage{xypic}

% there are many more packages, add them here as you need them

% define commands here
\begin{document}
One of the two smallest non-commutative rings is the 
{\em Klein 4-ring}\, $(R,+,\cdot)$\, where\, $(R,+)$\, is the Klein 4-group \,$\{0,\,a,\,b,\,c\}$\, with $0$ the neutral element and the binary operation ``$\cdot$'' given by the table

$$\begin{array}{c|cccc}
\cdot & 0 & a & b & c \\
\hline
\;  0 & 0 & 0 & 0 & 0 \\
\;  a & 0 & a & 0 & a \\
\;  b & 0 & b & 0 & b \\
\;  c & 0 & c & 0 & c
  \end{array}$$

Note that this ring has two different right unities $a$ and $c$.\\

The Klein 4-ring has the subrings \,$\{0,\,a\}$,\, $\{0,\,b\}$\, and\, $\{0,\,c\}$\, and the two-sided ideal\, $\{0,\,b\}$.
%%%%%
%%%%%
\end{document}
