\documentclass[12pt]{article}
\usepackage{pmmeta}
\pmcanonicalname{QuasicyclicGroup}
\pmcreated{2013-03-22 15:35:22}
\pmmodified{2013-03-22 15:35:22}
\pmowner{yark}{2760}
\pmmodifier{yark}{2760}
\pmtitle{quasicyclic group}
\pmrecord{19}{37500}
\pmprivacy{1}
\pmauthor{yark}{2760}
\pmtype{Definition}
\pmcomment{trigger rebuild}
\pmclassification{msc}{20F50}
\pmclassification{msc}{20K10}
\pmsynonym{quasi-cyclic group}{QuasicyclicGroup}
\pmsynonym{Pr\"ufer group}{QuasicyclicGroup}
\pmdefines{quasicyclic}
\pmdefines{quasi-cyclic}
\pmdefines{Pr\"ufer p-group}

\endmetadata

\usepackage{amssymb}
\usepackage{amsmath}
\usepackage{amsfonts}

\def\N{\mathbb{N}}
\def\Q{\mathbb{Q}}
\def\Z{\mathbb{Z}}
\begin{document}
\PMlinkescapeword{component}
\PMlinkescapeword{equivalent}
\PMlinkescapeword{group}
\PMlinkescapeword{groups}
\PMlinkescapeword{maximal}
\PMlinkescapeword{simple}
\PMlinkescapeword{structure}
\PMlinkescapeword{subgroup}
\PMlinkescapeword{subgroups}

Let $p$ be a prime number.
The \emph{$p$-quasicyclic group} (or \emph{Pr\"ufer $p$-group}, or \emph{$p^\infty$ group}) is the $p$-primary component of $\Q/\Z$,
that is, the unique maximal \PMlinkname{$p$-subgroup}{PGroup4} of $\Q/\Z$.
Any \PMlinkname{group}{Group} isomorphic to this will also be called a $p$-quasicyclic group.

The $p$-quasicyclic group will be denoted by $\Z(p^\infty)$.
Other notations in use include $\Z[p^\infty]$, $\Z/p^\infty\Z$,
$\Z_{p^\infty}$ and $C_{p^\infty}$.

$\Z(p^\infty)$ may also be defined in a number of other (equivalent) ways
(again, up to isomorphism):
\begin{itemize}
\item $\Z(p^\infty)$ is
the group of all $p^n$-th complex roots of $1$, for $n\in\N$.
\item $\Z(p^\infty)$ is
the injective hull of $\Z/p\Z$ (viewing abelian groups as $\Z$-\PMlinkname{modules}{Module}).
\item $\Z(p^\infty)$ is the direct limit of the groups $\Z/p^n\Z$.
\end{itemize}

A \emph{quasicyclic group} (or \emph{Pr\"ufer group}) is
a group that is $p$-quasicyclic for some prime $p$.

The \PMlinkname{subgroup}{Subgroup} structure of $\Z(p^\infty)$ is particularly simple:
all proper subgroups are finite and cyclic,
and there is exactly one of order $p^n$ for each non-negative integer $n$.
In particular,
this means that the subgroups are linearly ordered by inclusion,
and all subgroups are fully invariant.
The quasicyclic groups are
the only infinite groups with a linearly ordered subgroup lattice.
They are also
the only infinite solvable groups whose proper subgroups are all finite.

Quasicyclic groups are locally cyclic, \PMlinkname{divisible}{DivisibleGroup} and co-Hopfian.

Every infinite locally cyclic $p$-group is isomorphic to $\Z(p^\infty)$.
%%%%%
%%%%%
\end{document}
