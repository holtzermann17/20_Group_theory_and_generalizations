\documentclass[12pt]{article}
\usepackage{pmmeta}
\pmcanonicalname{ZeroTimesAnElementIsZeroInARing}
\pmcreated{2013-03-22 14:13:57}
\pmmodified{2013-03-22 14:13:57}
\pmowner{alozano}{2414}
\pmmodifier{alozano}{2414}
\pmtitle{zero times an element is zero in a ring}
\pmrecord{8}{35673}
\pmprivacy{1}
\pmauthor{alozano}{2414}
\pmtype{Theorem}
\pmcomment{trigger rebuild}
\pmclassification{msc}{20-00}
\pmclassification{msc}{16-00}
\pmclassification{msc}{13-00}
\pmsynonym{$0\cdot a=0$}{ZeroTimesAnElementIsZeroInARing}
\pmrelated{1cdotAA}
\pmrelated{AbsorbingElement}

\endmetadata

% this is the default PlanetMath preamble.  as your knowledge
% of TeX increases, you will probably want to edit this, but
% it should be fine as is for beginners.

% almost certainly you want these
\usepackage{amssymb}
\usepackage{amsmath}
\usepackage{amsthm}
\usepackage{amsfonts}

% used for TeXing text within eps files
%\usepackage{psfrag}
% need this for including graphics (\includegraphics)
%\usepackage{graphicx}
% for neatly defining theorems and propositions
%\usepackage{amsthm}
% making logically defined graphics
%%%\usepackage{xypic}

% there are many more packages, add them here as you need them

% define commands here

\newtheorem{thm}{Theorem}
\newtheorem{defn}{Definition}
\newtheorem{prop}{Proposition}
\newtheorem{lemma}{Lemma}
\newtheorem{cor}{Corollary}

% Some sets
\newcommand{\Nats}{\mathbb{N}}
\newcommand{\Ints}{\mathbb{Z}}
\newcommand{\Reals}{\mathbb{R}}
\newcommand{\Complex}{\mathbb{C}}
\newcommand{\Rats}{\mathbb{Q}}
\begin{document}
\begin{lemma}
Let $R$ be a ring with zero element $0$ (i.e. $0$ is the additive identity of $R$). Then for any element $a\in R$ we have $0\cdot a = a\cdot 0 = 0$.
\end{lemma}
\begin{proof}
\begin{eqnarray*}
0\cdot a &=& (0+0)\cdot a , \quad \text{ by definition of zero}\\
&=& 0\cdot a + 0\cdot a, \quad \text{ by the distributive law}\\
\end{eqnarray*}
Thus $0\cdot a=0\cdot a + 0\cdot a$. Let $b$ be the additive inverse of $0\cdot a \in R$. Hence:
\begin{eqnarray*}
b+0\cdot a=b+(0\cdot a + 0\cdot a)\\
(b+0\cdot a)=(b+0\cdot a) + 0\cdot a\\
0=0 + 0\cdot a\\
0=0\cdot a
\end{eqnarray*}
as claimed. The proof of $a\cdot 0=0$ is done analogously.
\end{proof}
%%%%%
%%%%%
\end{document}
