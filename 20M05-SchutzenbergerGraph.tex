\documentclass[12pt]{article}
\usepackage{pmmeta}
\pmcanonicalname{SchutzenbergerGraph}
\pmcreated{2013-03-22 16:10:50}
\pmmodified{2013-03-22 16:10:50}
\pmowner{Mazzu}{14365}
\pmmodifier{Mazzu}{14365}
\pmtitle{Sch\"{u}tzenberger graph}
\pmrecord{34}{38268}
\pmprivacy{1}
\pmauthor{Mazzu}{14365}
\pmtype{Definition}
\pmcomment{trigger rebuild}
\pmclassification{msc}{20M05}
\pmclassification{msc}{20M18}
%\pmkeywords{Inverse Semigroups}
%\pmkeywords{Word Problem}
\pmrelated{MunnTree}
\pmdefines{Sch\"{u}tzenberger graph}
\pmdefines{left Sch\"{u}tzenberger graph}
\pmdefines{right Sch\"{u}tzenberger graph}

% this is the default PlanetMath preamble.  as your knowledge
% of TeX increases, you will probably want to edit this, but
% it should be fine as is for beginners.

% almost certainly you want these
\usepackage{amssymb}
\usepackage{amsmath}
\usepackage{amsfonts}

% used for TeXing text within eps files
%\usepackage{psfrag}
% need this for including graphics (\includegraphics)
%\usepackage{graphicx}
% for neatly defining theorems and propositions
%\usepackage{amsthm}
% making logically defined graphics
%%%\usepackage{xypic}

% there are many more packages, add them here as you need them

% define commands here

\begin{document}
\PMlinkescapeword{inverse}
\PMlinkescapeword{graph}
\PMlinkescapeword{right}
\PMlinkescapeword{theory}
\PMlinkescapeword{argument}




\newcommand{\V}{\mathrm{V}}
\newcommand{\E}{\mathrm{E}}
\newcommand{\schG}{\mathcal{S}\Gamma}

\newcommand{\e}{\mathrm{e}}
\newcommand{\co}{\mathrm{c}}

\newcommand{\cbra}[1]{\left( #1 \right)}
\newcommand{\qbra}[1]{\left[ #1 \right]}
\newcommand{\gbra}[1]{\left\{ #1 \right\}}
\newcommand{\abra}[1]{\left\langle #1 \right\rangle}

\newcommand{\mipres}[2]{\mathrm{Inv}^1\abra{#1 | #2}}
\newcommand{\sipres}[2]{\mathrm{Inv}\abra{#1 | #2}}

\newcommand{\double}[1]{\cbra{#1\amalg #1^{-1}}}
\newcommand{\doubles}[1]{\cbra{#1\amalg #1^{-1}}^\ast}
\newcommand{\doublep}[1]{\cbra{#1\amalg #1^{-1}}^+}

Let $(X;T)$ be a presentation for the inverse monoid $\mipres{X}{T}$ [resp.  inverse semigroup $\sipres{X}{T}$]. In what   follows, the argument for inverse semigroups and inverse monoids is exactly the same, so we concentrate on the last one.

Given $m\in \mipres{X}{T}$, let $[m]_{\mathcal{R}}$ be the equivalence class of $m$ with respect to the Right Green relation $\mathcal{R}$. The \emph{Right Sch\"{u}tzenberger graph} of $[m]_{\mathcal{R}}$ with respect to the presentation $(X;T)$ is defined as the $X$-inverse word graph $\schG(X;T;m)$ with vertex and edge set respectively
$$\V(\schG(X;T;m))=\gbra{v\in \mipres{X}{T}\,|\,[v]_{\mathcal{R}}= [m]_{\mathcal{R}}},$$
$$\E(\schG(X;T;m))=\gbra{(v_1,x,v_2)\,|\,v_1,v_2\in \V(\schG(X;T;m)),\ x\in\double{X},\ v_2=v_1
\cdot [x]_\tau},$$
where $\tau=(T\cup\rho_X)^\co$, i.e. $\tau$ is the congruence generated by $T$ and the Wagner congruence $\rho_X$, and $[x]_\tau$ is the congruence class of the letter $x\in\double X$ with respect to the congruence $\tau$.

This is a good definition, in fact it can be easily shown that given $m,n\in\mipres{X}{T}$ with $[m]_{\mathcal R}=[n]_{\mathcal R}$ we have $\schG(X;T;m)=\schG(X;T;n)$.



Analogously we can define the \emph{Left Sch\"{u}tzenberger graph} using the Left  Green relation $\mathcal{L}$ instead of the Right Green relation $\mathcal{R}$, but this notion is not used in literature.

Sch\"{u}tzenberger graphs play in combinatorial inverse semigroups theory the role that Cayley graphs play in combinatorial group theory. In fact, if $G=\mipres{X}{T}$ happen to be a group (with identity $1_G$), then the Sch\"{u}tzenberger graph $\schG{(X;T;1_G)}$ of its unique $\mathcal{R}$-class is exactly  the Cayley graph of the group $G$.

\begin{thebibliography}{9}
\bibitem{b:petrich} N. Petrich, \emph{Inverse Semigroups}, Wiley, New York, 1984.
\bibitem{b:steph} J.B. Stephen, \emph{Presentation of inverse monoids}, J. Pure Appl. Algebra 63 (1990) 81-112.
\end{thebibliography}
%%%%%
%%%%%
\end{document}
