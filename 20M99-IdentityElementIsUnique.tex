\documentclass[12pt]{article}
\usepackage{pmmeta}
\pmcanonicalname{IdentityElementIsUnique}
\pmcreated{2013-03-22 18:01:20}
\pmmodified{2013-03-22 18:01:20}
\pmowner{pahio}{2872}
\pmmodifier{pahio}{2872}
\pmtitle{identity element is unique}
\pmrecord{11}{40539}
\pmprivacy{1}
\pmauthor{pahio}{2872}
\pmtype{Theorem}
\pmcomment{trigger rebuild}
\pmclassification{msc}{20M99}
\pmsynonym{neutral element is unique}{IdentityElementIsUnique}
\pmsynonym{uniqueness of identity element}{IdentityElementIsUnique}
%\pmkeywords{monoid}
\pmrelated{Group}
\pmrelated{UniquenessOfInverseForGroups}
\pmrelated{ZeroVectorInAVectorSpaceIsUnique}
\pmrelated{AbsorbingElement}

% this is the default PlanetMath preamble.  as your knowledge
% of TeX increases, you will probably want to edit this, but
% it should be fine as is for beginners.

% almost certainly you want these
\usepackage{amssymb}
\usepackage{amsmath}
\usepackage{amsfonts}

% used for TeXing text within eps files
%\usepackage{psfrag}
% need this for including graphics (\includegraphics)
%\usepackage{graphicx}
% for neatly defining theorems and propositions
 \usepackage{amsthm}
% making logically defined graphics
%%%\usepackage{xypic}

% there are many more packages, add them here as you need them

% define commands here

\theoremstyle{definition}
\newtheorem*{thmplain}{Theorem}

\begin{document}
\textbf{Theorem.}\, The identity element of a monoid is unique.\\

{\em Proof.}\, Let $e$ and $e'$ be identity elements of a monoid \,$(G,\,\cdot)$.\, Since $e$ is an identity element, one has\, $e\!\cdot\!e' = e'$.\, Since $e'$ is an identity element, one has also\, $e\!\cdot\!e' = e$.\, Thus
$$e' \;=\; e\!\cdot\!e' \;=\; e,$$
i.e. both identity elements are the same (in inferring this result from the two first equations, one has used the \PMlinkname{symmetry}{Symmetric} and transitivity of the equality relation).\\


\textbf{Note.}\, The theorem also proves the uniqueness of e.g. the identity element of a group, the \PMlinkname{additive identity}{Ring} 0 of a ring or a field, and the \PMlinkname{multiplicative identity}{Ring} 1 of a field.
%%%%%
%%%%%
\end{document}
