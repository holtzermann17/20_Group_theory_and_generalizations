\documentclass[12pt]{article}
\usepackage{pmmeta}
\pmcanonicalname{Klein4group}
\pmcreated{2013-03-22 12:49:02}
\pmmodified{2013-03-22 12:49:02}
\pmowner{Algeboy}{12884}
\pmmodifier{Algeboy}{12884}
\pmtitle{Klein 4-group}
\pmrecord{26}{33139}
\pmprivacy{1}
\pmauthor{Algeboy}{12884}
\pmtype{Topic}
\pmcomment{trigger rebuild}
\pmclassification{msc}{20K99}
\pmsynonym{Klein four-group}{Klein4group}
\pmsynonym{Viergruppe}{Klein4group}
\pmrelated{GroupsInField}
\pmrelated{Klein4Ring}
\pmrelated{PrimeResidueClass}
\pmrelated{AbelianGroup2}

\usepackage{latexsym}
\usepackage{amssymb}
\usepackage{amsmath}
\usepackage{amsfonts}
\usepackage{amsthm}

%%\usepackage{xypic}
\DeclareMathOperator{\Aut}{Aut }

%-----------------------------------------------------

%       Standard theoremlike environments.

%       Stolen directly from AMSLaTeX sample

%-----------------------------------------------------

%% \theoremstyle{plain} %% This is the default

\newtheorem{thm}{Theorem}

\newtheorem{coro}[thm]{Corollary}

\newtheorem{lem}[thm]{Lemma}

\newtheorem{lemma}[thm]{Lemma}

\newtheorem{prop}[thm]{Proposition}

\newtheorem{conjecture}[thm]{Conjecture}

\newtheorem{conj}[thm]{Conjecture}

\newtheorem{defn}[thm]{Definition}

\newtheorem{remark}[thm]{Remark}

\newtheorem{ex}[thm]{Example}



%\countstyle[equation]{thm}



%--------------------------------------------------

%       Item references.

%--------------------------------------------------


\newcommand{\exref}[1]{Example-\ref{#1}}

\newcommand{\thmref}[1]{Theorem-\ref{#1}}

\newcommand{\defref}[1]{Definition-\ref{#1}}

\newcommand{\eqnref}[1]{(\ref{#1})}

\newcommand{\secref}[1]{Section-\ref{#1}}

\newcommand{\lemref}[1]{Lemma-\ref{#1}}

\newcommand{\propref}[1]{Prop\-o\-si\-tion-\ref{#1}}

\newcommand{\corref}[1]{Cor\-ol\-lary-\ref{#1}}

\newcommand{\figref}[1]{Fig\-ure-\ref{#1}}

\newcommand{\conjref}[1]{Conjecture-\ref{#1}}


% Normal subgroup or equal.

\providecommand{\normaleq}{\unlhd}

% Normal subgroup.

\providecommand{\normal}{\lhd}

\providecommand{\rnormal}{\rhd}
% Divides, does not divide.

\providecommand{\divides}{\mid}

\providecommand{\ndivides}{\nmid}


\providecommand{\union}{\cup}

\providecommand{\bigunion}{\bigcup}

\providecommand{\intersect}{\cap}

\providecommand{\bigintersect}{\bigcap}










\begin{document}
The \emph{Klein 4-group} is the subgroup $V$ (Vierergruppe) of
$S_4$ (see symmetric group) consisting of the following
4 permutations:  
$$(),\; (12)(34),\; (13)(24),\; (14)(23).$$
(see cycle notation). This is an
abelian group, isomorphic to the product $\mathbb{Z}_2\oplus \mathbb{Z}_2$.
The group is named after \PMlinkexternal{Felix  Klein}{http://www-gap.dcs.st-and.ac.uk/~history/Mathematicians/Klein.html},
a pioneering figure in the field of geometric group theory.

\section{Klein 4-group as a symmetry group}

The group $V$ is isomorphic to the automorphism group of various planar
graphs, including graphs of 4 vertices.  Yet we have

\begin{prop}
$V$ is not the automorphism group of a simple graph.
\end{prop}
\begin{proof}
Suppose $V$ is the automorphism group of a simple graph $G$.
Because $V$ contains the permutations $(12)(34)$, $(13)(24)$ and $(14)(23)$
it follows the degree of every vertex is the same -- we can map
every vertex to every other.  So $G$ is a regular graph on 4 vertices.
This makes $G$ isomorphic to one of the following 4 graphs:
\[
\begin{xy}<10mm,0mm>:<0mm,10mm>::
( 0, 0) *+{1} = "1";
( 1, 0) *+{2} = "2";
( 1, 1) *+{3} = "3";
( 0, 1) *+{4} = "4";
\end{xy}
\qquad
\begin{xy}<10mm,0mm>:<0mm,10mm>::
( 0, 0) *+{1} = "1";
( 1, 0) *+{2} = "2";
( 1, 1) *+{3} = "3";
( 0, 1) *+{4} = "4";
"1"; "2" **@{-};
"3"; "4" **@{-};
\end{xy}
\qquad
\begin{xy}<10mm,0mm>:<0mm,10mm>::
( 0, 0) *+{1} = "1";
( 1, 0) *+{2} = "2";
( 1, 1) *+{3} = "3";
( 0, 1) *+{4} = "4";
"1"; "2" **@{-};
"2"; "3" **@{-};
"3"; "4" **@{-};
"4"; "1" **@{-};
\end{xy}
\qquad
\begin{xy}<10mm,0mm>:<0mm,10mm>::
( 0, 0) *+{1} = "1";
( 1, 0) *+{2} = "2";
(-0.5,0.86) *+{3} = "3";
(-0.5,-0.86) *+{4} = "4";
"1"; "2" **@{-};
"1"; "3" **@{-};
"1"; "4" **@{-};
"2"; "3" **@{-};
"2"; "4" **@{-};
"3"; "4" **@{-};
\end{xy}.
\]  
In order the automorphism groups of these graphs are $S_4$, 
$\langle (12),(34)\rangle$, $\langle (12),(1234)\rangle$ and $S_4$.
None of these are $V$, though the second is isomorphic to $V$.
\end{proof}

Though $V$ cannot be realized as an automorphism group of a planar graph
it can be realized as the set of symmetries of a polygon, in particular,
a non-square rectangle.
\[
\begin{xy}<10mm,0mm>:<0mm,10mm>::
( 0, 0) *+{1} = "1";
( 2, 0) *+{2} = "2";
( 2, 1) *+{3} = "3";
( 0, 1) *+{4} = "4";
"1"; "2" **@{-};
"2"; "3" **@{-};
"3"; "4" **@{-};
"4"; "1" **@{-};
\end{xy}
\]
We can rotate by $180^\circ$ which corresponds to the permutation
$(13)(24)$.  We can also flip the rectangle over the horizontal diagonal
which gives the permutation $(14)(23)$, and finally also over the vertical
diagonal which gives  the permutation $(12)(34)$.
\[
\begin{xy}<10mm,0mm>:<0mm,10mm>::
( 0, 0) *+{3} = "1";
( 2, 0) *+{4} = "2";
( 2, 1) *+{1} = "3";
( 0, 1) *+{2} = "4";
"1"; "2" **@{-};
"2"; "3" **@{-};
"3"; "4" **@{-};
"4"; "1" **@{-};
\end{xy},
\quad
\begin{xy}<10mm,0mm>:<0mm,10mm>::
( 0, 0) *+{4} = "1";
( 2, 0) *+{3} = "2";
( 2, 1) *+{2} = "3";
( 0, 1) *+{1} = "4";
"1"; "2" **@{-};
"2"; "3" **@{-};
"3"; "4" **@{-};
"4"; "1" **@{-};
\end{xy},
\quad
\begin{xy}<10mm,0mm>:<0mm,10mm>::
( 0, 0) *+{2} = "1";
( 2, 0) *+{1} = "2";
( 2, 1) *+{4} = "3";
( 0, 1) *+{3} = "4";
"1"; "2" **@{-};
"2"; "3" **@{-};
"3"; "4" **@{-};
"4"; "1" **@{-};
\end{xy}.
\]
An important corollary to this realization is
\begin{prop}\label{prop:d8}
Given a square with vertices labeled in any way by $\{1,2,3,4\}$, then the
full symmetry group (the dihedral group of order 8, $D_8$) contains $V$.
\end{prop}


\section{Klein 4-group as a vector space}

As $V$ is isomorphic to $\mathbb{Z}_2\oplus \mathbb{Z}_2$ it is a 2-dimensional
vector space over the Galois field $\mathbb{Z}_2$.  The projective geometry
of $V$ -- equivalently, the lattice of subgroups -- is given in the following
Hasse diagam:
\[
\begin{xy} <10mm,0mm>:<0mm,10mm>::
(0,0)	*+{\langle ()\rangle}	="1.1";
(-2,1)	*+{\langle (12)(34)\rangle}	="2.1";
(0,1)	*+{\langle (13)(24)\rangle}	="3.1";
(2,1)	*+{\langle (14)(23)\rangle}	="4.1";
(0,2)	*+{V}	="5.1";
"2.1";	"1.1"	**@{-};
"3.1";	"1.1"	**@{-};
"4.1";	"1.1"	**@{-};
"5.1";	"4.1"	**@{-};
"5.1";	"3.1"	**@{-};
"5.1";	"2.1"	**@{-};
\end{xy}
\]
The automorphism group of a vector space is called the general linear
group and so in our context $\Aut V\cong GL(2,2)$.  As we can interchange
any basis of a vector space we can label the elements $e_1=(12)(34)$,
$e_2=(13)(24)$ and $e_3=(14)(23)$ so that we have the permutations
$(e_1,e_2)$ and $(e_2,e_3)$ and so we generate all permutations on
$\{e_1,e_2,e_3\}$.  This proves:

\begin{prop}\label{prop:agl}
$\Aut V\cong GL(2,2)\cong S_3$.  Furthermore, the affine linear group
of $V$ is $AGL(2,2)=V\rtimes S_3$.
\end{prop}

\section{Klein 4-group as a normal subgroup}

Because $V$ is a subgroup of $S_4$ we can consider its conjugates.  Because
conjugation in $S_4$ respects the cycle structure.  From this we see that
the conjugacy class in $S_4$ of every element of $V$ lies again in $V$.  Thus
$V$ is normal.  This now allows us to combine both of the previous sections
to outline the exceptional nature (amongst $S_n$ families) of $S_4$.  We 
collect these into

\begin{thm}
\begin{enumerate}
\item $V$ is a normal subgroup of $S_4$.
\item $V$ is contained in $A_4$ and so it is a normal subgroup of $A_4$.
\item $V$ is the Sylow 2-subgroup of $A_4$.
\item $V$ is the intersection of all Sylow 2-subgroups of $S_4$, that is,
the $2$-core of $S_4$.
\item $S_4/V\cong S_3$.
\item $S_4\cong AGL(2,2)\cong V\rtimes S_3$.
\end{enumerate}
\end{thm}
\begin{proof}
We have already argued that $V$ is normal in $S_4$.  Upon inspecting the
elements of $V$ we see $V$ contains only even permutations so $V\leq A_4$
and consequently $V$ is normal in $A_4$ as well.  As $|A_4|=12$ and $|V|=4$
we establish $V$ is a Sylow 2-subgroup of $A_4$.  But $V$ is normal so it
the Sylow 2-subgroup of $A_4$ (Sylow subgroups are conjugate.)  

Now notice that the dihedral group $D_8$ acts on a square and so it is
represented as a permutation group on 4 vertices, so $D_8$ embeds in $S_4$.
As $|D_8|=8$ and $|S_4|=24$, $D_8$ is a Sylow 2-subgroup of $S_4$ and so
all Sylow 2-subgroups of $S_4$ are embeddings of $D_8$ (in particular various
relabellings of the vertices of the square.)  But by Proposition \ref{prop:d8}
we know that each embedding contains $V$.  As there are 3 non-equal 
embeddings of $D_8$ (think of the 3 non-equal labellings of a square) we 
know that the intersection of these $D_8$ is a proper subgroup of $D_8$.
As $V$ is a maximal subgroup of each $D_8$ and contained in each, $V$ is
the intersection of all these embeddings.

Now the action of $S_4$ by conjugation on the Sylow 2-subgroups $D_8$
permutes all 3 (again Sylow subgroups are conjugate) so $S_4\mapsto S_3$.
Indeed, $V$ is in the kernel of this action as $V$ is in each $D_8$.
Indeed a three cycle $(123)$ permutes the $D_8$'s with no fixed point
(consider the relabellings) and $(12)$ fixes only one.  So $S_4$ maps 
onto $S_3$ and so the kernel is precisely $V$.  Thus $S_4/V=S_3$.

Now we can embed $S_3$ into $S_4$ as $\langle (123),(12)\rangle$ so 
$V\intersect S_3=1$, $VS_3=S_4$ so $S_4=V\ltimes S_3$.  Finally, $AGL(2,2)$
acts transitively on the four points of the vector space $V$ so 
$AGL(2,2)$ embeds in $S_4$.  And by Proposition \ref{prop:agl} we conclude
$S_4\cong AGL(2,2)$.
\end{proof}

We can make similar arguments about subgroups of symmetries for 
larger regular polygons.  Likewise for other 2-dimensional vector spaces
we can establish similar structural properties.  However it is only
when we study we involve $V$ that we find these two methods intersect
in a this exceptionally parallel fashion.  Thus we establish the exceptional
structure of $S_4$.  For all other $S_n$'s, $A_n$ is the only proper normal
subgroup.

We can view the properties of our theorem in a geometric way as follows:
$S_4$ is the group of symmetries of a tetrahedron. There is an induced  action 
of  $S_4$ on the six edges of the tetrahedron. Observing that this action preserves incidence relations one gets an action of $S_4$ on the three pairs 
of opposite edges.

\section{Other properties}

$V$ is non-cyclic and of smallest possible order with this property.

$V$ is transitive and regular.  Indeed $V$ is the (unique) regular representation of 
$\mathbb{Z}_2\oplus \mathbb{Z}_2$.  The other 3 subgroups of $S_4$ which are isomorphic to 
$\mathbb{Z}_2\oplus \mathbb{Z}_2$ are not transitive.

$V$ is the symmetry group of the Riemannian curvature tensor.
%%%%%
%%%%%
\end{document}
