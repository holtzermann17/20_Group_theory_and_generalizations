\documentclass[12pt]{article}
\usepackage{pmmeta}
\pmcanonicalname{BicyclicSemigroup}
\pmcreated{2013-03-22 13:09:57}
\pmmodified{2013-03-22 13:09:57}
\pmowner{mclase}{549}
\pmmodifier{mclase}{549}
\pmtitle{bicyclic semigroup}
\pmrecord{8}{33609}
\pmprivacy{1}
\pmauthor{mclase}{549}
\pmtype{Definition}
\pmcomment{trigger rebuild}
\pmclassification{msc}{20M99}
\pmsynonym{bicyclic monoid}{BicyclicSemigroup}
\pmdefines{bicyclic}

% this is the default PlanetMath preamble.  as your knowledge
% of TeX increases, you will probably want to edit this, but
% it should be fine as is for beginners.

% almost certainly you want these
\usepackage{amssymb}
\usepackage{amsmath}
\usepackage{amsfonts}

% used for TeXing text within eps files
%\usepackage{psfrag}
% need this for including graphics (\includegraphics)
%\usepackage{graphicx}
% for neatly defining theorems and propositions
%\usepackage{amsthm}
% making logically defined graphics
%%%\usepackage{xypic}

% there are many more packages, add them here as you need them

% define commands here
\newcommand{\Bicyc}[2]{\mathcal{C}({#1},{#2})}
\begin{document}
The \emph{bicyclic semigroup} $\Bicyc{p}{q}$ is the
monoid generated by $\{p, q\}$ with
the single relation $pq = 1$.

The elements of $\Bicyc{p}{q}$ are all words of the
form $q^n p^m$ for $m, n \geq 0$ (with the understanding $p^0 = q^0 = 1$).
These words are multiplied as follows:
\begin{displaymath}
  q^n p^m q^k p^l = \begin{cases}
    q^{n+k-m} p^l \qquad & \text{if }m \leq k, \\
    q^n p^{l+m-k}        & \text{if }m \geq k.
  \end{cases}
\end{displaymath}

It is apparent that $\Bicyc{p}{q}$ is simple, for if
$q^n p^m$ is an element of $\Bicyc{p}{q}$, then
$1 = p^n (q^n p^m) q^m$ and so $S^1q^n p^mS^1 = S$.

It is also easy to see that $\Bicyc{p}{q}$ is an inverse semigroup: the element $q^np^m$ has inverse $q^mp^n$.

It is useful to picture some further properties of $\Bicyc{p}{q}$ by
arranging the elements in a table:
\begin{displaymath}
  \begin{matrix}
    1   & p    & p^2    & p^3    & p^4    & \dots \\
    q   & qp   & qp^2   & qp^3   & qp^4   & \dots \\
    q^2 & q^2p & q^2p^2 & q^2p^3 & q^2p^4 & \dots \\
    q^3 & q^3p & q^3p^2 & q^3p^3 & q^3p^4 & \dots \\
    q^4 & q^4p & q^4p^2 & q^4p^3 & q^4p^4 & \dots \\
	\vdots & \vdots & \vdots & \vdots & \vdots & \ddots
  \end{matrix}
\end{displaymath}

Then the elements below any horizontal line drawn through this
table form a right ideal and the elements to the right of any vertical
line form a left ideal.
Further, the elements on the diagonal are all idempotents
and their standard ordering is
$$1 > qp > q^2 p^2 > q^3 p^3 > \dotsb.$$
%%%%%
%%%%%
\end{document}
