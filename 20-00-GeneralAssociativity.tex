\documentclass[12pt]{article}
\usepackage{pmmeta}
\pmcanonicalname{GeneralAssociativity}
\pmcreated{2013-03-22 14:35:50}
\pmmodified{2013-03-22 14:35:50}
\pmowner{pahio}{2872}
\pmmodifier{pahio}{2872}
\pmtitle{general associativity}
\pmrecord{21}{36165}
\pmprivacy{1}
\pmauthor{pahio}{2872}
\pmtype{Theorem}
\pmcomment{trigger rebuild}
\pmclassification{msc}{20-00}
\pmrelated{Semigroup}
\pmrelated{EveryRingIsAnIntegerAlgebra}
\pmrelated{InverseFormingInProportionToGroupOperation}
\pmrelated{CosineAtMultiplesOfStraightAngle}
\pmrelated{InfixNotation}
\pmrelated{OperationsOnRelations}
\pmrelated{Difference2}
\pmrelated{FactorsWithMinusSign}
\pmrelated{IdealOfElementsWithFiniteOrder}
\pmrelated{GeneralCommutativity}
\pmrelated{Characteri}
\pmdefines{power}
\pmdefines{multiple}
\pmdefines{even power}
\pmdefines{odd power}
\pmdefines{even multiple}
\pmdefines{odd multiple}

% this is the default PlanetMath preamble.  as your knowledge
% of TeX increases, you will probably want to edit this, but
% it should be fine as is for beginners.

% almost certainly you want these
\usepackage{amssymb}
\usepackage{amsmath}
\usepackage{amsfonts}

% used for TeXing text within eps files
%\usepackage{psfrag}
% need this for including graphics (\includegraphics)
%\usepackage{graphicx}
% for neatly defining theorems and propositions
 \usepackage{amsthm}
% making logically defined graphics
%%%\usepackage{xypic}

% there are many more packages, add them here as you need them

% define commands here
\theoremstyle{definition}
\newtheorem*{thmplain}{Theorem}
\begin{document}
If an associative binary operation of a set $S$ is denoted by ``$\cdot$'', the associative law in $S$ is usually expressed as
  $$(a\!\cdot\!b)\!\cdot\!c = a\!\cdot\!(b\!\cdot\!c),$$
or leaving out the dots,\, $(ab)c = a(bc)$.\, Thus the common value of both \PMlinkescapetext{sides} may be denoted as $abc$.\, With four elements of $S$ we can \PMlinkescapetext{calculate}, using only the associativity, as follows:
     $$(ab)(cd) = a(b(cd)) = a((bc)d)= (a(bc))d = ((ab)c)d$$
So we may denote the common value of those five expressions as $abcd$. 

\begin{thmplain}
 \, The expression formed of elements $a_1$, $a_2$, \ldots, $a_n$ of $S$ \PMlinkescapetext{represents always the same element of $S$ independently on how one has joined them together with the associative operation and parentheses, if only the order of the elements is every time the same}.\, The common value is denoted by $a_1a_2\ldots a_n$.
\end{thmplain}

\textbf{Note.}\, The $n$ elements can be joined, without changing their \PMlinkescapetext{order}, in $\frac{(2n-2)!}{n!(n-1)!}$ ways (see the Catalan numbers).

The theorem is proved by induction on $n$.\, The cases\, $n = 3$\, and\, $n = 4$\, have been stated \PMlinkescapetext{right} above.

Let\, $n \in \mathbb{Z}_+$.\, The expression $aa \ldots a$ with $n$ equal ``factors'' $a$ may be denoted by $a^n$ and called a {\em power} of $a$.\, If the associative operation is denoted ``additively'', then the ``sum''\, 
$a\!+\!a\!+\cdots+\!a$\, of $n$ equal elements $a$ is denoted by $na$ and called a {\em multiple} of $a$; hence in every ring one may consider powers and multiples.  According to whether $n$ is an even or an odd number, one may speak of {\em even powers}, {\em odd powers}, {\em even multiples}, {\em odd multiples}.

The following two laws can be proved by induction:
   $$a^m\cdot a^n = a^{m+n}$$
     $$(a^m)^n = a^{mn}$$
In \PMlinkescapetext{additive} notation:
     $$ma\!+\!na = (m\!+\!n)a,$$
      $$n(ma) = (mn)a$$

\textbf{Note.}\, If the set $S$ together with its operation is a group, then the notion of multiple $na$ resp. power $a^n$ can be extended for negative integer and zero values of $n$ by means of the inverse and identity elements.\, The above laws remain in \PMlinkescapetext{force}.
%%%%%
%%%%%
\end{document}
