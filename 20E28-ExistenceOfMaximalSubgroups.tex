\documentclass[12pt]{article}
\usepackage{pmmeta}
\pmcanonicalname{ExistenceOfMaximalSubgroups}
\pmcreated{2013-03-22 16:24:54}
\pmmodified{2013-03-22 16:24:54}
\pmowner{Algeboy}{12884}
\pmmodifier{Algeboy}{12884}
\pmtitle{existence of maximal subgroups}
\pmrecord{5}{38564}
\pmprivacy{1}
\pmauthor{Algeboy}{12884}
\pmtype{Example}
\pmcomment{trigger rebuild}
\pmclassification{msc}{20E28}
\pmrelated{PropertyOfInfiniteSimpleGroup}
\pmrelated{JordanHolderDecomposition}
\pmrelated{EveryRingHasAMaximalIdeal}

\usepackage{latexsym}
\usepackage{amssymb}
\usepackage{amsmath}
\usepackage{amsfonts}
\usepackage{amsthm}

%%\usepackage{xypic}

%-----------------------------------------------------

%       Standard theoremlike environments.

%       Stolen directly from AMSLaTeX sample

%-----------------------------------------------------

%% \theoremstyle{plain} %% This is the default

\newtheorem{thm}{Theorem}

\newtheorem{coro}[thm]{Corollary}

\newtheorem{lem}[thm]{Lemma}

\newtheorem{lemma}[thm]{Lemma}

\newtheorem{prop}[thm]{Proposition}

\newtheorem{conjecture}[thm]{Conjecture}

\newtheorem{conj}[thm]{Conjecture}

\newtheorem{defn}[thm]{Definition}

\newtheorem{remark}[thm]{Remark}

\newtheorem{ex}[thm]{Example}



%\countstyle[equation]{thm}



%--------------------------------------------------

%       Item references.

%--------------------------------------------------


\newcommand{\exref}[1]{Example-\ref{#1}}

\newcommand{\thmref}[1]{Theorem-\ref{#1}}

\newcommand{\defref}[1]{Definition-\ref{#1}}

\newcommand{\eqnref}[1]{(\ref{#1})}

\newcommand{\secref}[1]{Section-\ref{#1}}

\newcommand{\lemref}[1]{Lemma-\ref{#1}}

\newcommand{\propref}[1]{Prop\-o\-si\-tion-\ref{#1}}

\newcommand{\corref}[1]{Cor\-ol\-lary-\ref{#1}}

\newcommand{\figref}[1]{Fig\-ure-\ref{#1}}

\newcommand{\conjref}[1]{Conjecture-\ref{#1}}


% Normal subgroup or equal.

\providecommand{\normaleq}{\unlhd}

% Normal subgroup.

\providecommand{\normal}{\lhd}

\providecommand{\rnormal}{\rhd}
% Divides, does not divide.

\providecommand{\divides}{\mid}

\providecommand{\ndivides}{\nmid}


\providecommand{\union}{\cup}

\providecommand{\bigunion}{\bigcup}

\providecommand{\intersect}{\cap}

\providecommand{\bigintersect}{\bigcap}










\begin{document}
Because every finite group is a finite set, every chain of proper subgroups
of a finite group has a maximal element and thus every finite group has
a maximal subgroup.  The same applies to maximal normal subgroups.

However, there are infinite groups, even abelian, with no maximal subgroups and
no maximal normal subgroups.  The Pr\"ufer group 
  \[\mathbb{Z}_{p^\infty}=\lim_{\longleftarrow}\mathbb{Z}_{p^i}\]
(for any prime $p$) is an example of an abelian group with no maximal subgroups.
As the group is abelian all subgroups are normal so it also has no maximal
normal subgroups.  Such groups fail to fit the hypothesis of the Jordan-H\"older decomposition theorem as they do not have the ascending chain condition and so we cannot assign a composition series to such groups.

This stands in contrast to the category of unital rings where if one assumes Zorn's lemma (axiom of choice) then one may prove every unital ring
has a maximal ideal. 
%%%%%
%%%%%
\end{document}
