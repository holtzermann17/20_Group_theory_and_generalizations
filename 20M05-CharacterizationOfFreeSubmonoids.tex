\documentclass[12pt]{article}
\usepackage{pmmeta}
\pmcanonicalname{CharacterizationOfFreeSubmonoids}
\pmcreated{2013-03-22 18:21:32}
\pmmodified{2013-03-22 18:21:32}
\pmowner{Ziosilvio}{18733}
\pmmodifier{Ziosilvio}{18733}
\pmtitle{characterization of free submonoids}
\pmrecord{7}{40997}
\pmprivacy{1}
\pmauthor{Ziosilvio}{18733}
\pmtype{Theorem}
\pmcomment{trigger rebuild}
\pmclassification{msc}{20M05}
\pmclassification{msc}{20M10}
\pmdefines{intersection of free submonoids is free}

% this is the default PlanetMath preamble.  as your knowledge
% of TeX increases, you will probably want to edit this, but
% it should be fine as is for beginners.

% almost certainly you want these
\usepackage{amssymb}
\usepackage{amsmath}
\usepackage{amsfonts}

% used for TeXing text within eps files
%\usepackage{psfrag}
% need this for including graphics (\includegraphics)
%\usepackage{graphicx}
% for neatly defining theorems and propositions
%\usepackage{amsthm}
% making logically defined graphics
%%%\usepackage{xypic}

% there are many more packages, add them here as you need them

% define commands here

\begin{document}
\newcommand{\mgs}{\ensuremath{\mathrm{mgs}}}

\newtheorem{definition}{Definition}
\newtheorem{theorem}{Theorem}
\newtheorem{corollary}{Corollary}

Let $A$ be an arbitrary set,
let $A^\ast$ be the free monoid on $A$,
and let $e$ be the identity element (empty word) of $A^\ast$.
Let $M$ be a submonoid of $A^\ast$
and let $\mgs(M)$ be its minimal generating set.

We recall the universal property of free monoids:
for every mapping $f:A\to M$ with $M$ a monoid,
there exists a unique morphism $\phi:A^\ast\to M$
such that $\phi(a)=f(a)$ for every $a\in A$.

\begin{theorem} \label{thm:char-fsm}
The following are equivalent.
\begin{enumerate}
\item \label{item:char-fsm-fsm}
$M$ is a free submonoid of $A^\ast$.
\item \label{item:char-fsm-eq}
Any equation
\begin{equation} \label{eq:char-fsm-eq}
x_1\cdots x_n=y_1\cdots y_m\;,\;
x_1,\ldots,x_n,y_1,\ldots,y_m\in\mgs(M)
\end{equation}
has only the trivial solutions $n=m,x_1=y_1,\ldots,x_n=y_n$.
\item \label{item:char-fsm-word}
For every $w\in A^\ast$,
if $p,q\in M$ exist such that $pw,wq\in M$,
then $w\in M$.
\end{enumerate}
\end{theorem}

From point~\ref{item:char-fsm-word} of Theorem~\ref{thm:char-fsm} follows

\begin{corollary} \label{cor:free-monoid-inters}
An intersection of free submonoids of $A^\ast$
is a free submonoid of $A^\ast$.
\end{corollary}

As a consequence of Theorem~\ref{thm:char-fsm},
there is no Nielsen-Schreier theorem for monoids.
In fact, consider $A=\{a,b\}$ and $Y=\{a,ab,ba\}\subseteq A^\ast$:
then $\mgs(Y^\ast)=Y$,
but $x_1x_2=y_1y_2$ has a nontrivial solution over $Y$,
namely, $(ab)a=a(ba)$.

We now prove Theorem~\ref{thm:char-fsm}.

\emph{Point~\ref{item:char-fsm-eq} implies point~\ref{item:char-fsm-fsm}.}
Let $f:\mgs(M)\to B$ be a bijection.
By the universal property of free monoids,
there exists a unique morphism $\phi:B^\ast\to M$ that extends $f^{-1}$;
such a morphism is clearly surjective.
Moreover, any equation $\phi(b_1\cdots b_n)=\phi(b'_1\cdots b'_m)$
translates into an equation of the form (\ref{eq:char-fsm-eq}),
which by hypothesis has only trivial solutions:
therefore $n=m$, $b_i=b'_i$ for all $i$, and $\phi$ is injective.

\emph{Point~\ref{item:char-fsm-word} implies point~\ref{item:char-fsm-eq}.}
Suppose the existence of $p,q\in M$ such that $pw,wq\in M$
implies $w\in A^\ast$ is actually in $M$.
Consider an equation of the form (\ref{eq:char-fsm-eq})
which is a counterexample to the thesis,
and such that the length of the compared words is minimal:
we may suppose $x_1$ is a prefix of $y_1$,
so that $y_1=x_1w$ for some $w\in A^\ast$.
Put $p=x_1,q=y_2\cdots y_m$:
then $pw=y_1$ and $wq=x_2\cdots x_n$ belong to $M$ by construction.
By hypothesis, this implies $w\in M$:
then $y_1\in\mgs(M)$ equals a product $x_1w$ with $x_1,w\in M$---which,
by definition of $\mgs(M)$,
is only possible if $w=e$.
Then $x_1=y_1$ and $x_2\cdots x_n=y_2\cdots y_m$:
since we had chosen a counterexample of minimal length,
$n=m,x_2=y_2,\ldots,x_n=y_n$.
Then the original equation has only trivial solutions,
and is not a counterexample after all.

\emph{Point~\ref{item:char-fsm-fsm} implies point~\ref{item:char-fsm-word}.}
Let $\phi:B^\ast\to M$ be an isomorphism of monoids.
Then clearly $\mgs(M)\subseteq\phi(B)$;
since removing $m=\phi(b)$ from $\mgs(M)$ removes $\phi(b^\ast)$ from $M$,
the equality holds.
Let $w\in A^\ast$ and let $p,q\in M$ satisfy $pw,wq\in M$:
put $x=\phi^{-1}(p)$, $y=\phi^{-1}(q)$,
$r=\phi^{-1}(pw)$, $s=\phi^{-1}(wq)$.
Then $\phi(xs)=\phi(ry)=pwq\in M$, so $xs=ry$:
this is an equality over $B^\ast$,
and is satisfied only by $r=xu$, $s=uy$ for some $u$.
Then $w=\phi(u)\in M$.

\begin{thebibliography}{99}

\bibitem{l97}
M. Lothaire.
\textit{Combinatorics on words.}
Cambridge University Press 1997.

\end{thebibliography}

%%%%%
%%%%%
\end{document}
