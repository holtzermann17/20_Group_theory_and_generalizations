\documentclass[12pt]{article}
\usepackage{pmmeta}
\pmcanonicalname{GL2mathbbZ}
\pmcreated{2013-03-22 16:31:38}
\pmmodified{2013-03-22 16:31:38}
\pmowner{rm50}{10146}
\pmmodifier{rm50}{10146}
\pmtitle{$GL_2(\mathbb{Z})$}
\pmrecord{7}{38707}
\pmprivacy{1}
\pmauthor{rm50}{10146}
\pmtype{Application}
\pmcomment{trigger rebuild}
\pmclassification{msc}{20G15}

\endmetadata

% this is the default PlanetMath preamble.  as your knowledge
% of TeX increases, you will probably want to edit this, but
% it should be fine as is for beginners.

% almost certainly you want these
\usepackage{amssymb}
\usepackage{amsmath}
\usepackage{amsfonts}

% used for TeXing text within eps files
%\usepackage{psfrag}
% need this for including graphics (\includegraphics)
%\usepackage{graphicx}
% for neatly defining theorems and propositions
%\usepackage{amsthm}
% making logically defined graphics
%%%\usepackage{xypic}

% there are many more packages, add them here as you need them

% define commands here
\newcommand{\Ints}{\mathbb{Z}}

\begin{document}
Let $M_2(\Ints)$ be the ring of $2\textrm{x}2$ matrices with integer entries, and define
$GL_2(\Ints)$ to be the subring of matrices invertible over $\Ints$. Thus for $M\in M_2(\Ints)$,
\[M\in GL_2(\Ints)\iff \det M=\pm 1\]

Let $Aut_{\Ints}(\Ints\oplus\Ints)$ be the ring of automorphisms of $\Ints\oplus\Ints$ as a $\Ints$-module. Then $GL_2(\Ints)\cong Aut_{\Ints}(\Ints\oplus\Ints)$
as rings, under the obvious operations.

To see this, we demonstrate a natural correspondence between endomorphisms of $\Ints\oplus\Ints$ and $M_2(\Ints)$ and show that invertible endomorphisms correspond to invertible matrices. Let $\varphi: \Ints\oplus\Ints\to \Ints\oplus\Ints$ be any ring homomorphism. It is clear that $\varphi$ is determined by its action on $(1,0)$ and $(0,1)$, since
\[\varphi(x,y)=\varphi(x(1,0)+y(0,1))=x\varphi(1,0)+y\varphi(0,1)\]
Suppose then that $\varphi(1,0)=(a,b)$ and $\varphi(0,1)=(c,d)$. Then
\[\varphi(x,y)=(ax,bx)+(cy,dy)=(ax+cy,bx+dy)=\begin{pmatrix}a&c\\b&d\end{pmatrix}\begin{pmatrix}x\\y\end{pmatrix}\]
Now, $\varphi$ is surjective if both $(1,0)$ and $(0,1)$ are in its image. But $(1,0)\in\textrm{im}\  \varphi$ if and only if there is some $(x,y)$ such that
\begin{align*}
ax+cy&=1\\
bx+dy&=0
\end{align*}
Solving this pair of equations for $y$ we see that we must have $y(bc-ad)=1$ and thus $bc-ad=\pm 1$. Similarly, $(0,1)\in\textrm{im}\ \varphi$ if and only if $y(ad-bc)=\pm 1$. Thus $\varphi$ is surjective precisely when $ad-bc=\pm 1$, i.e. precisely when the matrix representation of $\varphi$ has determininant $\pm 1$. This then gives a map from $Aut_{\Ints}(\Ints\oplus\Ints)$ to $GL_2(\Ints)$ that is obviously a ring isomorphism. This concludes the proof.

$A=Aut_{\Ints}(\Ints\oplus\Ints)$ has a simple and well-known set of generators as a group:
\begin{align*}r&=(x,y)\mapsto(y,x)\\
s&=(x,y)\mapsto(x,x+y)
\end{align*}
Note that $s^m=(x,y)\mapsto(x,mx+y)$ for any integer $m$. We now prove this fact.

Define the subgroup $A'\subset A$ by $A'=<r,s>$, the subgroup of $A$ generated by $r$ and $s$. If $\varphi_1,\varphi_2\in A$, define $\varphi_1\sim\varphi_2$ if $\varphi_1$ and $\varphi_2$ are in the same $A'$-coset. 

Our objective is to show that $A'=A$, which we can do by showing that each $\varphi\sim e$, where $e$ is the identity transformation of $A$. This demonstration is essentially an application of the Euclidean algorithm. For suppose
\[\varphi(x,y)=(ax+cy,bx+dy)\]
Assume, by applying $r$ if necessary, that $a\leq b$, and choose $m$ such that $b=am+q, 0\leq q<a$. Then $rs^{-m}\varphi(x,y)=r(ax+cy,(b-am)x+(d-cm)y)=r(ax+cy,qx+dy)=(qx+dy,ax+cy)$, so that
\[\varphi\sim (x,y)\mapsto(qx+dy,ax+cy)\]
Continuing this process, we eventually see that
\[\varphi\sim (x,y)\mapsto(cy,bx+dy)\]
But $ad-bc=\pm 1$, so we have $bc=\pm 1$. Applying either $s^d$ or $s^{-d}$ as appropriate, we get
\[\varphi\sim (x,y)\mapsto(cy,bx)\sim(x,y)\mapsto(bx,cy)\]

Thus, we are done if we show that all such forms $(bx,cy)$ with $b,c=\pm 1$ are in the same $A'$-coset as $e$. The case where $b=c=1$ is obvious. For the other cases, note that
\begin{align*}(x,y)\mapsto(x,-y)&=s^{-1}rsrs^{-1}r\\
(x,y)\mapsto(-x,y)&=rs^{-1}rsrs^{-1}r
\end{align*}
and $(x,y)\mapsto(-x,-y)$ is obviously the composition of these two.

This result is often phrased by saying that the matrices
\[\begin{pmatrix}1&0\\1&1\end{pmatrix},\quad\begin{pmatrix}0&1\\1&0\end{pmatrix}\]
generate $GL_2(\Ints)$ as a multiplicative group.
%%%%%
%%%%%
\end{document}
