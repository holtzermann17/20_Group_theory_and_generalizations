\documentclass[12pt]{article}
\usepackage{pmmeta}
\pmcanonicalname{DirectProductsOfGroups}
\pmcreated{2013-03-22 14:52:54}
\pmmodified{2013-03-22 14:52:54}
\pmowner{yark}{2760}
\pmmodifier{yark}{2760}
\pmtitle{direct products of groups}
\pmrecord{17}{36560}
\pmprivacy{1}
\pmauthor{yark}{2760}
\pmtype{Definition}
\pmcomment{trigger rebuild}
\pmclassification{msc}{20A99}
\pmrelated{SubdirectProductOfGroups}
\pmrelated{DirectProduct2}
\pmdefines{direct product}
\pmdefines{unrestricted direct product}
\pmdefines{complete direct product}
\pmdefines{restricted direct product}
\pmdefines{direct sum}
\pmdefines{direct product of groups}
\pmdefines{unrestricted direct product of groups}
\pmdefines{restricted direct product of groups}
\pmdefines{direct sum of groups}
\pmdefines{Cartesian product of g}

\endmetadata

\usepackage{amssymb}
\usepackage{amsmath}
\usepackage{amsfonts}

\begin{document}
\PMlinkescapeword{finite support}
\PMlinkescapeword{index}
\PMlinkescapeword{pointwise}
\PMlinkescapeword{term}
\PMlinkescapephrase{cartesian product}

Let $(G_i)_{i\in I}$ be a family of groups.

The \emph{unrestricted direct product}
(or \emph{complete direct product}, or \emph{Cartesian product})
$\prod_{i\in I}G_i$
is the \PMlinkname{Cartesian product}{GeneralizedCartesianProduct}
$\prod_{i\in I}G_i$ with multiplication defined pointwise,
that is, for all $f,g\in\prod_{i\in I}G_i$ and all $i\in I$
we have $(fg)(i)=f(i)g(i)$.
It is easily verified that this multiplication
makes the Cartesian product into a group.
This construction is in fact the \PMlinkname{categorical direct product}{CategoricalDirectProduct} in the category of groups.

The \emph{restricted direct product} $\bigoplus_{i\in I}G_i$ is the subgroup of $\prod_{i\in I}G_i$ consisting of all those elements with finite support. That is, 
\[\bigoplus_{i\in I}G_i=\biggl\{f\in\prod_{i\in I}G_i\biggm| f(i)=1\hbox{ for all but finitely many }i\in I\biggr\}.\]
The restricted direct product is also called the \emph{direct sum}, although this usage is usually reserved for the case where all the $G_i$ are abelian (see \PMlinkname{direct sum of modules}{DirectSum} and \PMlinkname{categorical direct sum}{CategoricalDirectSum}).

The unqualified term \emph{direct product} can refer either to the unrestricted direct product or to the restricted direct product, depending on the author.
Note that if $I$ is finite then the unrestricted direct product and the restricted direct product are in fact the same.
The direct product of two groups $G$ and $H$ is usually written $G\times H$,
or sometimes $G\oplus H$ (or $G\coprod H$) if $G$ and $H$ are both abelian.
%%%%%
%%%%%
\end{document}
