\documentclass[12pt]{article}
\usepackage{pmmeta}
\pmcanonicalname{CubeOfANumber}
\pmcreated{2013-03-22 15:07:13}
\pmmodified{2013-03-22 15:07:13}
\pmowner{pahio}{2872}
\pmmodifier{pahio}{2872}
\pmtitle{cube of a number}
\pmrecord{15}{36859}
\pmprivacy{1}
\pmauthor{pahio}{2872}
\pmtype{Definition}
\pmcomment{trigger rebuild}
\pmclassification{msc}{20-00}
\pmsynonym{cube}{CubeOfANumber}
\pmsynonym{third power}{CubeOfANumber}
\pmrelated{SquareOfANumber}
\pmrelated{PowerFunction}
\pmrelated{CubeRoot}
\pmrelated{CubanPrime}
\pmrelated{SemicubicalParabola}
\pmdefines{cube function}

% this is the default PlanetMath preamble.  as your knowledge
% of TeX increases, you will probably want to edit this, but
% it should be fine as is for beginners.

% almost certainly you want these
\usepackage{amssymb}
\usepackage{amsmath}
\usepackage{amsfonts}

% used for TeXing text within eps files
%\usepackage{psfrag}
% need this for including graphics (\includegraphics)
%\usepackage{graphicx}
% for neatly defining theorems and propositions
%\usepackage{amsthm}
% making logically defined graphics
%%%\usepackage{xypic}

% there are many more packages, add them here as you need them

% define commands here
\begin{document}
The {\em cube of a number} $x$ is the third \PMlinkname{power}{GeneralAssociativity} $x^3$ of $x$.\, Similarly one may speak of the cube of an element $x$ in any semigroup with the operation denoted multiplicatively (cf. general associativity).

The volume of a cube (i.e. \PMlinkname{regular}{RegularPolyhedron} \PMlinkname{hexahedron}{Hexahedron}) with \PMlinkescapetext{edge length} $a$ is $a^3$; hence the name.

The {\em cube function}\, $x\mapsto x^3$\, from $\mathbb{R}$ to $\mathbb{R}$ is injective, but not as a mapping from $\mathbb{C}$ to $\mathbb{C}$; one has\, $x^3 = y^3$\, always when\, $\frac{x}{y} = \frac{-1\pm i\sqrt{3}}{2}$,\, the \PMlinkescapetext{primitive} third root of unity.
%%%%%
%%%%%
\end{document}
