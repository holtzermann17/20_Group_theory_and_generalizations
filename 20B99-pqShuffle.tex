\documentclass[12pt]{article}
\usepackage{pmmeta}
\pmcanonicalname{pqShuffle}
\pmcreated{2013-03-22 13:33:59}
\pmmodified{2013-03-22 13:33:59}
\pmowner{mathcam}{2727}
\pmmodifier{mathcam}{2727}
\pmtitle{$(p,q)$ shuffle}
\pmrecord{7}{34176}
\pmprivacy{1}
\pmauthor{mathcam}{2727}
\pmtype{Definition}
\pmcomment{trigger rebuild}
\pmclassification{msc}{20B99}
\pmclassification{msc}{05A05}
\pmsynonym{shuffle}{pqShuffle}
\pmrelated{ShuffleOfLanguages}

\endmetadata

% this is the default PlanetMath preamble.  as your knowledge
% of TeX increases, you will probably want to edit this, but
% it should be fine as is for beginners.

% almost certainly you want these
\usepackage{amssymb}
\usepackage{amsmath}
\usepackage{amsfonts}
\usepackage{amsthm}

% used for TeXing text within eps files
%\usepackage{psfrag}
% need this for including graphics (\includegraphics)
%\usepackage{graphicx}
% for neatly defining theorems and propositions
%\usepackage{amsthm}
% making logically defined graphics
%%%\usepackage{xypic}

% there are many more packages, add them here as you need them

% define commands here

\newtheorem{df}{Definition}
\newtheorem*{defnn}{Definition}
\begin{document}
\begin{defnn}
Let $p$ and $q$ be positive natural numbers. Further,
let $S(k)$ be the set of permutations of the numbers $\{1,\ldots, k\}$.
A permutation $\tau\in S(p+q)$  is a $(p,q)$ \emph{shuffle}  if
\begin{eqnarray*}
\tau(1)< &\cdots& < \tau(p), \\
\tau(p+1)< &\cdots& < \tau(p+q).
\end{eqnarray*}
The set of all  $(p,q)$ shuffles is denoted by $S(p,q)$.
\end{defnn}

It is clear that $S(p,q)\subset S(p+q)$. Since a $(p,q)$ shuffle is
completely determined by how the $p$ first elements are mapped, the
cardinality of $S(p,q)$ is ${p+q \choose p}$.
The wedge product of a $p$-form and a $q$-form can be defined as a sum
over $(p,q)$ shuffles.
%%%%%
%%%%%
\end{document}
