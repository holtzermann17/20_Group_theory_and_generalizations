\documentclass[12pt]{article}
\usepackage{pmmeta}
\pmcanonicalname{AbelianGroupIsDivisibleIfAndOnlyIfItIsAnInjectiveObject}
\pmcreated{2013-03-22 18:48:15}
\pmmodified{2013-03-22 18:48:15}
\pmowner{joking}{16130}
\pmmodifier{joking}{16130}
\pmtitle{abelian group is divisible if and only if it is an injective object}
\pmrecord{8}{41605}
\pmprivacy{1}
\pmauthor{joking}{16130}
\pmtype{Theorem}
\pmcomment{trigger rebuild}
\pmclassification{msc}{20K99}

% this is the default PlanetMath preamble.  as your knowledge
% of TeX increases, you will probably want to edit this, but
% it should be fine as is for beginners.

% almost certainly you want these
\usepackage{amssymb}
\usepackage{amsmath}
\usepackage{amsfonts}

% used for TeXing text within eps files
%\usepackage{psfrag}
% need this for including graphics (\includegraphics)
%\usepackage{graphicx}
% for neatly defining theorems and propositions
%\usepackage{amsthm}
% making logically defined graphics
%%%\usepackage{xypic}

% there are many more packages, add them here as you need them

% define commands here

\begin{document}
\textbf{Proposition.} Abelian group $A$ is divisible if and only if $A$ is an injective object in the category of abelian groups.

\textit{Proof.} ,,$\Leftarrow$'' Assume that $A$ is not divisible, i.e. there exists $a\in A$ and $n\in\mathbb{N}$ such that the equation $nx=a$ has no solution in $A$. Let $B=<a>$ be a cyclic subgroup generated by $a$ and $i:B\to A$ the canonical inclusion. Now there are two possibilities: either $B$ is finite or infinite.

If $B$ is infinite, then let $H=\mathbb{Z}$ and let $f:B\to H$ be defined on generator by $f(a)=n$. Now $A$ is injective, thus there exists $h:H\to A$ such that $h\circ f=i$. Thus $$n\cdot h(1)=h(1)+\cdots +h(1)=h(1+\cdots +1)=h(n)=h(f(a))=i(a)=a.$$
Contradiction with definition of $n\in\mathbb{N}$ and $a\in A$.

If $B$ is finite, then let $k=|B|$ (note that $n$ does not divide $k$) and let $H=\mathbb{Z}_{n\cdot k}$. Furtheremore define $f:B\to H$ on generator by $f(a)=n$ (note that in this case $f$ is a well defined homomorphism). Again injectivity of $A$ implies existence of $h:H\to A$ such that $h\circ f=i$. Similarly we get contradiction:
$$n\cdot h(1)=h(1)+\cdots +h(1)=h(1+\cdots +1)=h(n)=h(f(a))=i(a)=a.$$
This completes first implication.

,,$\Rightarrow$'' This implication is proven \PMlinkname{here}{ExampleOfInjectiveModule}. $\square$

\textbf{Remark.} It is clear that in the category of abelian groups $\mathcal{AB}$, a group $A$ is projective if and only if $A$ is free. This is since $\mathcal{AB}$ is equivalent to the category of $\mathbb{Z}$-modules and projective modules are direct summands of free modules. Since $\mathbb{Z}$ is a principal ideal domain, then every submodule of a free module is free, thus projective $\mathbb{Z}$-modules are free.
%%%%%
%%%%%
\end{document}
