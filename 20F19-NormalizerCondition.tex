\documentclass[12pt]{article}
\usepackage{pmmeta}
\pmcanonicalname{NormalizerCondition}
\pmcreated{2013-03-22 16:14:41}
\pmmodified{2013-03-22 16:14:41}
\pmowner{yark}{2760}
\pmmodifier{yark}{2760}
\pmtitle{normalizer condition}
\pmrecord{6}{38348}
\pmprivacy{1}
\pmauthor{yark}{2760}
\pmtype{Definition}
\pmcomment{trigger rebuild}
\pmclassification{msc}{20F19}
\pmsynonym{normaliser condition}{NormalizerCondition}
\pmrelated{LocallyNilpotentGroup}
\pmdefines{N-group}

\usepackage{amssymb}
\usepackage{amsmath}
\usepackage{amsfonts}

\begin{document}
\PMlinkescapeword{satisfies}
\PMlinkescapeword{satisfy}

A group is said to satisfy the \emph{normalizer condition} if every proper subgroup is properly contained in its own normalizer.
That is, a group $G$ satisfies the normalizer condition if and only if
$H<N_G(H)$ for all $H<G$.
A group that satisfies the normalizer condition is sometimes called an \emph{N-group}.

Every nilpotent group is an N-group, and every N-group is locally nilpotent.
In particular, a finitely generated group is an N-group
if and only if it is nilpotent.

A group satisfies the normalizer condition if and only if all its subgroups are ascendant.
%%%%%
%%%%%
\end{document}
