\documentclass[12pt]{article}
\usepackage{pmmeta}
\pmcanonicalname{ProofOfBasicTheoremAboutOrderedGroups}
\pmcreated{2013-03-22 14:54:46}
\pmmodified{2013-03-22 14:54:46}
\pmowner{rspuzio}{6075}
\pmmodifier{rspuzio}{6075}
\pmtitle{proof of basic theorem about ordered groups}
\pmrecord{14}{36598}
\pmprivacy{1}
\pmauthor{rspuzio}{6075}
\pmtype{Proof}
\pmcomment{trigger rebuild}
\pmclassification{msc}{20F60}
\pmclassification{msc}{06A05}

% this is the default PlanetMath preamble.  as your knowledge
% of TeX increases, you will probably want to edit this, but
% it should be fine as is for beginners.

% almost certainly you want these
\usepackage{amssymb}
\usepackage{amsmath}
\usepackage{amsfonts}

% used for TeXing text within eps files
%\usepackage{psfrag}
% need this for including graphics (\includegraphics)
%\usepackage{graphicx}
% for neatly defining theorems and propositions
%\usepackage{amsthm}
% making logically defined graphics
%%%\usepackage{xypic}

% there are many more packages, add them here as you need them

% define commands here
\begin{document}
\PMlinkescapeword{Property}
\subsubsection*{Property 1:}

Consider $a b^{-1} \in G$.  Since $G$ can be written as a pairwise disjoint union, exactly one of the following conditions must hold:
 $$a b^{-1} \in S \qquad a b^{-1} = 1 \qquad a b^{-1} \in S^{-1}$$
By definition of the ordering relation, $a < b$ if the first condition holds.  If the second condition holds, then $a = b$.  If the third condition holds, then we must have $a b^{-1} = s^{-1}$ for some $s \in S$.  Taking inverses, this means that $b a^{-1} = s$, so $b < a$, or equivalently $a > b$.  Hence, one of the following three conditions must hold:
 $$a < b \qquad a = b \qquad b < a$$

\subsubsection*{Property 2:}

The hypotheses can be rewritten as
 $$a b^{-1} \in S \qquad b c^{-1} \in S$$
Multiplying, and remembering that $S$ is closed under multiplication,
 $$a c^{-1} = ( a b^{-1} )( b c^{-1} ) \in S.$$
In other words, $a < c$.

\subsubsection*{Property 3:}

Suppose that $a < b$, so $a b ^{-1} = s \in S$.   Then
 $$s = a b^{-1} = a 1 b^{-1} = a c c^{-1} b^{-1} = (ac) (bc)^{-1}$$
so $ac < bc$.

By the defining property of $S$, we have $c s c^{-1} \in S$.  Also,
 $$c s c^{-1} = c a b^{-1} c^{-1} = (ca) (cb)^{-1},$$
hence $(ca) (cb)^{-1} \in S$, so $ca < cb$

\subsubsection*{Property 4:}

By property 3, $a < b$ implies $ac < bc$ and likewise $c < d$ implies $bc < bd$.  Then, by property 2, we conclude $ac < bd$.

\subsubsection*{Property 5:}

By the hypothesis, $a b^{-1} = s \in S$.  By the defining property, $b^{-1} s b \in S$.  Since $b^{-1} s b = b^{-1} a$, we have $b^{-1} a \in S$.  In other words, $b^{-1} < a^{-1}$.

\subsubsection*{Property 6:}

By definition, $a < 1$ means that $a 1^{-1} \in S$.  Since $1^{-1} = 1$ and $a 1 = a$, this is equivalent to stating that $a \in S$.
%%%%%
%%%%%
\end{document}
