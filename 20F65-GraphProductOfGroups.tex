\documentclass[12pt]{article}
\usepackage{pmmeta}
\pmcanonicalname{GraphProductOfGroups}
\pmcreated{2013-03-22 16:10:36}
\pmmodified{2013-03-22 16:10:36}
\pmowner{mps}{409}
\pmmodifier{mps}{409}
\pmtitle{graph product of groups}
\pmrecord{8}{38263}
\pmprivacy{1}
\pmauthor{mps}{409}
\pmtype{Definition}
\pmcomment{trigger rebuild}
\pmclassification{msc}{20F65}
\pmdefines{graph product of groups}
\pmdefines{graph product}

\endmetadata

% this is the default PlanetMath preamble.  as your knowledge
% of TeX increases, you will probably want to edit this, but
% it should be fine as is for beginners.

% almost certainly you want these
\usepackage{amssymb}
\usepackage{amsmath}
\usepackage{amsfonts}

% used for TeXing text within eps files
%\usepackage{psfrag}
% need this for including graphics (\includegraphics)
%\usepackage{graphicx}
% for neatly defining theorems and propositions
%\usepackage{amsthm}
% making logically defined graphics
%%%\usepackage{xypic}

% there are many more packages, add them here as you need them
\usepackage{url}

% define commands here

\begin{document}
\PMlinkescapeword{collection}

Let $\Gamma$ be a finite undirected graph and let $\{G_v \colon v\in V(\Gamma)\}$ be a collection of groups associated with the vertices of $\Gamma$.  Then the \emph{graph product of the groups} $G_v$ is the group $G=F/R$, where 
$F$ is the free product of the $G_v$ and $R$ is generated by the relations that elements of $G_u$ commute with elements of $G_v$ whenever $u$ and $v$ are adjacent in $\Gamma$.

The free product and the direct product are the extreme examples of the graph product.  To obtain the free product, let $\Gamma$ be an anticlique, and to obtain the direct product, let $\Gamma$ be a clique.

\begin{thebibliography}{9}
\bibitem{G}
E.R. Green, \emph{Graph products of groups}, Doctoral thesis, The University of Leeds, 1990.
\bibitem{HM}
S. Hermiller and J. Meier, \emph{Algorithms and geometry for graph products of groups}, Journal of Algebra 117 (1995), 230--257.
\bibitem{LS}
M. Lohrey and G. S\'{e}nizergues, \emph{When is a graph product of groups virtually-free?}, to appear in Communications in Algebra.  2006 preprint available online at \url{http://inf.informatik.uni-stuttgart.de/fmi/ti/personen/Lohrey/05-Graphprod.pdf}.
\bibitem{BBP} R.Brown, M. Bullejos, and T. Porter,`Crossed complexes, free
crossed resolutions and graph products of groups', Proceedings
Workshop Korea 2000,  J. Mennicke, Moo Ha Woo (eds.) Recent
Advances in Group Theory, Heldermann Verlag  Research and
Exposition in Mathematics  27 (2002) 8--23. arXiv:math/0101220


\end{thebibliography}
%%%%%
%%%%%
\end{document}
