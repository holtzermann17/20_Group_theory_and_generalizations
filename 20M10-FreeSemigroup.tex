\documentclass[12pt]{article}
\usepackage{pmmeta}
\pmcanonicalname{FreeSemigroup}
\pmcreated{2013-03-22 16:11:41}
\pmmodified{2013-03-22 16:11:41}
\pmowner{yark}{2760}
\pmmodifier{yark}{2760}
\pmtitle{free semigroup}
\pmrecord{15}{38287}
\pmprivacy{1}
\pmauthor{yark}{2760}
\pmtype{Definition}
\pmcomment{trigger rebuild}
\pmclassification{msc}{20M10}
\pmclassification{msc}{20M05}
%\pmkeywords{semigroup}
\pmrelated{Word}
\pmdefines{word}
\pmdefines{empty word}
\pmdefines{free semigroup}
\pmdefines{free monoid}

\usepackage{amsfonts}
%%\usepackage{xypic} 

\newcommand{\gbra}[1]{\left\{ #1 \right\}}

\begin{document}
Let $X$ be a set.
We define the power of $X$ in a language-theoretical manner as
\[
  X^n=\gbra{x_1x_2\ldots x_n\mid x_j\in X
  \hbox{ for all }j\in\gbra{1,\ldots,n}}
\]
for all $n\in\mathbb{N}\setminus \gbra{0}$, and
\[
  X^0=\gbra{\varepsilon},
\]
where $\varepsilon\notin X$.
Note that the set $X$ is not necessarily an alphabet,
that is, it may be infinite; for example, we may choose $X=\mathbb{R}$.

We define the sets $X^+$ and $X^*$ as
\[
  X^+=\bigcup_{n\in\mathbb{N}\setminus\gbra{0}}   X^n
\]
and
\[
  X^*=\bigcup_{n\in\mathbb{N}}   X^n  =  X^+\cup\gbra{\varepsilon}.
\]
The elements of $X^*$ are called \emph{words} on $X$,
and $\varepsilon$ is called the \emph{empty word} on $X$.

We define the \emph{juxtaposition} of two words $v,w\in X^*$ as 
\[
  vw=v_1v_2\ldots v_n w_1w_2\ldots w_m,
\]
where $v=v_1v_2\ldots v_n$ and $w=w_1w_2\ldots w_m$,
with $v_i,w_j\in X$ for each $i$ and $j$.
It is easy to see that the juxtaposition is associative,
so if we equip $X^+$ and $X^*$ with it
we obtain respectively a semigroup and a monoid.
Moreover, $X^+$ is the \emph{free semigroup} on $X$ and $X^*$
is the \emph{free monoid} on $X$,
in the sense that they solve the following universal mapping problem:
given a semigroup $S$ (resp.\ a monoid $M$)
and a map $\Phi\colon X\to S$ (resp.\ $\Phi\colon X\to M$),
a semigroup homomorphism $\overline\Phi\colon X^+\to S$
(resp.\ a monoid homomorphism $\overline\Phi\colon X^*\to M$)
exists such that the following diagram commutes:
\[
  \xymatrix{
  & X \ar[r]^{\iota} \ar[d]_{\Phi} & X^+ \ar[dl]^{\overline{\Phi}} \\
  & S &
  }
\]
  (resp. 
\[
  \xymatrix{
  & X \ar[r]^{\iota} \ar[d]_{\Phi} & X^* \ar[dl]^{\overline{\Phi}} \\
  & M &
  }
\]
), where $\iota\colon X\to X^+$ (resp.\ $\iota\colon X\to X^*$)
is the inclusion map.
It is well known from universal algebra
that $X^+$ and $X^*$ are unique up to isomorphism.

\begin{thebibliography}{9}
\bibitem{b:howie}
 J.M. Howie,
 \it{Fundamentals of Semigroup Theory},
 Oxford University Press, Oxford, 1991.
\end{thebibliography}
%%%%%
%%%%%
\end{document}
