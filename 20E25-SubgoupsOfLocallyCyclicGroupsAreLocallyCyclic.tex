\documentclass[12pt]{article}
\usepackage{pmmeta}
\pmcanonicalname{SubgoupsOfLocallyCyclicGroupsAreLocallyCyclic}
\pmcreated{2013-03-22 17:14:46}
\pmmodified{2013-03-22 17:14:46}
\pmowner{rspuzio}{6075}
\pmmodifier{rspuzio}{6075}
\pmtitle{subgoups of locally cyclic groups are locally cyclic}
\pmrecord{12}{39577}
\pmprivacy{1}
\pmauthor{rspuzio}{6075}
\pmtype{Theorem}
\pmcomment{trigger rebuild}
\pmclassification{msc}{20E25}
\pmclassification{msc}{20K99}

% this is the default PlanetMath preamble.  as your knowledge
% of TeX increases, you will probably want to edit this, but
% it should be fine as is for beginners.

% almost certainly you want these
\usepackage{amssymb}
\usepackage{amsmath}
\usepackage{amsfonts}

% used for TeXing text within eps files
%\usepackage{psfrag}
% need this for including graphics (\includegraphics)
%\usepackage{graphicx}
% for neatly defining theorems and propositions
\usepackage{amsthm}
% making logically defined graphics
%%%\usepackage{xypic}

% there are many more packages, add them here as you need them

% define commands here
\newtheorem{thm}{Theorem}
\begin{document}
\begin{thm}
A group $G$ is locally cyclic iff every subgroup $H\le G$ is locally cyclic.
\end{thm}

\begin{proof}
% We use additive notation.  Let $G$ be a locally cyclic group, $H$ be a subgroup of $G$ and  $\alpha, \beta$ be elements of $H$.  Then there exists an element $\gamma$ of $G$ and integers $m,n$ such that $\alpha = n \gamma$ and $\beta = m \gamma$.  Let $r$ be the greatest common divisor of $m$ and $n$ and set $p=m/r$ and $q = n/r$ and $\delta = r \gamma$.  Then we have $\alpha = q \delta$ and $\beta = p \delta$.  Since $p$ and $q$ are relatively prime, there exist integers $s,t$ such that $ps + qt = 1$.  Hence, $t \alpha + s \beta = (qs + pt) \delta = \delta$, so $\delta \in H$.  Summarizing, we have shown that, for any elements $\alpha, \beta$ of $H$, there exists an element $\delta$ of $H$ and integers $p,q$ such that $\alpha = q \delta$ and $\beta = p \delta$.  This assertion is equivalent to stating that $H$ is locally cyclic.

Let $G$ be a locally cyclic group and $H$ a subgroup of $G$.  Let $S$ be a finite subset of $H$.  Then the group $\langle S\rangle$ generated by $S$ is a cyclic subgroup of $G$, by assumption.  Since every element $a$ of $\langle S\rangle$ is a product of elements or inverses of elements of $S$, and $S$ is a subset of group $H$, $a\in H$.  Hence $\langle S\rangle$ is a cyclic subgroup of $H$, so $H$ is locally cyclic.

Conversely, suppose for every subgroup of $G$ is locally cyclic.  Let $H$ be a subgroup generated by a finite subset of $G$.  Since $H$ is locally cyclic, and $H$ itself is finitely generated, $H$ is cyclic, and therefore $G$ is locally cyclic.
\end{proof}

%\begin{thm}
%Every finitely generated subset of a locally cyclic group is cyclic.
%\end{thm}

%\begin{proof}
%Let $H$ be a finitely generated subset of a locally cyclic group $G$.  We
%already know that $H$ is locally cyclic.  Since every finite subset of a
%localy cyclic group generates a cyclic group, it follows that $H$ is cyclic
%\end{proof}
%%%%%
%%%%%
\end{document}
