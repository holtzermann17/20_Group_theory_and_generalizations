\documentclass[12pt]{article}
\usepackage{pmmeta}
\pmcanonicalname{VirtuallyAbelianSubgroupTheorem}
\pmcreated{2013-03-22 18:58:42}
\pmmodified{2013-03-22 18:58:42}
\pmowner{juanman}{12619}
\pmmodifier{juanman}{12619}
\pmtitle{virtually abelian subgroup theorem}
\pmrecord{13}{41842}
\pmprivacy{1}
\pmauthor{juanman}{12619}
\pmtype{Theorem}
\pmcomment{trigger rebuild}
\pmclassification{msc}{20F99}
\pmclassification{msc}{20E99}
\pmclassification{msc}{20E07}
\pmsynonym{subgroup theorem}{VirtuallyAbelianSubgroupTheorem}
%\pmkeywords{finite index abelian subgroup}

\endmetadata

% this is the default PlanetMath preamble.  as your knowledge
% of TeX increases, you will probably want to edit this, but
% it should be fine as is for beginners.

% almost certainly you want these
\usepackage{amssymb}
\usepackage{amsmath}
\usepackage{amsfonts}

% used for TeXing text within eps files
%\usepackage{psfrag}
% need this for including graphics (\includegraphics)
%\usepackage{graphicx}
% for neatly defining theorems and propositions
%\usepackage{amsthm}
% making logically defined graphics
%%%\usepackage{xypic}

% there are many more packages, add them here as you need them

% define commands here

\begin{document}
Let us suppose that $G$ is virtually abelian and $H$ is an abelian subgroup of $G$ with a the finite right coset
partition
$$G=He\sqcup Hx_2\sqcup...\sqcup Hx_q,\eqno{(*)}$$
so if $K$ is any other subgroup in $G$ we are going to prove:

{\sl $K$ is also virtually abelian}

\begin{quote}
{\bf Proof}: From $(*)$ above we have
$$K=K\cap G=K\cap(He\sqcup Hx_2\sqcup...\sqcup Hx_q),$$
$$=(K\cap H)\sqcup(K\cap Hx_2)\sqcup...\sqcup(K\cap Hx_q).\eqno{(**)}$$

Here we consider the two cases:\\
1) $x_i\in K$\\
2) $x_j\notin K$

In the first case $K=Kx_i$, and then $K\cap Hx_i=Kx_i\cap Hx_i=(K\cap H)x_i$.
In the second, find $y_j\in K\cap Hx_j$ hence $K\cap Hx_j=Ky_j\cap Hy_j=(K\cap H)y_j$

So, in the equation $(**)$ above we can replace (reordering subindexation perhaps) to get
$$K=\underbrace{(K\cap H)\sqcup(K\cap H)x_2\sqcup...\sqcup(K\cap H)x_s}_{1)}
\sqcup\underbrace{(K\cap H)y_{s+1}\sqcup...\sqcup(K\cap H)y_q}_{2)}$$
relation which shows that the index $[K:K\cap H]\le [G:H]$. 

It could be $<$ since it is posible that $K\cap Hx_r=\emptyset$ for some indexes $r$
$\Box$
\end{quote}


%%%%%
%%%%%
\end{document}
