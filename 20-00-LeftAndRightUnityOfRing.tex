\documentclass[12pt]{article}
\usepackage{pmmeta}
\pmcanonicalname{LeftAndRightUnityOfRing}
\pmcreated{2013-03-22 15:10:54}
\pmmodified{2013-03-22 15:10:54}
\pmowner{rspuzio}{6075}
\pmmodifier{rspuzio}{6075}
\pmtitle{left and right unity of ring}
\pmrecord{6}{36936}
\pmprivacy{1}
\pmauthor{rspuzio}{6075}
\pmtype{Definition}
\pmcomment{trigger rebuild}
\pmclassification{msc}{20-00}
\pmclassification{msc}{16-00}
\pmrelated{InversesInRings}
\pmdefines{left unity}
\pmdefines{right unity}

\endmetadata

% this is the default PlanetMath preamble.  as your knowledge
% of TeX increases, you will probably want to edit this, but
% it should be fine as is for beginners.

% almost certainly you want these
\usepackage{amssymb}
\usepackage{amsmath}
\usepackage{amsfonts}

% used for TeXing text within eps files
%\usepackage{psfrag}
% need this for including graphics (\includegraphics)
%\usepackage{graphicx}
% for neatly defining theorems and propositions
 \usepackage{amsthm}
% making logically defined graphics
%%%\usepackage{xypic}

% there are many more packages, add them here as you need them

% define commands here

\theoremstyle{definition}
\newtheorem*{thmplain}{Theorem}
\begin{document}
If a ring \, $(R,\,+,\,\cdot)$\, \PMlinkescapetext{contains a multiplicative} left identity element $e$, i.e. if
                $$e\cdot a = a \quad \forall a,$$
then $e$ is called the {\em left unity} of $R$.

If a ring $R$ \PMlinkescapetext{contains a multiplicative} right identity element $e'$, i.e. if
                $$a\cdot e' = a \quad \forall a,$$
then $e'$ is called the {\em right unity} of $R$.

A ring may have several left or right unities (see e.g. the Klein four-ring).

If a ring $R$ has both a left unity $e$ and a right unity $e'$, then they must coincide, since
                  $$e' = e\cdot e' = e.$$
This situation means that every right unity equals to $e$, likewise every left unity. \,Then we speak simply of a unity of the ring.
%%%%%
%%%%%
\end{document}
