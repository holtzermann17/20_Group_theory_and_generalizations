\documentclass[12pt]{article}
\usepackage{pmmeta}
\pmcanonicalname{UnityOfSubring}
\pmcreated{2013-03-22 14:49:40}
\pmmodified{2013-03-22 14:49:40}
\pmowner{pahio}{2872}
\pmmodifier{pahio}{2872}
\pmtitle{unity of subring}
\pmrecord{7}{36492}
\pmprivacy{1}
\pmauthor{pahio}{2872}
\pmtype{Theorem}
\pmcomment{trigger rebuild}
\pmclassification{msc}{20-00}
\pmclassification{msc}{16-00}
\pmclassification{msc}{13-00}
\pmrelated{UnitiesOfRingAndSubring}
\pmrelated{CornerOfARing}

\endmetadata

% this is the default PlanetMath preamble.  as your knowledge
% of TeX increases, you will probably want to edit this, but
% it should be fine as is for beginners.

% almost certainly you want these
\usepackage{amssymb}
\usepackage{amsmath}
\usepackage{amsfonts}

% used for TeXing text within eps files
%\usepackage{psfrag}
% need this for including graphics (\includegraphics)
%\usepackage{graphicx}
% for neatly defining theorems and propositions
 \usepackage{amsthm}
% making logically defined graphics
%%%\usepackage{xypic}

% there are many more packages, add them here as you need them

% define commands here
\theoremstyle{definition}
\newtheorem*{thmplain}{Theorem}
\begin{document}
\begin{thmplain}
\, \,Let $S$ be a proper subring of the ring $R$. \,If $S$ has a non-zero unity $u$ which is not unity of $R$, then $u$ is a zero divisor of $R$.
\end{thmplain}

{\em Proof.} \,Because $u$ is not unity of $R$, there exists an element $r$ of $R$ such that \,$ru \neq r$. \,Then we have \,$(ru)u = r(uu) = ru$, which implies that \,$0 = (ru)u-ru = (ru-r)\cdot u$. \,Since neither \,$ru-r$\, nor \,$u$\, is 0, the element \,$u$\, is a zero divisor in $R$.
%%%%%
%%%%%
\end{document}
