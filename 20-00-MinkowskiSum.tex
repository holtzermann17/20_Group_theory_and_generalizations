\documentclass[12pt]{article}
\usepackage{pmmeta}
\pmcanonicalname{MinkowskiSum}
\pmcreated{2013-03-22 15:16:22}
\pmmodified{2013-03-22 15:16:22}
\pmowner{matte}{1858}
\pmmodifier{matte}{1858}
\pmtitle{Minkowski sum}
\pmrecord{4}{37060}
\pmprivacy{1}
\pmauthor{matte}{1858}
\pmtype{Definition}
\pmcomment{trigger rebuild}
\pmclassification{msc}{20-00}
\pmclassification{msc}{15-00}
\pmclassification{msc}{13-00}
\pmclassification{msc}{16-00}
\pmrelated{VectorSpace}
\pmrelated{Sumset}

\endmetadata

% this is the default PlanetMath preamble.  as your knowledge
% of TeX increases, you will probably want to edit this, but
% it should be fine as is for beginners.

% almost certainly you want these
\usepackage{amssymb}
\usepackage{amsmath}
\usepackage{amsfonts}
\usepackage{amsthm}

\usepackage{mathrsfs}

% used for TeXing text within eps files
%\usepackage{psfrag}
% need this for including graphics (\includegraphics)
%\usepackage{graphicx}
% for neatly defining theorems and propositions
%
% making logically defined graphics
%%%\usepackage{xypic}

% there are many more packages, add them here as you need them

% define commands here

\newcommand{\sR}[0]{\mathbb{R}}
\newcommand{\sC}[0]{\mathbb{C}}
\newcommand{\sN}[0]{\mathbb{N}}
\newcommand{\sZ}[0]{\mathbb{Z}}

 \usepackage{bbm}
 \newcommand{\Z}{\mathbbmss{Z}}
 \newcommand{\C}{\mathbbmss{C}}
 \newcommand{\F}{\mathbbmss{F}}
 \newcommand{\R}{\mathbbmss{R}}
 \newcommand{\Q}{\mathbbmss{Q}}



\newcommand*{\norm}[1]{\lVert #1 \rVert}
\newcommand*{\abs}[1]{| #1 |}



\newtheorem{thm}{Theorem}
\newtheorem{defn}{Definition}
\newtheorem{prop}{Proposition}
\newtheorem{lemma}{Lemma}
\newtheorem{cor}{Corollary}
\begin{document}
{\bf Definition}
Suppose $A$ and $B$ are sets in a vector space $V$ over a field $K$, 
and suppose $\lambda \in K$. Then
\begin{eqnarray*}
A+B&=&\{a+b \mid a\in A, b\in B\},\\
A-B&=&\{a-b \mid a\in A, b\in B\},\\
\lambda A&=& \{\lambda a \mid a\in A\},\\
-A &=& (-1) A.
\end{eqnarray*}
The set $A+B$ is called the \emph{Minkowski sum} of $A$ and $B$. 
If either $A$ or $B$ is a single point (a singleton), say $B=\{x\}$, then we 
write $A+x$ instead of $A+\{x\}$. Similarly we define $A-x$, $x-A$ and $x+A$.

\subsubsection*{Properties}
Suppose $A$,$B$, $V$, and $\lambda$ are as above. Then 
\begin{itemize}
\item $A+B=B+A$
\item $\lambda(A+B)=\lambda A+\lambda B$
\item $2A\subseteq A+A$, $3A\subseteq A+A+A$, etc, 
but in general, $A+A\neq 2A$. 
(Consider $A=\{(0,0),(0,1)\}$ in $\sR^2$.)
\end{itemize}
%%%%%
%%%%%
\end{document}
