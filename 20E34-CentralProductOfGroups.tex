\documentclass[12pt]{article}
\usepackage{pmmeta}
\pmcanonicalname{CentralProductOfGroups}
\pmcreated{2013-03-22 18:49:45}
\pmmodified{2013-03-22 18:49:45}
\pmowner{Algeboy}{12884}
\pmmodifier{Algeboy}{12884}
\pmtitle{central product of groups}
\pmrecord{6}{41633}
\pmprivacy{1}
\pmauthor{Algeboy}{12884}
\pmtype{Definition}
\pmcomment{trigger rebuild}
\pmclassification{msc}{20E34}
\pmsynonym{central decomposition}{CentralProductOfGroups}
\pmdefines{central decomposition}
\pmdefines{central product}

\endmetadata

\usepackage{latexsym}
\usepackage{amssymb}
\usepackage{amsmath}
\usepackage{amsfonts}
\usepackage{amsthm}
\usepackage{enumerate}
%%\usepackage{xypic}

%-----------------------------------------------------

%       Standard theoremlike environments.

%       Stolen directly from AMSLaTeX sample

%-----------------------------------------------------

%% \theoremstyle{plain} %% This is the default

\newtheorem{thm}{Theorem}

\newtheorem{coro}[thm]{Corollary}

\newtheorem{lem}[thm]{Lemma}

\newtheorem{lemma}[thm]{Lemma}

\newtheorem{prop}[thm]{Proposition}

\newtheorem{conjecture}[thm]{Conjecture}

\newtheorem{conj}[thm]{Conjecture}

\newtheorem{defn}[thm]{Definition}

\newtheorem{remark}[thm]{Remark}

\newtheorem{ex}[thm]{Example}



%\countstyle[equation]{thm}



%--------------------------------------------------

%       Item references.

%--------------------------------------------------


\newcommand{\exref}[1]{Example-\ref{#1}}

\newcommand{\thmref}[1]{Theorem-\ref{#1}}

\newcommand{\defref}[1]{Definition-\ref{#1}}

\newcommand{\eqnref}[1]{(\ref{#1})}

\newcommand{\secref}[1]{Section-\ref{#1}}

\newcommand{\lemref}[1]{Lemma-\ref{#1}}

\newcommand{\propref}[1]{Prop\-o\-si\-tion-\ref{#1}}

\newcommand{\corref}[1]{Cor\-ol\-lary-\ref{#1}}

\newcommand{\figref}[1]{Fig\-ure-\ref{#1}}

\newcommand{\conjref}[1]{Conjecture-\ref{#1}}


% Normal subgroup or equal.

\providecommand{\normaleq}{\unlhd}

% Normal subgroup.

\providecommand{\normal}{\lhd}

\providecommand{\rnormal}{\rhd}
% Divides, does not divide.

\providecommand{\divides}{\mid}

\providecommand{\ndivides}{\nmid}


\providecommand{\union}{\cup}

\providecommand{\bigunion}{\bigcup}

\providecommand{\intersect}{\cap}

\providecommand{\bigintersect}{\bigcap}










\begin{document}
\section{Definitions}

A \emph{central decomposition} is a set $\mathcal{H}$ of subgroups of a group $G$ where
\begin{enumerate}
\item for $\mathcal{J}\subseteq\mathcal{H}$, $G=\langle\mathcal{J}\rangle$  
if, and only if, $\mathcal{J}=\mathcal{H}$, and
\item $[H,\langle\mathcal{H}-\{H\}\rangle]=1$ for all $H\in\mathcal{H}$.
\end{enumerate}
A group $G$ is \emph{centrally indecomposable} if its only central decomposition is $\{G\}$.
A central decomposition is \emph{fully refined} if its members are centrally indecomposable.

\begin{remark}
Condition 1 is often relaxed to $G=\langle\mathcal{H}\rangle$ but this
has the negative affect of allowing, for example, $\mathbb{R}^2$ to have the central decomposition
such sets as $\{\langle (1,0)\rangle,\langle(0,1)\rangle,\langle (1,1)\rangle,0,\mathbb{R}^2\}$ and
in general a decomposition of any possible size.  By impossing 1, we then restrict
the central decompositions of $\mathbb{R}^2$ to direct decompositions.  Furthermore, with condition 1,
the meaning of indecomposable is easily had.
\end{remark}

A \emph{central product} is a group $G=\left(\prod_{H\in\mathcal{H}} H\right)\large/ N$
where $N$ is a normal subgroup of $\prod_{H\in\mathcal{H}} H$ and $H\intersect N=1$ for all $H\in\mathcal{H}$.

\begin{prop}
Every finite central decomposition $\mathcal{H}$ is a central product of the members in $\mathcal{H}$.
\end{prop}
\begin{proof}
Suppose that $\mathcal{H}$ is a a finite central decomposition of $G$.  Then
define $\pi:\prod_{H\in\mathcal{H}} H\to G$ by $(x_H: H\in\mathcal{H})\mapsto \prod_{H\in\mathcal{H}} x_H$.
Then $G=\left(\prod_{H\in\mathcal{H}} H\right)/\ker \pi$.  Furthermore, $H\intersect \ker \pi=1$ for 
each direct factor $H$ of $\prod_{K\in\mathcal{H}} K$.  Thus, $G$ is a central product of $\mathcal{H}$.
\end{proof}

\section{Examples}

\begin{enumerate}
\item
Every direct product is also a central product and so also every direct decomposition is a 
central decomposition.  The converse is generally false.
\item
Let $E=\left\{\begin{bmatrix} 1 & \alpha & \gamma\\ 0 & 1 & \beta \\ 0 & 0 & 1\end{bmatrix}:
\alpha,\beta,\gamma\in K\right\}$, for a field $K$.  Then $G$ is a centrally indecomposable group.
\item
If
$$F=\left\{\begin{bmatrix} 
1 & \alpha_1 & \dots & \alpha_n & \gamma\\ 
0 & 1 & 0 & & \beta_1\\
\vdots & &\ddots & 0 & \vdots\\
 &  &  & 1 & \beta_n \\ 
 &  & & & 1 \end{bmatrix}:
\alpha_i,\beta_i,\gamma\in K,1\leq i\leq n\right\}$$
and 
$$H_i = \{A\in F: \alpha_j=0=\beta_j \forall j\neq i, 1\leq j\leq n\}$$
then $\{H_1,\dots,H_n\}$ is a central decomposition of $G$.  Furthermore, each
$H_i$ is isomorphic to $E$ and so $\mathcal{H}$ is a fully refined central decomposition.
\item
If $D_8=\langle a,b | a^4,b^2,(ab)^2\rangle$ -- the dihedral group of order 8,
and $Q_8=\langle i,j | i^4, i^2=j^2, i^j=i^{-1}\rangle$ -- the quaternion group of order $8$, then
$D_8 \circ D_8 = D_8 \times D_8/\langle (a^2,a^2)\rangle$ is isomorphic to 
$Q_8\circ Q_8=(Q_8\times Q_8)/\langle (i^2,i^2)\rangle$; yet, $D_8$ and $Q_8$ are nonisomorphic and
centrally indecomposable.  In particular, central decompositions are not unique even up to automorphisms.
This is in contrast the well-known Krull-Remak-Schmidt theorem for direct products of groups.
\end{enumerate}



\section{History}

The name \emph{central product} appears to have been coined by Philip Hall \cite[Section 3.2]{Hall} 
though the principal concept of such a product appears in earlier work (e.g. \cite[Theorem II]{Neumann}).
Hall describes central products as ``...the group obtained from the direct product by identifying
the centres of the direct factors...''.   The modern definition clearly out grows this original version 
as now centers may be only partially identified.



\thispagestyle{plain}

\begin{thebibliography}{[]}

\bibitem{Hall}
P. Hall, \emph{Finite-by-nilpotent groups}, Proc. Camb. Phil. Soc., 52 (1956), 611-616.

\bibitem{Neumann}
B. H. Neumann, and H. Neumann, \emph{A remark on generalized free products}, J. London Math. Soc. 25 (1950), 202-204.
\end{thebibliography}
%%%%%
%%%%%
\end{document}
