\documentclass[12pt]{article}
\usepackage{pmmeta}
\pmcanonicalname{GroupCohomology}
\pmcreated{2013-03-22 13:50:07}
\pmmodified{2013-03-22 13:50:07}
\pmowner{alozano}{2414}
\pmmodifier{alozano}{2414}
\pmtitle{group cohomology}
\pmrecord{11}{34571}
\pmprivacy{1}
\pmauthor{alozano}{2414}
\pmtype{Definition}
\pmcomment{trigger rebuild}
\pmclassification{msc}{20J06}
\pmsynonym{cohomology}{GroupCohomology}
%\pmkeywords{cohomology}
%\pmkeywords{coboundary}
%\pmkeywords{cocycle}
\pmrelated{SelmerGroup}
\pmrelated{CohomologyGroupTheorem}
\pmrelated{ProofOfCohomologyGroupTheorem}
\pmrelated{OmegaSpectrum}
\pmrelated{NaturalEquivalenceOfC_GAndC_MCategories}
\pmdefines{group cohomology}
\pmdefines{coboundary}
\pmdefines{cocycle}
\pmdefines{crossed homomorphism}

% this is the default PlanetMath preamble.  as your knowledge
% of TeX increases, you will probably want to edit this, but
% it should be fine as is for beginners.

% almost certainly you want these
\usepackage{amssymb}
\usepackage{amsmath}
\usepackage{amsthm}
\usepackage{amsfonts}

% used for TeXing text within eps files
%\usepackage{psfrag}
% need this for including graphics (\includegraphics)
%\usepackage{graphicx}
% for neatly defining theorems and propositions
%\usepackage{amsthm}
% making logically defined graphics
%%%\usepackage{xypic}

% there are many more packages, add them here as you need them

% define commands here

\newtheorem{thm}{Theorem}
\newtheorem{defn}{Definition}
\newtheorem{prop}{Proposition}
\newtheorem{lemma}{Lemma}
\newtheorem{cor}{Corollary}
\begin{document}
Let $G$ be a group and let $M$ be a (left) $G$-module. The
$0^{th}$ \emph{cohomology group} of the $G$-module $M$ is
$$H^0(G,M)=\{m\in M: \forall \sigma \in G,\ \sigma m=m\}$$
which is the set of elements of $M$ which are $G$-invariant, also
denoted by $M^G$.

A map $\phi\colon G\to M$ is said to be a \emph{crossed
homomorphism} (or \emph{1-cocycle}) if
$$\phi(\alpha\beta)=\phi(\alpha)+\alpha\phi(\beta)$$
for all $\alpha,\beta \in G$. If we fix $m\in M$, the map
$\rho\colon G\to M$ defined by
$$\rho(\alpha)=\alpha m-m$$
is clearly a crossed homomorphism, said to be \emph{principal} (or
\emph{1-coboundary}). We define the following groups:
\begin{eqnarray}
\nonumber Z^1(G,M)&=&\{\phi: G\to M\colon \phi \text{ is a 1-cocycle}\}\\
\nonumber B^1(G,M)&=&\{\rho: G\to M\colon \rho \text{ is a
1-coboundary}\}
\end{eqnarray}
Finally, the $1^{st}$ \emph{cohomology group} of the $G$-module
$M$ is defined to be the quotient group:
$$H^1(G,M)=Z^1(G,M)/B^1(G,M)$$

The following proposition is very useful when trying to compute
cohomology groups:
\begin{prop}
Let $G$ be a group and let $A,B,C$ be $G$-modules related by an
exact sequence:
$$0\to A\to B\to C\to 0$$
Then there is a long exact sequence in cohomology:
$$0\to H^0(G,A)\to H^0(G,B)\to H^0(G,C)\to H^1(G,A)\to H^1(G,B)\to
H^1(G,C)\to \ldots$$
\end{prop}

In general, the cohomology groups $H^n(G,M)$ can be defined as
follows:

\begin{defn}
Define $C^0(G,M)=M$ and for $n\geq 1$ define the additive group:
$$C^n(G,M)=\{\phi\colon G^n \to M\}$$
The elements of $C^n(G,M)$ are called $n$-cochains. Also, for
$n\geq 0$ define the $n^{th}$ coboundary homomorphism $d_n\colon
C^n(G,M) \to C^{n+1}(G,M)$:
\begin{eqnarray}
\nonumber d_n(\phi)(g_1,...,g_{n+1})&=&g_1\cdot \phi(g_2,...,g_{n+1})\\
\nonumber &+&
\sum_{i=1}^n(-1)^i\phi(g_1,...,g_{i-1},g_ig_{i+1},g_{i+2},
...,g_{n+1})\\
\nonumber &+& (-1)^{n+1}\phi(g_1,...,g_n)
\end{eqnarray}
Let $Z^n(G,M)=\operatorname{ker} d_n$ for $n\geq 0$, the set of
$n$-cocyles. Also, let $B^0(G,M)=1$ and for $n\geq 1$ let
$B^n(G,M)=\operatorname{image}d_{n-1}$, the set of
$n$-coboundaries.\\

Finally we define the $n^{th}$-cohomology group of $G$ with
coefficients in $M$ to be $$H^n(G,M)=Z^n(G,M)/B^n(G,M)$$
\end{defn}

\begin{thebibliography}{9}
\bibitem{serre} J.P. Serre, {\em Galois Cohomology},
Springer-Verlag, New York.
\bibitem{milne} James Milne, {\em Elliptic Curves}.
\bibitem{silverman} Joseph H. Silverman, {\em The Arithmetic of Elliptic Curves}. Springer-Verlag, New York, 1986.
\end{thebibliography}
%%%%%
%%%%%
\end{document}
