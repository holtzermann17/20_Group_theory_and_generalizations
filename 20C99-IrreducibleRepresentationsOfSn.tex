\documentclass[12pt]{article}
\usepackage{pmmeta}
\pmcanonicalname{IrreducibleRepresentationsOfSn}
\pmcreated{2013-03-22 17:14:13}
\pmmodified{2013-03-22 17:14:13}
\pmowner{rm50}{10146}
\pmmodifier{rm50}{10146}
\pmtitle{irreducible representations of $S_n$}
\pmrecord{8}{39566}
\pmprivacy{1}
\pmauthor{rm50}{10146}
\pmtype{Result}
\pmcomment{trigger rebuild}
\pmclassification{msc}{20C99}

\endmetadata

% this is the default PlanetMath preamble.  as your knowledge
% of TeX increases, you will probably want to edit this, but
% it should be fine as is for beginners.

% almost certainly you want these
\usepackage{amssymb}
\usepackage{amsmath}
\usepackage{amsfonts}

% used for TeXing text within eps files
%\usepackage{psfrag}
% need this for including graphics (\includegraphics)
%\usepackage{graphicx}
% for neatly defining theorems and propositions
%\usepackage{amsthm}
% making logically defined graphics
\usepackage[all,web]{xypic}

% there are many more packages, add them here as you need them

% define commands here
\newcommand{\Complex}{\mathbb{C}}

\def\drawsqlat{%
\begin{xy}{
0;<1pc,0pc>:<0pc,1pc>::
\xylattice{0}{5}{0}{5}}
\end{xy}}
\def\drawsq{\ar@{-}c;c+(1,0)\ar@{-}c;c+(0,1)\ar@{-}c+(1,0);c+(1,1)\ar@{-}c+(0,1);c+(1,1)}
\begin{document}
\PMlinkescapeword{c}
This article describes the theory of complex representations of $S_n$ by Young diagrams, as developed by Frobenius, Schur, and Young. The situation for representations in nonzero characteristic is more complicated.

Recall the well-known result that the conjugacy class of any $\sigma\in S_n$ is determined by its cycle type. The number of different cycle structures is simply the number of partitions of $n$. Therefore the number of conjugacy classes of $S_n$, and thus the number of irreducible representations of $S_n$, is just the number of partitions of $n$.

For example, for $S_4$ the partitions, and a representative of each conjugacy class, are
\begin{center}
\begin{tabular}{c c}
$4$ & $(1~2~3~4)$\\
$3,1$ & $(1~2~3)$\\
$2,2$ & $(1~2)(3~4)$\\
$2,1,1$ & $(1~2)$\\
$1,1,1,1$ & $e$
\end{tabular}
\end{center}

The partitions can be represented visually using Young diagrams, and there is an algorithm for explicitly extracting the dimensions of the irreducible representations of $S_n$ from the Young diagrams. Of course, each diagram corresponds to a single irreducible representation. \PMlinkescapetext{Associate} with each square in a given Young diagram the number of squares directly to its right plus the number of squares directly below it, and add $1$ for the square itself. Multiply these numbers together for all squares in the diagram, and divide into $n!$. The result is the dimension of an irreducible representation of $S_n$.

Taking again $S_4$, we have
\begin{center}
\begin{tabular}{c c c}
$4$ &
\parbox{12pc}{
\begin{renewcommand}{\latticebody}{
\ifnum\latticeA=4 \ifnum\latticeB=1 \drawsq\fi\fi
\ifnum\latticeA=3 \ifnum\latticeB=1 \drawsq\fi\fi
\ifnum\latticeA=2 \ifnum\latticeB=1 \drawsq\fi\fi
\ifnum\latticeA=1 \ifnum\latticeB=1 \drawsq\fi\fi
}
\drawsqlat
\end{renewcommand}
}
\vspace{6pt}
&
$\frac{4!}{4\cdot 3\cdot 2\cdot 1} = 1$
\\
$3,1$ &
\parbox{12pc}{
\begin{renewcommand}{\latticebody}{
\ifnum\latticeA=3 \ifnum\latticeB=2 \drawsq\fi\fi
\ifnum\latticeA=2 \ifnum\latticeB=2 \drawsq\fi\fi
\ifnum\latticeA=1 \ifnum\latticeB=2 \drawsq\fi\fi
\ifnum\latticeA=1 \ifnum\latticeB=1 \drawsq\fi\fi
}
\drawsqlat
\end{renewcommand}
}
\vspace{6pt}
&
$\frac{4!}{4\cdot 2\cdot 1\cdot 1} = 3$
\\

$2,2$ &
\parbox{12pc}{
\begin{renewcommand}{\latticebody}{
\ifnum\latticeA=2 \ifnum\latticeB=2 \drawsq\fi\fi
\ifnum\latticeA=2 \ifnum\latticeB=1 \drawsq\fi\fi
\ifnum\latticeA=1 \ifnum\latticeB=2 \drawsq\fi\fi
\ifnum\latticeA=1 \ifnum\latticeB=1 \drawsq\fi\fi
}
\drawsqlat
\end{renewcommand}
}
\vspace{6pt}
&
$\frac{4!}{3\cdot 2\cdot 2\cdot 1} = 2$
\\

$2,1,1$ &
\parbox{12pc}{
\begin{renewcommand}{\latticebody}{
\ifnum\latticeA=1 \ifnum\latticeB=3 \drawsq\fi\fi
\ifnum\latticeA=1 \ifnum\latticeB=2 \drawsq\fi\fi
\ifnum\latticeA=1 \ifnum\latticeB=1 \drawsq\fi\fi
\ifnum\latticeA=2 \ifnum\latticeB=3 \drawsq\fi\fi
}
\drawsqlat
\end{renewcommand}
}
\vspace{6pt}
&
$\frac{4!}{4\cdot 2\cdot 1\cdot 1} = 3$
\\

$1,1,1,1$ &
\parbox{12pc}{
\begin{renewcommand}{\latticebody}{
\ifnum\latticeA=1 \ifnum\latticeB=4 \drawsq\fi\fi
\ifnum\latticeA=1 \ifnum\latticeB=3 \drawsq\fi\fi
\ifnum\latticeA=1 \ifnum\latticeB=2 \drawsq\fi\fi
\ifnum\latticeA=1 \ifnum\latticeB=1 \drawsq\fi\fi
}
\drawsqlat
\end{renewcommand}
}
\vspace{6pt}
&
$\frac{4!}{4\cdot 3\cdot 2\cdot 1} = 1$
\end{tabular}
\end{center}

$S_4$ has two $1$-dimensional irreducible representations - $\epsilon$ and sgn. One of the $3$-dimensional representations is the augmentation of the natural action of $S_4$ on $\Complex^4$, and the other is the tensor of that representation with sgn. The irreducible representation of dimension $2$ arises from the map $S_4\to S_4/V_4 \cong S_3$.
%%%%%
%%%%%
\end{document}
