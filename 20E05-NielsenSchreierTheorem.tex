\documentclass[12pt]{article}
\usepackage{pmmeta}
\pmcanonicalname{NielsenSchreierTheorem}
\pmcreated{2013-03-22 12:54:07}
\pmmodified{2013-03-22 12:54:07}
\pmowner{yark}{2760}
\pmmodifier{yark}{2760}
\pmtitle{Nielsen-Schreier theorem}
\pmrecord{9}{33251}
\pmprivacy{1}
\pmauthor{yark}{2760}
\pmtype{Theorem}
\pmcomment{trigger rebuild}
\pmclassification{msc}{20E05}

\endmetadata

\usepackage{amssymb}
\usepackage{amsmath}
\usepackage{amsfonts}

\begin{document}
\PMlinkescapeword{subgroup}
\PMlinkescapeword{subgroups}
\PMlinkescapeword{theorem}

Every \PMlinkname{subgroup}{Subgroup} of a free group is free.

This theorem is named after Jakob Nielsen, who in 1921 proved that every subgroup of a free group of \PMlinkname{finite rank}{FreeGroup} is free, and Otto Schreier, who proved the full result in 1927.
%%%%%
%%%%%
\end{document}
