\documentclass[12pt]{article}
\usepackage{pmmeta}
\pmcanonicalname{CmncongCmtimesCnWhenMNAreRelativelyPrime}
\pmcreated{2013-03-22 17:59:46}
\pmmodified{2013-03-22 17:59:46}
\pmowner{yesitis}{13730}
\pmmodifier{yesitis}{13730}
\pmtitle{$C_{mn}\cong C_m\times C_n$ when $m, n$ are relatively prime}
\pmrecord{8}{40508}
\pmprivacy{1}
\pmauthor{yesitis}{13730}
\pmtype{Proof}
\pmcomment{trigger rebuild}
\pmclassification{msc}{20A05}

\endmetadata

% this is the default PlanetMath preamble.  as your knowledge
% of TeX increases, you will probably want to edit this, but
% it should be fine as is for beginners.

% almost certainly you want these
\usepackage{amssymb}
\usepackage{amsmath}
\usepackage{amsfonts}

% used for TeXing text within eps files
%\usepackage{psfrag}
% need this for including graphics (\includegraphics)
%\usepackage{graphicx}
% for neatly defining theorems and propositions
%\usepackage{amsthm}
% making logically defined graphics
%%%\usepackage{xypic}

% there are many more packages, add them here as you need them

% define commands here

\begin{document}
We show that $C_{mn}$, gcd$(m, n)=1$, is isomorphic to $C_m\times C_n$, where $C_r$ denotes the cyclic group of order $r$ for any positive integer $r$.

Let $C_m=\langle x\rangle$ and $C_n=\langle y\rangle$. Then the external direct product $C_m\times C_n$ consists of elements $(x^i, y^j)$, where $0\leq i\leq m-1$ and $0\leq j\leq n-1$.

Next, we show that the group $C_m\times C_n$ is cyclic. We do so by showing that it is generated by an element, namely $(x, y)$:
if $(x, y)$ generates $C_m\times C_n$, then for each $(x^i, y^j)\in C_m\times C_n$, we must have $(x^i, y^j)=(x, y)^k$ for some $k\in\{0, 1, 2, \ldots, mn-1\}$. Such $k$, if exists, would satisfy
\begin{eqnarray*}
k &\equiv& i\;(mod\;m) \\
k &\equiv& j\;(mod\;n).
\end{eqnarray*}
Indeed, by the Chinese Remainder Theorem, such $k$ exists and is unique modulo $mn$. (Here is where the relative primality of $m, n$ comes into play.) Thus, $C_m\times C_n$ is generated by $(x, y)$, so it is cyclic.

The order of $C_m\times C_n$ is $mn$, so is the order of $C_{mn}$. Since cyclic groups of the same order are isomorphic, we finally have $C_{mn}\cong C_m\times C_n$.
%%%%%
%%%%%
\end{document}
