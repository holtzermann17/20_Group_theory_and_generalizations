\documentclass[12pt]{article}
\usepackage{pmmeta}
\pmcanonicalname{Homeotopy}
\pmcreated{2013-03-22 15:41:54}
\pmmodified{2013-03-22 15:41:54}
\pmowner{juanman}{12619}
\pmmodifier{juanman}{12619}
\pmtitle{homeotopy}
\pmrecord{17}{37642}
\pmprivacy{1}
\pmauthor{juanman}{12619}
\pmtype{Definition}
\pmcomment{trigger rebuild}
\pmclassification{msc}{20F38}
\pmsynonym{mapping class group}{Homeotopy}
%\pmkeywords{homotopy group functor}
\pmrelated{isotopy}
\pmrelated{group}
\pmrelated{homeomorphism}
\pmrelated{Group}
\pmrelated{Isotopy}
\pmrelated{Homeomorphism}

% this is the default PlanetMath preamble.  as your knowledge
% of TeX increases, you will probably want to edit this, but
% it should be fine as is for beginners.

% almost certainly you want these
\usepackage{amssymb}
\usepackage{amsmath}
\usepackage{amsfonts}

% used for TeXing text within eps files
%\usepackage{psfrag}
% need this for including graphics (\includegraphics)
%\usepackage{graphicx}
% for neatly defining theorems and propositions
%\usepackage{amsthm}
% making logically defined graphics
%%%\usepackage{xypic}

% there are many more packages, add them here as you need them

% define commands here
\begin{document}
Let $X$ be a topological Hausdorff space. Let ${\rm Homeo}(X)$ be the group of homeomorphisms $X\to X$, which can be also turn into a topological space by means of the compact-open topology. And let $\pi_k$ be the k-th homotopy group functor. 

Then the {\bf k-th homeotopy} is defined as:
$${\cal{H}}_k(X)=\pi_k({\rm Homeo}(X))$$
that is, the group of homotopy classes of maps $S^k\to {\rm Homeo}(X)$.
Which is different from $\pi_k(X)$, the group of homotopy classes of maps $S^k\to X$.

One important result for any low dimensional topologist is that for a surface $F$
$${\cal{H}}_0(F)={\rm Out}(\pi_1(F))$$
which is the $F$'s extended mapping class group.

{\bf Reference}

G.S. McCarty, {\it Homeotopy groups}, Trans. A.M.S. 106(1963)293-304.
%%%%%
%%%%%
\end{document}
