\documentclass[12pt]{article}
\usepackage{pmmeta}
\pmcanonicalname{PrimitivePermutationGroup}
\pmcreated{2013-03-22 14:00:49}
\pmmodified{2013-03-22 14:00:49}
\pmowner{Thomas Heye}{1234}
\pmmodifier{Thomas Heye}{1234}
\pmtitle{primitive permutation group}
\pmrecord{20}{34913}
\pmprivacy{1}
\pmauthor{Thomas Heye}{1234}
\pmtype{Definition}
\pmcomment{trigger rebuild}
\pmclassification{msc}{20B15}

\usepackage{amssymb}
\usepackage{amsmath}
\usepackage{amsfonts}

\DeclareMathOperator{\Stab}{Stab}
\begin{document}
\PMlinkescapeword{opposite}
\PMlinkescapeword{vertices}

Let $X$ be a set, and $G$ a transitive permutation group on $X$.
Then $G$ is said to be a \emph{primitive permutation group} if it has no nontrivial \PMlinkname{blocks}{BlockSystem}.

For example, the symmetric group $S_4$ is a primitive permutation group on $\{1,2,3,4\}$.

Note that $D_8$ is not a primitive permutation group on the vertices of a square, because the pairs of opposite points form a nontrivial block.

It can be shown that a transitive permutation group $G$ on a set $X$ is primitive if and only if the stabilizer $\Stab_G(x)$ is a maximal subgroup of $G$ for all $x\in X$.
%%%%%
%%%%%
\end{document}
