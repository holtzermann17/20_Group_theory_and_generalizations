\documentclass[12pt]{article}
\usepackage{pmmeta}
\pmcanonicalname{ExampleOfMultiplyTransitive}
\pmcreated{2013-03-22 17:21:56}
\pmmodified{2013-03-22 17:21:56}
\pmowner{Algeboy}{12884}
\pmmodifier{Algeboy}{12884}
\pmtitle{example of multiply transitive}
\pmrecord{4}{39726}
\pmprivacy{1}
\pmauthor{Algeboy}{12884}
\pmtype{Example}
\pmcomment{trigger rebuild}
\pmclassification{msc}{20B20}

\usepackage{latexsym}
\usepackage{amssymb}
\usepackage{amsmath}
\usepackage{amsfonts}
\usepackage{amsthm}

%%\usepackage{xypic}

%-----------------------------------------------------

%       Standard theoremlike environments.

%       Stolen directly from AMSLaTeX sample

%-----------------------------------------------------

%% \theoremstyle{plain} %% This is the default

\newtheorem{thm}{Theorem}

\newtheorem{coro}[thm]{Corollary}

\newtheorem{lem}[thm]{Lemma}

\newtheorem{lemma}[thm]{Lemma}

\newtheorem{prop}[thm]{Proposition}

\newtheorem{conjecture}[thm]{Conjecture}

\newtheorem{conj}[thm]{Conjecture}

\newtheorem{defn}[thm]{Definition}

\newtheorem{remark}[thm]{Remark}

\newtheorem{ex}[thm]{Example}



%\countstyle[equation]{thm}



%--------------------------------------------------

%       Item references.

%--------------------------------------------------


\newcommand{\exref}[1]{Example-\ref{#1}}

\newcommand{\thmref}[1]{Theorem-\ref{#1}}

\newcommand{\defref}[1]{Definition-\ref{#1}}

\newcommand{\eqnref}[1]{(\ref{#1})}

\newcommand{\secref}[1]{Section-\ref{#1}}

\newcommand{\lemref}[1]{Lemma-\ref{#1}}

\newcommand{\propref}[1]{Prop\-o\-si\-tion-\ref{#1}}

\newcommand{\corref}[1]{Cor\-ol\-lary-\ref{#1}}

\newcommand{\figref}[1]{Fig\-ure-\ref{#1}}

\newcommand{\conjref}[1]{Conjecture-\ref{#1}}


% Normal subgroup or equal.

\providecommand{\normaleq}{\unlhd}

% Normal subgroup.

\providecommand{\normal}{\lhd}

\providecommand{\rnormal}{\rhd}
% Divides, does not divide.

\providecommand{\divides}{\mid}

\providecommand{\ndivides}{\nmid}


\providecommand{\union}{\cup}

\providecommand{\bigunion}{\bigcup}

\providecommand{\intersect}{\cap}

\providecommand{\bigintersect}{\bigcap}










\begin{document}
\begin{thm}
\begin{enumerate}
\item The general linear group $GL(V)$ acts transitively on the set of points (1-dimensional subspaces) in the
projective geometry $PG(V)$.
\item $PGL(V)$ is doubly transitive on the set of all of points in $PG(V)$.
\item $PGL(V)$ is not 3-transitive on the set of all points in $PG(V)$ if $\dim V\neq 2$.
\end{enumerate}
\end{thm}
\begin{proof}
Evidently 2 implies 1.  So suppose we have pairs of distinct points $(P,Q)$ and $(R,S)$.  Then take
$P=\langle x\rangle$, $Q=\langle y\rangle$, $R=\langle z\rangle$ and $S=\langle w\rangle$.
As $P\neq Q$, $x$ and $y$ are linearly independent, just as $z$ and $w$ are.  Therefore extending
$\{x,y\}$ to a basis $B$ and $\{z,w\}$ to a basis $C$, we know there is a linear transformation
$f\in GL(V)$ taking $B$ to $C$ -- consider the change of basis matrix.  Therefore $GL(V)$ is 
2-transitive.

Now suppose $\dim V\geq 2$.  Then there exists a linearly indepedent set $\{x,y,z\}$ which 
gives three distinct non-collinear points $(P,Q,R)$, $P=\langle x\rangle$, $Q=\langle y\rangle$ 
and $R=\langle z\rangle$.   But then we also have three collinear points
$(P,Q,S)$ where $S=\langle x+y\rangle$.  As $GL(V)$ prevserves the geometry of $PG(V)$,
we cannot have a map in $GL(V)$ send $(P,Q,R)$ to $(P,Q,S)$.
\end{proof}

Note that the action of $GL(V)$ on $PG(V)$ is not faithful so we use instead $PGL(V)$.
%%%%%
%%%%%
\end{document}
