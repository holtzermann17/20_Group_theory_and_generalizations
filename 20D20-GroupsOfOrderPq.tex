\documentclass[12pt]{article}
\usepackage{pmmeta}
\pmcanonicalname{GroupsOfOrderPq}
\pmcreated{2013-03-22 12:51:05}
\pmmodified{2013-03-22 12:51:05}
\pmowner{yark}{2760}
\pmmodifier{yark}{2760}
\pmtitle{groups of order pq}
\pmrecord{22}{33183}
\pmprivacy{1}
\pmauthor{yark}{2760}
\pmtype{Example}
\pmcomment{trigger rebuild}
\pmclassification{msc}{20D20}
\pmrelated{SylowTheorems}
\pmrelated{SemidirectProductOfGroups}

\endmetadata

\usepackage{amssymb}
\usepackage{amsmath}
\usepackage{amsfonts}

\DeclareMathOperator{\Aut}{Aut}
\def\Integer{\mathbb{Z}}
\def\inn#1{{\langle #1\rangle}}
\def\set#1{{\{#1\}}}
\begin{document}
\PMlinkescapeword{class}
\PMlinkescapeword{classes}
\PMlinkescapeword{divide}
\PMlinkescapeword{divides}
\PMlinkescapeword{prime}
\PMlinkescapeword{primes}
\PMlinkescapeword{subgroup}
\PMlinkescapeword{subgroups}

We can use Sylow's theorems to examine a group $G$ of order $pq$, where $p$ and $q$ are \PMlinkname{primes}{Prime} and $p<q$.

Let $n_p$ and $n_q$ denote, respectively, the number of Sylow $p$-subgroups and Sylow $q$-subgroups of $G$.

Sylow's theorems tell us that 
$n_q=1+kq$ for some integer $k$
and $n_q$ divides $pq$.
But $p$ and $q$ are prime and $p<q$, so this implies that $n_q=1$.
So there is exactly one Sylow $q$-subgroup, which is therefore normal (indeed, fully invariant) in $G$.

Denoting the Sylow $q$-subgroup by $Q$, and letting $P$ be a Sylow $p$-subgroup, then $Q\cap P=\{1\}$ and $QP=G$, so $G$ is a semidirect product of $Q$ and $P$. In particular, if there is only one Sylow $p$-subgroup, then $G$ is a direct product of $Q$ and $P$, and is therefore cyclic.

Given $G=Q \rtimes P$, it remains to determine the action of $P$ on $Q$ by conjugation. There are two cases:

Case 1: If $p$ does not divide $q-1$, then since $n_p=1+mp$ cannot equal $q$ we must have $n_p=1$, and so $P$ is a normal subgroup of $G$. This gives $G=C_p \times C_q$ a direct product, which is isomorphic to the cyclic group $C_{pq}$.

Case 2: If $p$ divides $q-1$,
then $\Aut(Q) \cong C_{q-1}$ has a unique \PMlinkname{subgroup}{Subgroup} $P'$ of order $p$,
where $P'=\set{x \mapsto x^i \mid i \in \Integer/q \Integer, i^p=1}$.
Let $a$ and $b$ be generators for $P$ and $Q$ respectively,
and suppose the action of $a$ on $Q$ by conjugation is $x \mapsto x^{i_0}$,
where $i_0 \neq 1$ in $\Integer/q \Integer$.
Then $G=\inn{a,b \mid a^p=b^q=1, aba^{-1}=b^{i_0}}$.
Choosing a different $i_0$
amounts to choosing a different generator $a$ for $P$,
and hence does not result in a new isomorphism class.
So there are exactly two isomorphism classes of groups of order $pq$.
%%%%%
%%%%%
\end{document}
