\documentclass[12pt]{article}
\usepackage{pmmeta}
\pmcanonicalname{PresentationOfInverseMonoidsAndInverseSemigroups}
\pmcreated{2013-03-22 16:11:01}
\pmmodified{2013-03-22 16:11:01}
\pmowner{Mazzu}{14365}
\pmmodifier{Mazzu}{14365}
\pmtitle{presentation of inverse monoids and  inverse semigroups}
\pmrecord{10}{38271}
\pmprivacy{1}
\pmauthor{Mazzu}{14365}
\pmtype{Definition}
\pmcomment{trigger rebuild}
\pmclassification{msc}{20M05}
\pmclassification{msc}{20M18}
\pmsynonym{presentation}{PresentationOfInverseMonoidsAndInverseSemigroups}
\pmsynonym{generators and relators}{PresentationOfInverseMonoidsAndInverseSemigroups}
%\pmkeywords{Inverse Semigroups}
%\pmkeywords{Word Problem}
%\pmkeywords{Isomorphism Problem}

\endmetadata

% this is the default PlanetMath preamble.  as your knowledge
% of TeX increases, you will probably want to edit this, but
% it should be fine as is for beginners.

% almost certainly you want these
\usepackage{amssymb}
\usepackage{amsmath}
\usepackage{amsfonts}

% used for TeXing text within eps files
%\usepackage{psfrag}
% need this for including graphics (\includegraphics)
%\usepackage{graphicx}
% for neatly defining theorems and propositions
%\usepackage{amsthm}
% making logically defined graphics
%%%\usepackage{xypic}

% there are many more packages, add them here as you need them

% define commands here

\begin{document}
\newcommand{\e}{\mathrm{e}}
\newcommand{\co}{\mathrm{c}}

\newcommand{\cbra}[1]{\left( #1 \right)}
\newcommand{\qbra}[1]{\left[ #1 \right]}
\newcommand{\gbra}[1]{\left\{ #1 \right\}}
\newcommand{\abra}[1]{\left\langle #1 \right\rangle}

\newcommand{\mipres}[2]{\mathrm{Inv}^1\abra{#1 | #2}}
\newcommand{\sipres}[2]{\mathrm{Inv}\abra{#1 | #2}}

\newcommand{\double}[1]{\cbra{#1\amalg #1^{-1}}}
\newcommand{\doubles}[1]{\cbra{#1\amalg #1^{-1}}^\ast}
\newcommand{\doublep}[1]{\cbra{#1\amalg #1^{-1}}^+}
\newcommand{\fim}{\mathrm{FIM}}
\newcommand{\fis}{\mathrm{FIS}}


Let $\doubles{X}$  be the free monoid  
with involution on $X$, and  $T\subseteq \doubles X\times \doubles X$ be a binary relation between words. We denote by $T^\e$ [resp. $T^\co$] the equivalence relation [resp. congruence] generated by $T$.

A \emph{presentation (for an inverse monoid)} is a couple $(X;T)$. We use this couple of objects to define an inverse monoid $\mipres{X}{T}$. Let $\rho_X$ be the Wagner congruence on $X$, we define the inverse monoid $\mipres{X}{T}$ \emph{presented} by $(X;T)$ as $$\mipres{X}{T}=\doubles{X}/(T\cup\rho_X)^\co.$$

In the previous dicussion, if we replace everywhere $\doubles X$ with $\doublep X$ we obtain a \emph{presentation (for an inverse semigroup)} $(X;T)$ and an inverse semigroup $\sipres{X}{T}$ \emph{presented} by $(X;T)$.

A trivial but important example is the Free Inverse Monoid [resp. Free Inverse Semigroup] on $X$, that is usually denoted by $\fim(X)$ [resp. $\fis(X)$] and is defined by $$\fim(X)=\mipres{X}{\varnothing}=\doubles{X}/\rho_X,\ \ \mbox{[resp. $\fis(X)=\sipres{X}{\varnothing}=\doublep{X}/\rho_X$]}.$$


\begin{thebibliography}{9}
\bibitem{b:petrich} N. Petrich, \emph{Inverse Semigroups}, Wiley, New York, 1984.
\bibitem{b:steph} J.B. Stephen, \emph{Presentation of inverse monoids}, J. Pure Appl. Algebra 63 (1990) 81-112.
\end{thebibliography}
%%%%%
%%%%%
\end{document}
