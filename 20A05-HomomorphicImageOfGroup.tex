\documentclass[12pt]{article}
\usepackage{pmmeta}
\pmcanonicalname{HomomorphicImageOfGroup}
\pmcreated{2013-03-22 18:56:27}
\pmmodified{2013-03-22 18:56:27}
\pmowner{pahio}{2872}
\pmmodifier{pahio}{2872}
\pmtitle{homomorphic image of group}
\pmrecord{14}{41796}
\pmprivacy{1}
\pmauthor{pahio}{2872}
\pmtype{Theorem}
\pmcomment{trigger rebuild}
\pmclassification{msc}{20A05}
\pmclassification{msc}{08A05}
\pmrelated{GroupHomomorphism}
\pmrelated{CorrespondenceBetweenNormalSubgroupsAndHomomorphicImages}

\endmetadata

% this is the default PlanetMath preamble.  as your knowledge
% of TeX increases, you will probably want to edit this, but
% it should be fine as is for beginners.

% almost certainly you want these
\usepackage{amssymb}
\usepackage{amsmath}
\usepackage{amsfonts}

% used for TeXing text within eps files
%\usepackage{psfrag}
% need this for including graphics (\includegraphics)
%\usepackage{graphicx}
% for neatly defining theorems and propositions
 \usepackage{amsthm}
% making logically defined graphics
%%%\usepackage{xypic}

% there are many more packages, add them here as you need them

% define commands here

\theoremstyle{definition}
\newtheorem*{thmplain}{Theorem}

\begin{document}
\textbf{Theorem.}\, The homomorphic image of a group is a group.\, More detailed, if $f$ is a homomorphism from the group\, $(G,\,\ast)$\, to the groupoid \,$(\Gamma,\,\star)$,\, then the groupoid\, $(f(G),\,\star)$\, also is a group.\, Especially, the isomorphic image of a group is a group.\\

\emph{Proof.}\, Let $\alpha,\,\beta,\,\gamma$ be arbitrary elements of the image $f(G)$ and $a,\,b,\,c$ some elements of $G$ such that\, $f(a) = \alpha,\, f(b) = \beta,\, f(c) = \gamma$.\, Then 
$$\alpha \star \beta \;=\; f(a) \star f(b) \;=\; f(a \ast b) \;\in\; f(G),$$
whence $f(G)$ is closed under ``$\star$'', and we, in fact, can speak of a groupoid\, $(f(G),\,\star)$.

Secondly, we can calculate
\begin{align*}
(\alpha\star\beta)\star\gamma &\;=\; (f(a) \star f(b)) \star f(c)\\
&\;=\; f(a \ast b) \star f(c)\\
&\;=\; f((a \ast b) \ast c)\\
&\;=\; f(a \ast(b \ast c))\\
&\;=\; f(a) \star f(b \ast c)\\
&\;=\; f(a) \star (f(b) \star f(c))\\
&\;=\; \alpha \star (\beta \star \gamma),
\end{align*}
whence the associativity is in \PMlinkescapetext{force} in the groupoid $(f(G),\,\star)$.

Let $e$ be the identity element of\, $(G,\,\ast)$\, and\, $f(e) = \varepsilon$.\, Then
$$\varepsilon \star \alpha \;=\; f(e) \star f(a) \;=\; f(e \ast a) \;=\; f(a) \;=\; \alpha,$$
$$\alpha \star \varepsilon \;=\; f(a) \star f(e) \;=\; f(a \ast e) \;=\; f(a) \;=\; \alpha,$$
and therefore $\varepsilon$ is an identity element in $f(G)$.

If\, $f(a^{-1}) = \alpha'$, then
$$\alpha \star \alpha' \;=\; f(a) \star f(a^{-1}) \;=\; f(a \ast a^{-1}) \;=\; f(e) \;=\; \varepsilon,$$
$$\alpha' \star \alpha \;=\; f(a^{-1}) \star f(a) \;=\; f(a^{-1} \ast a) \;=\; f(e) \;=\; \varepsilon.$$
Thus any element $\alpha$ of $f(G)$ has in $f(G)$ an inverse.

Accordingly,\, $(f(G),\, \star)$\, is a group.\\


\textbf{Remark 1.}\, If\, $(G,\,\ast)$\, is Abelian, the same is true for\, $(f(G),\, \star)$.

\textbf{Remark 2.}\, Analogically, one may prove that the homomorphic image of a ring is a ring.\\


\textbf{Example.}\, If we define the mapping $f$ from the group\, $(\mathbb{Z},\,+)$\, to the groupoid\, 
$(\mathbb{Z}_9,\,\cdot)$\, by
$$f(n) \;:=\; \langle4\rangle^n,$$
then $f$ is homomorphism:
$$f(m\!+\!n) \;=\; \langle4\rangle^{m+n} \;=\; \langle4\rangle^m\langle4\rangle^n \;=\; f(m)f(n).$$
The image $f(\mathbb{Z})$ consists of powers of the \PMlinkname{residue class}{Congruences} $\langle4\rangle$, which are
$$\langle4\rangle,\;\; \langle16\rangle = \langle7\rangle,\;\; \langle64\rangle = \langle1\rangle.$$
These apparently form the cyclic group of order 3.

%%%%%
%%%%%
\end{document}
