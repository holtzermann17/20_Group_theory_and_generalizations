\documentclass[12pt]{article}
\usepackage{pmmeta}
\pmcanonicalname{PolycyclicGroup}
\pmcreated{2013-03-22 14:40:50}
\pmmodified{2013-03-22 14:40:50}
\pmowner{yark}{2760}
\pmmodifier{yark}{2760}
\pmtitle{polycyclic group}
\pmrecord{14}{36285}
\pmprivacy{1}
\pmauthor{yark}{2760}
\pmtype{Definition}
\pmcomment{trigger rebuild}
\pmclassification{msc}{20F16}
\pmrelated{SupersolvableGroup}
\pmdefines{polycyclic}
\pmdefines{polycyclic series}
\pmdefines{Hirsch number}
\pmdefines{Hirsch length}

\usepackage{amssymb}
\usepackage{amsmath}
\usepackage{amsfonts}

\def\normal{\triangleleft}

% The below lines should work as the command
% \renewcommand{\bibname}{References}
% without creating havoc when rendering an entry in 
% the page-image mode.
\makeatletter
\@ifundefined{bibname}{}{\renewcommand{\bibname}{References}}
\makeatother
\begin{document}
\PMlinkescapephrase{amenable groups}
\PMlinkescapeword{cover}
\PMlinkescapeword{factors}
\PMlinkescapeword{independent}
\PMlinkescapeword{refinement}
\PMlinkescapeword{satisfy}
\PMlinkescapeword{theorem}

A group $G$ is said to be \emph{polycyclic} if it has a subnormal series
\[\{1\}=G_0\normal G_1\normal\dots\normal G_{n-1}\normal G_n=G\]
such that $G_{i+1}/G_i$ is cyclic for each $i=0,\dots,n-1$.
(Note that this differs from the definition of a supersolvable group in that it does not require each $G_i$ to be normal in $G$.)
A subnormal series of this form is called a \emph{polycyclic series}.

Polycyclic groups are obviously solvable.
In fact, the polycyclic groups are precisely those solvable groups that satisfy the maximal condition (that is, those solvable groups all of whose \PMlinkname{subgroups}{Subgroup} are finitely generated).
In particular, a finite group is polycyclic if and only if it is solvable.

The \emph{Hirsch length} (or \emph{Hirsch number}, named after \PMlinkexternal{Kurt Hirsch}{http://www-history.mcs.st-and.ac.uk/history//Biographies/Hirsch.html})
of a polycyclic group $G$ is the number of infinite factors in a polycyclic series of $G$.
This is independent of the choice of polycyclic series, as a consequence of the Schreier Refinement Theorem.
More generally, the Hirsch length of a polycyclic-by-finite group $G$ is the Hirsch length of a polycyclic normal subgroup of finite index in $G$ (all such subgroups having the same Hirsch length).
J.~A.~Hillman\cite{hillman} has further extended the concept of Hirsch length to cover all elementary amenable groups.

\begin{thebibliography}{9}
\bibitem{hillman}
 Jonathan A.~Hillman,
 \PMlinkescapetext{{\sl Elementary amenable groups and 4-manifolds 
                    with Euler characteristic 0}},
 J.\ Austral.\ Math.\ Soc.\ (Series A) 50 (1991), 160--170.
 (This paper can be viewed \PMlinkexternal{on the Australian Mathematical Society website}{http://anziamj.austms.org.au/JAMSA/V50/Part1/Hillman/p0160.html}.)
\end{thebibliography}
%%%%%
%%%%%
\end{document}
