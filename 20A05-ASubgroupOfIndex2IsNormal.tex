\documentclass[12pt]{article}
\usepackage{pmmeta}
\pmcanonicalname{ASubgroupOfIndex2IsNormal}
\pmcreated{2013-03-22 15:09:25}
\pmmodified{2013-03-22 15:09:25}
\pmowner{alozano}{2414}
\pmmodifier{alozano}{2414}
\pmtitle{a subgroup of index 2 is normal}
\pmrecord{5}{36905}
\pmprivacy{1}
\pmauthor{alozano}{2414}
\pmtype{Theorem}
\pmcomment{trigger rebuild}
\pmclassification{msc}{20A05}
\pmrelated{Coset}
\pmrelated{QuotientGroup}
\pmrelated{NormalityOfSubgroupsOfPrimeIndex}

\endmetadata

% this is the default PlanetMath preamble.  as your knowledge
% of TeX increases, you will probably want to edit this, but
% it should be fine as is for beginners.

% almost certainly you want these
\usepackage{amssymb}
\usepackage{amsmath}
\usepackage{amsthm}
\usepackage{amsfonts}

% used for TeXing text within eps files
%\usepackage{psfrag}
% need this for including graphics (\includegraphics)
%\usepackage{graphicx}
% for neatly defining theorems and propositions
%\usepackage{amsthm}
% making logically defined graphics
%%%\usepackage{xypic}

% there are many more packages, add them here as you need them

% define commands here

\newtheorem{thm}{Theorem}
\newtheorem{defn}{Definition}
\newtheorem{prop}{Proposition}
\newtheorem*{lemma}{Lemma}
\newtheorem{cor}{Corollary}

\theoremstyle{definition}
\newtheorem{exa}{Example}

% Some sets
\newcommand{\Nats}{\mathbb{N}}
\newcommand{\Ints}{\mathbb{Z}}
\newcommand{\Reals}{\mathbb{R}}
\newcommand{\Complex}{\mathbb{C}}
\newcommand{\Rats}{\mathbb{Q}}
\newcommand{\Gal}{\operatorname{Gal}}
\newcommand{\Cl}{\operatorname{Cl}}
\begin{document}
\begin{lemma}
Let $(G,\cdot)$ be a group and let $H$ be a subgroup of $G$ of index 2. Then $H$ is normal in $G$.
\end{lemma}
\begin{proof}
Let $G$ be a group and let $H$ be an index 2 subgroup of $G$. By definition of index, there are only two left cosets of $H$ in $G$, namely:
$$H, \quad g_1H$$
where $g_1$ is any element of $G$ which is not in $H$. Notice that if $g_1,\ g_2$ are two elements in $G$ which are not in $H$ then $g_1\cdot g_2$ belongs to $H$. Indeed, the coset $g_1g_2H\neq g_1H$ (because $g_1g_2=g_1h$ would immediately yield $g_2=h\in H$) and so $g_1g_2H=H$ and $g_1g_2\in H$.

Let $h\in H$ be an arbitrary element of $H$ and let $g\in G$. If $g\in H$ then $ghg^{-1}\in H$ and we are done. Otherwise, assume that $g\notin H$. Thus $gh\notin H$ and by the remark above $ghg^{-1}=(gh)g^{-1} \in H$, as desired.
\end{proof}
%%%%%
%%%%%
\end{document}
