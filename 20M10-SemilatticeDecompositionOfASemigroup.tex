\documentclass[12pt]{article}
\usepackage{pmmeta}
\pmcanonicalname{SemilatticeDecompositionOfASemigroup}
\pmcreated{2013-03-22 13:07:09}
\pmmodified{2013-03-22 13:07:09}
\pmowner{mclase}{549}
\pmmodifier{mclase}{549}
\pmtitle{semilattice decomposition of a semigroup}
\pmrecord{6}{33552}
\pmprivacy{1}
\pmauthor{mclase}{549}
\pmtype{Definition}
\pmcomment{trigger rebuild}
\pmclassification{msc}{20M10}
\pmdefines{semilattice congruence}

% this is the default PlanetMath preamble.  as your knowledge
% of TeX increases, you will probably want to edit this, but
% it should be fine as is for beginners.

% almost certainly you want these
\usepackage{amssymb}
\usepackage{amsmath}
\usepackage{amsfonts}

% used for TeXing text within eps files
%\usepackage{psfrag}
% need this for including graphics (\includegraphics)
%\usepackage{graphicx}
% for neatly defining theorems and propositions
%\usepackage{amsthm}
% making logically defined graphics
%%%\usepackage{xypic}

% there are many more packages, add them here as you need them

% define commands here
\begin{document}
A semigroup $S$ has a \emph{semilattice decomposition} if we can write $S = \bigcup_{\gamma \in \Gamma}S_\gamma$ as a disjoint union of subsemigroups, indexed by elements of a semilattice $\Gamma$, with the additional condition that $x \in S_\alpha$ and $y \in S_\beta$ implies $xy \in S_{\alpha\beta}$.

Semilattice decompositions arise from homomorphims of semigroups onto semilattices.  If $\phi \colon S \to \Gamma$ is a surjective homomorphism, then it is easy to see that we get a semilattice decomposition by putting $S_\gamma = \phi^{-1}(\gamma)$ for each $\gamma \in \Gamma$.  Conversely, every semilattice decomposition defines a map from $S$ to the indexing set $\Gamma$ which is easily seen to be a homomorphism.

A third way to look at semilattice decompositions is to consider the congruence $\rho$ defined by the homomorphism $\phi \colon S \to \Gamma$.  Because $\Gamma$ is a semilattice, $\phi(x^2) = \phi(x)$ for all $x$, and so $\rho$ satisfies the constraint that $x \,\rho\, x^2$ for all $x \in S$.  Also, $\phi(xy) = \phi(yx)$ so that $xy \,\rho\, yx$ for all $x, y \in S$.
A congruence $\rho$ which satisfies these two conditions is called a \emph{semilattice congruence}.

Conversely, a semilattice congruence $\rho$ on $S$ gives rise to a homomorphism from $S$ to a semilattice $S/\rho$.  The $\rho$-classes are the components of the decomposition.
%%%%%
%%%%%
\end{document}
