\documentclass[12pt]{article}
\usepackage{pmmeta}
\pmcanonicalname{OcticGroup}
\pmcreated{2013-03-22 14:47:59}
\pmmodified{2013-03-22 14:47:59}
\pmowner{Daume}{40}
\pmmodifier{Daume}{40}
\pmtitle{octic group}
\pmrecord{5}{36453}
\pmprivacy{1}
\pmauthor{Daume}{40}
\pmtype{Example}
\pmcomment{trigger rebuild}
\pmclassification{msc}{20F55}
\pmsynonym{$D_4$}{OcticGroup}

% this is the default PlanetMath preamble.  as your knowledge
% of TeX increases, you will probably want to edit this, but
% it should be fine as is for beginners.

% almost certainly you want these
\usepackage{amssymb}
\usepackage{amsmath}
\usepackage{amsfonts}

% used for TeXing text within eps files
%\usepackage{psfrag}
% need this for including graphics (\includegraphics)
%\usepackage{graphicx}
% for neatly defining theorems and propositions
%\usepackage{amsthm}
% making logically defined graphics
%%%\usepackage{xypic} 
\usepackage[all,2cell,ps]{xy}
\newcommand{\edge}[1]{\ar@{-}[#1]}
\newcommand{\node}{*+[o][F-]{ }}
% there are many more packages, add them here as you need them

% define commands here

% The below lines should work as the command
% \renewcommand{\bibname}{References}
% without creating havoc when rendering an entry in
% the page-image mode.
\makeatletter
\@ifundefined{bibname}{}{\renewcommand{\bibname}{References}}
\makeatother
\begin{document}
The \emph{octic group} also known as the $4^{th}$ dihedral group, is a non-Abelian group with eight elements. It is traditionally denoted by $D_4$. This group is defined by the presentation
$$< s,t \mid s^4 = t^2 = e, st = ts^{-1}>$$
or, equivalently, defined by the multiplication table
\begin{center}
\begin{tabular}{r|rrrrrrrr}
$\cdot$ & $e$ & $s$   & $s^2$ & $s^3$ & $t$ & $ts$ & $ts^2$ & $ts^3$\\
\hline
$e$     & $e$ & $s$   & $s^2$ & $s^3$ & $t$ & $ts$ & $ts^2$ & $ts^3$\\
$s$     & $s$ & $s^2$ & $s^3$ & $e$   & $ts^3$&$t$ & $ts$   & $ts^2$\\
$s^2$   & $s^2$&$s^3$ & $e$   & $s$   & $ts^2$&$ts^3$&$t$   & $ts$\\
$s^3$   & $s^3$&$e$   & $s$   & $s^2$ & $ts$  &$ts^2$&$ts^3$& $t$\\
$t$     & $t$  &$ts$  & $ts^2$& $ts^3$& $e$ & $s$  & $s^2$  & $s^3$\\
$ts$    & $ts$ &$ts^2$& $ts^3$& $t$   & $s^3$& $e$ & $s$    & $s^2$\\
$ts^2$  & $ts^2$&$ts^3$&$t$   & $ts$  & $s^2$& $s^3$&$e$    & $s$\\
$ts^3$  & $ts^3$&$t$  & $ts$  & $ts^2$& $s$  & $s^2$&$s^3$  & $e$\\ 
\end{tabular}
\end{center}
\noindent
where we have put each product $xy$ into row $x$ and column $y$.  The lattice of the subgroups is given below:
\begin{center}
\begin{figure}
\centerline{
\xymatrix{
&                       & D_4 \\
& \{ e, s^2, t, ts^2 \} \edge{ur}\edge{dr} & < s > \edge{u}\edge{d} & \{ e, s^2, st, ts\} \edge{ul}\edge{dl} \\
<ts^2> \edge{ur}\edge{drr} & <t>\edge{u}\edge{dr} & <s^2>\edge{ul}\edge{u}\edge{ur}\edge{d} & <st> \edge{u}\edge{dl} & <ts> \edge{ul}\edge{dll}\\
& & <1>\edge{ull}\edge{ul}\edge{u}\edge{ur}\edge{urr}\\                     
}
}
\end{figure}
\end{center}
where $<a_1,\ldots,a_n>$ denotes the subgroup generated by $a_1,\ldots ,a_n$ and $\{b_1,\ldots ,b_n\}$ denotes the subgroup.  Of those subgroups, the following are its proper normal subgroup: $\{e,s^2,t,ts^2\}$, $<s>$, $\{e,s^2,st,ts\}$, and $<s^2>$.  In addition the center and commutator subgroup of the octic group is $<s^2>$.  It can also be shown that the automorphism  of the octic group \textit{($\operatorname{Aut}(D_4)$)} is isomorphic to itself\textit{($D_4$)}.\cite{PJ}  An additional property is that the subgroup of the general linear group of dimension 2 over the real numbers generated by:
$$\left[ \begin{array}{cc}
0 & 1 \\
-1 & 0 
\end{array}\right],\left[ \begin{array}{cc}
0 & 1 \\
1 & 0 
\end{array}\right]$$
is isomorphic to the octic group.

\begin{thebibliography}{1}
\bibitem[PJ]{PJ} Pedersen, John: Groups of small order. \PMlinkexternal{http://www.math.usf.edu/~eclark/algctlg/small_groups.html}{http://www.math.usf.edu/~eclark/algctlg/small_groups.html}
\end{thebibliography}
%%%%%
%%%%%
\end{document}
