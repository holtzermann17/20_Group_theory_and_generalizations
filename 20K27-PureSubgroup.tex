\documentclass[12pt]{article}
\usepackage{pmmeta}
\pmcanonicalname{PureSubgroup}
\pmcreated{2013-03-22 14:57:47}
\pmmodified{2013-03-22 14:57:47}
\pmowner{CWoo}{3771}
\pmmodifier{CWoo}{3771}
\pmtitle{pure subgroup}
\pmrecord{9}{36661}
\pmprivacy{1}
\pmauthor{CWoo}{3771}
\pmtype{Definition}
\pmcomment{trigger rebuild}
\pmclassification{msc}{20K27}
\pmclassification{msc}{13C13}
\pmdefines{pure submodule}
\pmdefines{pure exact sequence}

% this is the default PlanetMath preamble.  as your knowledge
% of TeX increases, you will probably want to edit this, but
% it should be fine as is for beginners.

% almost certainly you want these
\usepackage{amssymb,amscd}
\usepackage{amsmath}
\usepackage{amsfonts}

% used for TeXing text within eps files
%\usepackage{psfrag}
% need this for including graphics (\includegraphics)
%\usepackage{graphicx}
% for neatly defining theorems and propositions
%\usepackage{amsthm}
% making logically defined graphics
%%%\usepackage{xypic}

% there are many more packages, add them here as you need them

% define commands here
\begin{document}
\textbf{Definition}.  A \emph{pure subgroup} $H$ of an abelian group
$G$ is
\begin{enumerate}
\item a subgroup of $G$, such that
\item $H\cap mG=mH$ for all $m\in\mathbb{Z}$.
\end{enumerate}
The second condition says that for any $h\in H$ such that $h=ma$ for
some integer $m$ and some $a\in G$, then there exists $b\in H$ such
that $h=mb$.  In other words, if $h$ is divisible in $G$ by an
integer, then it is divisible in $H$ by that same integer.  Purity in
abelian groups is a relative notion, and we denote $H<_pG$ to mean
that $H$ is a pure subgroup of $G$.
\par
\textbf{Examples}.  All groups mentioned below are abelian groups.
\begin{enumerate}
\item For any group, two trivial examples of pure subgroups are the trivial
subgroup and the group itself.
\item Any \PMlinkname{divisible subgroup}{DivisibleGroup} or any direct summand of a group is pure.
\item \textit{The} torsion subgroup (= the subgroup of all torsion elements) of any group is pure.
\item If $K<_pH$, $H<_pG$, then $K<_pG$.
\item If $H=\bigcup_{i=1}^{\infty}H_i$ with $H_i\leq H_{i+1}$ and
$H_i<_pG$, then $H<_pG$.
\item In $Z_{n^2}$, $\langle n \rangle$ is an example of a subgroup that is not
pure.
\item In general, $\langle m \rangle<_p Z_n$ if $\operatorname{gcd}(s,t)=1$,
where $s=\operatorname{gcd}(m,n)$ and $t=n/s$.
\end{enumerate}
\textbf{Remark}. This definition can be generalized to modules over
commutative rings.

\textbf{Definition}. Let $R$ be a commutative ring and
$\mathcal{E}\colon 0\rightarrow A\rightarrow B\rightarrow
C\rightarrow 0$ a short exact sequence of $R$-modules.  Then
$\mathcal{E}$ is said to be \emph{pure} if it remains exact after
tensoring with any $R$-module.  In other words, if $D$ is any
$R$-module, then
$$D\otimes\mathcal{E}\colon0\rightarrow D\otimes A\rightarrow D\otimes B\rightarrow D\otimes C\rightarrow
0,$$ is exact.

\textbf{Definition}. Let $N$ be a submodule of $M$ over a ring $R$.
Then $N$ is said to be a \emph{pure submodule} of $M$ if the exact
sequence $$0\rightarrow N\rightarrow M\rightarrow M/N\rightarrow 0$$
is a pure exact sequence.

From this definition, it is clear that $H$ is a pure subgroup of $G$
iff $H$ is a pure $\mathbb{Z}$-submodule of $G$.

\textbf{Remark}.  $N$ is a pure submodule of $M$ over $R$ iff
whenever a finite sum $$\sum r_im_i=n\in N,$$ where $m_i\in M$ and
$r_i\in R$ implies that $$n=\sum r_in_i$$ for some $n_i\in N$.  As a
result, if $I$ is an ideal of $R$, then the purity of $N$ in $M$ means
that $N\cap IM=IN$, which is a generalization of the second
condition in the definition of a pure subgroup above.
%%%%%
%%%%%
\end{document}
