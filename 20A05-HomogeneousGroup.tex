\documentclass[12pt]{article}
\usepackage{pmmeta}
\pmcanonicalname{HomogeneousGroup}
\pmcreated{2013-03-22 16:12:12}
\pmmodified{2013-03-22 16:12:12}
\pmowner{whm22}{2009}
\pmmodifier{whm22}{2009}
\pmtitle{homogeneous group}
\pmrecord{15}{38298}
\pmprivacy{1}
\pmauthor{whm22}{2009}
\pmtype{Definition}
\pmcomment{trigger rebuild}
\pmclassification{msc}{20A05}
\pmdefines{homomorphism of homogeneous groups}

\endmetadata

% this is the default PlanetMath preamble.  as your knowledge
% of TeX increases, you will probably want to edit this, but
% it should be fine as is for beginners.

% almost certainly you want these
\usepackage{amssymb}
\usepackage{amsmath}
\usepackage{amsfonts}

% used for TeXing text within eps files
%\usepackage{psfrag}
% need this for including graphics (\includegraphics)
%\usepackage{graphicx}
% for neatly defining theorems and propositions
%\usepackage{amsthm}
% making logically defined graphics
%%%\usepackage{xypic}

% there are many more packages, add them here as you need them

% define commands here

\begin{document}
A \emph{homogeneous group} is a set $G$ together with a map $():G \times G \times G \to G$ satisfying:

i)$(a,a,b)=b$

ii)$(a,b,b)=a$

iii)$((a,b,c),d,e)=(a,b,(c,d,e))$  

for all $a,b,c,d,e \in G$.

A map $f:G \to H$ of homogeneous groups is a homomorphism if it \PMlinkescapetext{satisfies} $f(a,b,c)=(fa,fb,fc)$, for all $a,b,c \in G$.

A non-empty homogeneous group is essentially a group, as given any $x \in G$, we may define the following product on $G$:

$ab=(a,x,b)$.

This gives $G$ the \PMlinkescapetext{structure} of a group with identity $x$.  The choice of $x$ does not affect the isomorphism class of the group obtained.

One may recover a homogeneous group from a group obtained this way, by setting

$(a,b,c)=ab^{-1}c$. 

Also, every group may be obtained from a homogeneous group.

Homogeneous groups are homogeneous:  Given $a,b \in G$ we have a homomorphism $f$ taking $a$ to $b$, given by $fx=(x,a,b)$.                

In this way homogeneous groups differ from groups, as whilst often used to describe symmetry, groups themselves have a distinct element: the identity.

Also the definition of homogeneous groups is given purely in 
\PMlinkescapetext{terms} of identities, and does not exclude the empty set.
%%%%%
%%%%%
\end{document}
