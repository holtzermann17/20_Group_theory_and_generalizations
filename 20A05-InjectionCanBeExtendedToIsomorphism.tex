\documentclass[12pt]{article}
\usepackage{pmmeta}
\pmcanonicalname{InjectionCanBeExtendedToIsomorphism}
\pmcreated{2013-03-22 18:56:50}
\pmmodified{2013-03-22 18:56:50}
\pmowner{pahio}{2872}
\pmmodifier{pahio}{2872}
\pmtitle{injection can be extended to isomorphism}
\pmrecord{10}{41804}
\pmprivacy{1}
\pmauthor{pahio}{2872}
\pmtype{Theorem}
\pmcomment{trigger rebuild}
\pmclassification{msc}{20A05}
\pmclassification{msc}{03E20}
\pmrelated{Restriction}
\pmrelated{Cardinality}

% this is the default PlanetMath preamble.  as your knowledge
% of TeX increases, you will probably want to edit this, but
% it should be fine as is for beginners.

% almost certainly you want these
\usepackage{amssymb}
\usepackage{amsmath}
\usepackage{amsfonts}

% used for TeXing text within eps files
%\usepackage{psfrag}
% need this for including graphics (\includegraphics)
%\usepackage{graphicx}
% for neatly defining theorems and propositions
 \usepackage{amsthm}
% making logically defined graphics
%%%\usepackage{xypic}

% there are many more packages, add them here as you need them

% define commands here

\theoremstyle{definition}
\newtheorem*{thmplain}{Theorem}
\DeclareMathOperator{\card}{card}
\begin{document}
\PMlinkescapeword{right}

\textbf{Theorem.}\, If $f$ is an injection from a set $S$ into a group $G$, then there exist a group $H$ containing $S$ and a group isomorphism \;$\varphi\!:H \to G$\; such that\; $\varphi|_S = f$.\\

\emph{Proof.}\, Let $M$ be a set such that\, $\card(M) \geqq \card(G)$.\, Because\, $\card(f(S)) = \card(S)$, we have\, 
$\card(M\!\smallsetminus\!S) \geqq \card(G\!\smallsetminus\!f(S))$,\, and therefore there exists an injection
$$\psi\!:G\!\smallsetminus\!f(S) \to M\!\smallsetminus\!S$$
(provided that\, $G\!\smallsetminus\!f(S) \neq \varnothing$;\, otherwise the mapping \,$f\!:S \to G$ would be a bijection).\, Define
$$H \;:=\; S\cup\psi(G\!\smallsetminus\!f(S)),$$
\begin{align*}
\varphi(h) \;:=\; 
\begin{cases}
f(h) \qquad\mbox{for}\;\; h \in S,\\
\psi^{-1}(h) \quad\mbox{for}\;\, h \in H\!\smallsetminus\!S.
\end{cases}
\end{align*}
Then apparently,\, $\varphi\!:H \to G$\, is a bijection and\, $\varphi|_S = f$.\, Moreover, define the binary operation ``$*$'' of the set $H$ by
\begin{align}
h_1\ast h_2 \;:=\; \varphi^{-1}(\varphi(h_1)\!\cdot\!\varphi(h_2)).
\end{align}
We see first that
\begin{align*}
(h_1\ast h_2)\ast h_3 &\;=\; 
\varphi^{-1}\!\left(\varphi\left(\varphi^{-1}(\varphi(h_1)\!\cdot\!\varphi(h_2))\right)\!\cdot\!\varphi(h_3)\right)\\
                      &\;=\; \varphi^{-1}((\varphi(h_1)\!\cdot\!\varphi(h_2))\!\cdot\!\varphi(h_3))\\
                      &\;=\; \varphi^{-1}(\varphi(h_1)\!\cdot\!(\varphi(h_2)\!\cdot\!\varphi(h_3)))\\ &\;=\; 
\varphi^{-1}\!\left(\varphi(h_1)\!\cdot\!\varphi\left(\varphi^{-1}(\varphi(h_2)\!\cdot\!\varphi(h_3))\right)\right)\\
                      &\;=\; h_1\ast(h_2\ast h_3).
\end{align*}
Secondly,
$$h\ast\varphi^{-1}(e) \;=\; 
\varphi^{-1}\!\left(\varphi(h)\!\cdot\!\varphi(\varphi^{-1}(e))\right) \;=\; \varphi^{-1}(\varphi(h)) \;=\; h,$$
whence $\varphi^{-1}(e)$ is the right identity element of $H$.\, Then,
$$h\ast \varphi^{-1}\!\left((\varphi(h))^{-1}\right) 
\;=\; \varphi^{-1}\left(\varphi(h)\!\cdot\!\varphi\left(\varphi^{-1}(\varphi(h)^{-1})\right)\right) \;=\; \varphi^{-1}(e),$$
and accordingly $\displaystyle\varphi^{-1}\!\left((\varphi(h))^{-1}\right)$ is the right inverse of $h$ in $H$.\, Consequently, 
$(H,\,\ast)$ is a group.\, The equation (1) implies that
$$\varphi(h_1\ast h_2) \;=\; \varphi(h_1)\!\cdot\!\varphi(h_2),$$
whence $\varphi$ is an isomorphism from $H$ onto $G$.\, Q.E.D.

%%%%%
%%%%%
\end{document}
