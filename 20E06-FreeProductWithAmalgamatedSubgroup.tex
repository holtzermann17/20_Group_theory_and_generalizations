\documentclass[12pt]{article}
\usepackage{pmmeta}
\pmcanonicalname{FreeProductWithAmalgamatedSubgroup}
\pmcreated{2013-03-22 13:24:09}
\pmmodified{2013-03-22 13:24:09}
\pmowner{mathcam}{2727}
\pmmodifier{mathcam}{2727}
\pmtitle{free product with amalgamated subgroup}
\pmrecord{7}{33944}
\pmprivacy{1}
\pmauthor{mathcam}{2727}
\pmtype{Definition}
\pmcomment{trigger rebuild}
\pmclassification{msc}{20E06}
\pmsynonym{amalgam}{FreeProductWithAmalgamatedSubgroup}
\pmsynonym{sum with amalgamated subgroup}{FreeProductWithAmalgamatedSubgroup}
\pmsynonym{amalgamated free product}{FreeProductWithAmalgamatedSubgroup}
\pmrelated{AmalgamationProperty}
\pmrelated{CategoricalPullback}
\pmrelated{FreeProduct}
\pmdefines{pushout of groups}

\endmetadata

%\documentclass{amsart}
\usepackage{amsmath}
\usepackage[all,poly,knot,dvips]{xy}
%\usepackage{pstricks,pst-poly,pst-node,pstcol}


\usepackage{amssymb,latexsym}

\usepackage{amsthm,latexsym}
\usepackage{eucal,latexsym}

% THEOREM Environments --------------------------------------------------

\newtheorem{thm}{Theorem}
 \newtheorem*{mainthm}{Main~Theorem}
 \newtheorem{cor}[thm]{Corollary}
 \newtheorem{lem}[thm]{Lemma}
 \newtheorem{prop}[thm]{Proposition}
 \newtheorem{claim}[thm]{Claim}
 \theoremstyle{definition}
 \newtheorem{defn}[thm]{Definition}
 \theoremstyle{remark}
 \newtheorem{rem}[thm]{Remark}
 \numberwithin{equation}{subsection}


%---------------------  Greek letters, etc ------------------------- 

\newcommand{\CA}{\mathcal{A}}
\newcommand{\CC}{\mathcal{C}}
\newcommand{\CM}{\mathcal{M}}
\newcommand{\CP}{\mathcal{P}}
\newcommand{\CS}{\mathcal{S}}
\newcommand{\BC}{\mathbb{C}}
\newcommand{\BN}{\mathbb{N}}
\newcommand{\BR}{\mathbb{R}}
\newcommand{\BZ}{\mathbb{Z}}
\newcommand{\FF}{\mathfrak{F}}
\newcommand{\FL}{\mathfrak{L}}
\newcommand{\FM}{\mathfrak{M}}
\newcommand{\Ga}{\alpha}
\newcommand{\Gb}{\beta}
\newcommand{\Gg}{\gamma}
\newcommand{\GG}{\Gamma}
\newcommand{\Gd}{\delta}
\newcommand{\GD}{\Delta}
\newcommand{\Ge}{\varepsilon}
\newcommand{\Gz}{\zeta}
\newcommand{\Gh}{\eta}
\newcommand{\Gq}{\theta}
\newcommand{\GQ}{\Theta}
\newcommand{\Gi}{\iota}
\newcommand{\Gk}{\kappa}
\newcommand{\Gl}{\lambda}
\newcommand{\GL}{\Lamda}
\newcommand{\Gm}{\mu}
\newcommand{\Gn}{\nu}
\newcommand{\Gx}{\xi}
\newcommand{\GX}{\Xi}
\newcommand{\Gp}{\pi}
\newcommand{\GP}{\Pi}
\newcommand{\Gr}{\rho}
\newcommand{\Gs}{\sigma}
\newcommand{\GS}{\Sigma}
\newcommand{\Gt}{\tau}
\newcommand{\Gu}{\upsilon}
\newcommand{\GU}{\Upsilon}
\newcommand{\Gf}{\varphi}
\newcommand{\GF}{\Phi}
\newcommand{\Gc}{\chi}
\newcommand{\Gy}{\psi}
\newcommand{\GY}{\Psi}
\newcommand{\Gw}{\omega}
\newcommand{\GW}{\Omega}
\newcommand{\Gee}{\epsilon}
\newcommand{\Gpp}{\varpi}
\newcommand{\Grr}{\varrho}
\newcommand{\Gff}{\phi}
\newcommand{\Gss}{\varsigma}

\def\co{\colon\thinspace}
\begin{document}
\PMlinkescapeword{example}
\PMlinkescapeword{examples}
\PMlinkescapeword{free product}
\PMlinkescapeword{free products}

\begin{defn}
Let $G_k$, $k=0,1,2$ be groups and $i_k\co G_0\to G_i$, $k=1,2$  be monomorphisms.
The free product of $G_1$ and $G_2$ with amalgamated subgroup $G_0$,
 is defined to be a group $G$ that has the following 
 two properties
 \begin{enumerate}
 \item there are homomorphisms $j_k\co G_k\to G$,  $k=1,2$ that make the following
 diagram commute
$$\xymatrix{
&{G_1}\ar[dr]^{j_1}&\\
{G_0}\ar[ur]^{i_1}\ar[dr]_{i_2} & & {G}\\
&{G_2}\ar[ur]_{j_2}& 
}
$$
\item $G$  is universal with respect to the previous  property, that is for
  any other group $G'$ and  homomorphisms $j_k'\co G_k\to G'$,  $k=1,2$ that
  fit in such a commutative diagram there is a unique homomorphism $G\to G'$
so that the following diagram commutes
   $$\xymatrix{
&{G_1}\ar[dr]^{j_1}\ar@/^1pc/[drr]^{j_1'}& &\\
{G_0}\ar[ur]^{i_1}\ar[dr]_{i_2} & & {G}\ar[r]^{!}&{G'}\\
&{G_2}\ar[ur]_{j_2}\ar@/_1pc/[urr]_{j_2'}& &
}
$$
 \end{enumerate}
\end{defn}

It follows by ``general nonsense'' that the  free product 
of $G_1$ and $G_2$ with amalgamated subgroup $G_0$, if it exists,
is ``unique up to unique isomorphism.''  The  free product 
of $G_1$ and $G_2$ with amalgamated subgroup $G_0$, is denoted 
by $G_1\bigstar_{G_0} G_2$. The following theorem asserts its existence.

\begin{thm}
$G_1\bigstar_{G_0} G_2$ exists for any groups $G_k$, $k=0,1,2$  and
  monomorphisms $i_k\co G_0\to G_i$,
 $k=1,2$.
 \end{thm}

 \begin{proof}[\textbf{Sketch of proof}]
   Without loss of generality assume that $G_0$ is a subgroup of $G_k$ and
that $i_k$ is the inclusion for $k=1,2$. Let 
$$G_k=\langle (x_{k;s})_{s\in S}\, |\, (r_{k;t})_{t\in T} \rangle$$
be a presentation of $G_k$ for $k=1,2$.  Each $g\in G_0$ can be expressed as
a word in the generators of $G_k$; denote that word by $w_k(g)$ and let
$N$ be the normal closure of $\{w_1(g)w_2(g)^{-1}\,|\,g\in G_0\}$ in the \PMlinkname{free product}{FreeProduct} $G_1\bigstar G_2$. Define 
   $$G_1\bigstar_{G_0} G_2:=G_1\bigstar G_2/N\,$$
and for $k=0,1$ define $j_k$ to be the inclusion into the free product
followed by the canonical projection. Clearly (1) is satisfied, while (2)
follows from the universal properties of the free product and the quotient group. 
 \end{proof}

Notice that in the above proof it would be sufficient to divide by the relations 
$w_1(g)w_2(g)^{-1}$ for $g$ in a generating set of $G_0$. This
is useful in practice when one is interested in obtaining a presentation of 
$G_1\bigstar_{G_0} G_2$.

In case that the  $i_k$'s are not injective the above still goes through
verbatim. The group thusly obtained is called a ``pushout''.

Examples of free products with amalgamated subgroups are provided by Van
Kampen's theorem.
%%%%%
%%%%%
\end{document}
