\documentclass[12pt]{article}
\usepackage{pmmeta}
\pmcanonicalname{ChernikovGroup}
\pmcreated{2013-03-22 15:48:21}
\pmmodified{2013-03-22 15:48:21}
\pmowner{yark}{2760}
\pmmodifier{yark}{2760}
\pmtitle{Chernikov group}
\pmrecord{8}{37769}
\pmprivacy{1}
\pmauthor{yark}{2760}
\pmtype{Definition}
\pmcomment{trigger rebuild}
\pmclassification{msc}{20F50}
\pmsynonym{\v{C}ernikov group}{ChernikovGroup}
\pmrelated{MinimalCondition}

\usepackage{amsthm}
\newtheorem*{thm*}{Theorem}

\begin{document}
\PMlinkescapeword{property}
\PMlinkescapeword{satisfies}
\PMlinkescapeword{satisfy}
\PMlinkescapeword{state}
\PMlinkescapeword{theorem}

A \emph{Chernikov group} is a group $G$ that has a normal subgroup $N$ 
such that $G/N$ is finite 
and $N$ is a direct product of finitely many quasicyclic groups.

The significance of this somewhat arbitrary-looking definition is that all such groups satisfy the minimal condition, and for a long time they were the only known groups with this property.

Chernikov groups are named after \PMlinkexternal{Sergei Chernikov}{http://www-groups.dcs.st-and.ac.uk/~history/Biographies/Chernikov.html},
who proved that every solvable group that satisfies the minimal condition
is a Chernikov group.
We can state this result in the form of the following theorem.

\begin{thm*}
The following are equivalent for a group $G$:
\begin{itemize}
\item $G$ is a Chernikov group.
\item $G$ is virtually abelian and satisfies the minimal condition.
\item $G$ is virtually solvable and satisfies the minimal condition.
\end{itemize}
\end{thm*}
%%%%%
%%%%%
\end{document}
