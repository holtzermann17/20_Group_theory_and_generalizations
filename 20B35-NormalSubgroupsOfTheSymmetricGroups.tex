\documentclass[12pt]{article}
\usepackage{pmmeta}
\pmcanonicalname{NormalSubgroupsOfTheSymmetricGroups}
\pmcreated{2013-03-22 17:31:38}
\pmmodified{2013-03-22 17:31:38}
\pmowner{rm50}{10146}
\pmmodifier{rm50}{10146}
\pmtitle{normal subgroups of the symmetric groups}
\pmrecord{8}{39923}
\pmprivacy{1}
\pmauthor{rm50}{10146}
\pmtype{Theorem}
\pmcomment{trigger rebuild}
\pmclassification{msc}{20B35}
\pmclassification{msc}{20E07}
\pmclassification{msc}{20B30}

% this is the default PlanetMath preamble.  as your knowledge
% of TeX increases, you will probably want to edit this, but
% it should be fine as is for beginners.

% almost certainly you want these
\usepackage{amssymb}
\usepackage{amsmath}
\usepackage{amsfonts}

% used for TeXing text within eps files
%\usepackage{psfrag}
% need this for including graphics (\includegraphics)
%\usepackage{graphicx}
% for neatly defining theorems and propositions
\usepackage{amsthm}
% making logically defined graphics
%%%\usepackage{xypic}

% there are many more packages, add them here as you need them

% define commands here
\newtheorem{thm}{Theorem}
\begin{document}
\PMlinkescapeword{addition}
\PMlinkescapeword{arguments}

\begin{thm} For $n\geq 5$, $A_n$ is the only proper nontrivial normal subgroup of $S_n$.
\end{thm}

\begin{proof} This is essentially a corollary of the simplicity of the alternating groups $A_n$ for $n\geq 5$. Let $N\unlhd S_n$ be normal. Clearly $N\cap A_n\unlhd A_n$. But $A_n$ is simple, so $N\cap A_n=A_n$ or $N\cap A_n=\{e\}$. In the first case, either $N=A_n$, or else $N$ also contains an \PMlinkname{odd}{SignatureOfAPermutation} permutation, in which case $N=S_n$. In the second case, either $N=\{e\}$ or else $N$ consists solely of one or more odd permutations in addition to $\{e\}$. But if $N$ contains two distinct odd permutations, $\sigma$ and $\tau$, then either $\sigma^2\neq e$ or $\sigma\tau\neq e$, and both $\sigma^2$ and $\sigma\tau$ are \PMlinkname{even}{SignatureOfAPermutation}, contradicting the assumption that $N$ contains only odd nontrivial permutations. Thus $N$ must be of order $2$, consisting of a single odd permutation of order 2 together with the identity.

It is easy to see, however, that such a subgroup cannot be normal. An odd permutation of order $2$, $\sigma$, has as its cycle decomposition one or more (an odd number, in fact, though this does not matter here) of disjoint transpositions. Suppose wlog that $(1~2)$ is one of these transpositions. Then $\tau=(1~3)\sigma(1~3)=(1~3)(1~2)(\ldots)(1~3)$ takes $2$ to $3$ and thus is neither $\sigma$ nor $e$. So this group is not normal.
\end{proof}

If $n=1$, $S_1$ is the trivial group, so it has no nontrivial [normal] subgroups.

If $n=2$, $S_2=C_2$, the unique group on $2$ elements, so it has no nontrivial [normal] subgroups.

If $n=3$, $S_3$ has one nontrivial proper normal subgroup, namely the group generated by $(1~2~3)$.

$S_4$ is the most interesting case for $n\leq 5$. The arguments in the theorem above do not apply since $A_4$ is not simple. Recall that a normal subgroup must be a union of conjugacy classes of elements, and that conjugate elements in $S_n$ have the same cycle type. If we examine the sizes of the various conjugacy classes of $S_4$, we get
\begin{center}
\begin{tabular}{c|c}
Cycle Type & Size\\
\hline
4 & 6\\
3,1 & 8\\
2,2 & 3\\
2,1,1 & 6\\
1,1,1,1 & 1
\end{tabular}
\end{center}
A subgroup of $S_4$ must be of order $1,2,3,4,6,8$, or $12$ (the factors of $\lvert S_4\rvert=24$). Since each subgroup must contain $\{e\}$, it is easy to see that the only possible nontrivial normal subgroups have orders $4$ and $12$. The order $4$ subgroup is $H=\{e,(1~2)(3~4),(1~3)(2~4),(1~4)(2~3)\}$, while the order $12$ subgroup is $A_4$. $A_4$ is obviously normal, being of index $2$, and one can easily check that $H\cong V_4$ is also normal in $S_4$. So these are the only two nontrivial proper normal subgroups of $S_4$.


%%%%%
%%%%%
\end{document}
